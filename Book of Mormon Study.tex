\documentclass[12pt]{report}

\usepackage{cleveref}
\usepackage[margin=1in]{geometry}
\usepackage{makeidx}


\author{Jarin French}
\title{Book of Mormon Study}
\makeindex

%------------------------------------------------------------------------------------------
\iffalse
%%%%%%%%%%%%%%%%%%%%%%%%%%%%%%%%%%%%%%%%%%%%%%%%%%%%%%%%%%%%%%%%%%%%%%%%%%%%%%%%%%%%%%%%%%%
%%%%%%%%%%%%%%%%%%%%%%%%%%%%%%%%%%%%%%%%%%%%%%%%%%%%%%%%%%%%%%%%%%%%%%%%%%%%%%%%%%%%%%%%%%%
\chapter{1st Paragraph\label{example_chapter}}
\begin{center}
\quote{``"}
\end{center}

\begin{table}[h!]
\centering
\label{table:example_table}
\begin{tabular*}{\textwidth}{c @{\extracolsep{\fill}}cc}
Speaker & Important Characters & Target Audience \\
\hline
\rule{0pt}{3ex} --- & --- & --- 
\end{tabular*}
\end{table}

\section{Definitions\label{example_definition}}
\emph{Word}: \begin{itemize}
\item \emph{Definitions 1-n}
\end{itemize}
\section{Principles\label{example_principle}}
\begin{itemize}
\item \index{test}\emph{None} (yet)
\end{itemize}

\section{Comments\label{example_comment}}

\section{Additional References\label{example_references}}
\begin{itemize}
\item \emph{None} (yet)
\end{itemize}
%%%%%%%%%%%%%%%%%%%%%%%%%%%%%%%%%%%%%%%%%%%%%%%%%%%%%%%%%%%%%%%%%%%%%%%%%%%%%%%%%%%%%%%%%%%
%%%%%%%%%%%%%%%%%%%%%%%%%%%%%%%%%%%%%%%%%%%%%%%%%%%%%%%%%%%%%%%%%%%%%%%%%%%%%%%%%%%%%%%%%%%
\fi
%------------------------------------------------------------------------------------------

\begin{document}
\maketitle
\tableofcontents

\part*{Title Page of the Book of Mormon\label{BoM:titlePage}}
%%%%%%%%%%%%%%%%%%%%%%%%%%%%%%%%%%%%%%%%%%%%%%%%%%%%%%%%%%%%%%%%%%%%%%%%%%%%%%%%%%%%%%%%%%%
%%%%%%%%%%%%%%%%%%%%%%%%%%%%%%%%%%%%%%%%%%%%%%%%%%%%%%%%%%%%%%%%%%%%%%%%%%%%%%%%%%%%%%%%%%%
\chapter{1st Paragraph\label{titlePage:1st}}
\begin{center}
\quote{``The Book of Mormon: An account written by the hand of Mormon upon plates taken from the plates of Nephi."}
\end{center}

\begin{table}[h!]
\centering
\label{table:titlePage1}
\begin{tabular*}{\textwidth}{c @{\extracolsep{\fill}}cc}
Speaker & Important Characters & Target Audience \\
\hline
\rule{0pt}{3ex}Mormon & --- & The world 
\end{tabular*}
\end{table}

\section{Definitions\label{titlePage:DFN1}}
\section{Principles\label{titlePage:principles1}}
\begin{itemize}
\item \index{Record Keeping}Record keeping
\end{itemize}

\section{Comments\label{titlePage:comments1}}
Mormon summarized the entirety of Lehi's descendents' history, and used the plates that Nephi started (approximately 1000 years previously).

\section{Additional References\label{titlePage:references1}}
\begin{itemize}
\item \emph{None} (yet)
\end{itemize}
%%%%%%%%%%%%%%%%%%%%%%%%%%%%%%%%%%%%%%%%%%%%%%%%%%%%%%%%%%%%%%%%%%%%%%%%%%%%%%%%%%%%%%%%%%%
%%%%%%%%%%%%%%%%%%%%%%%%%%%%%%%%%%%%%%%%%%%%%%%%%%%%%%%%%%%%%%%%%%%%%%%%%%%%%%%%%%%%%%%%%%%

%%%%%%%%%%%%%%%%%%%%%%%%%%%%%%%%%%%%%%%%%%%%%%%%%%%%%%%%%%%%%%%%%%%%%%%%%%%%%%%%%%%%%%%%%%%
%%%%%%%%%%%%%%%%%%%%%%%%%%%%%%%%%%%%%%%%%%%%%%%%%%%%%%%%%%%%%%%%%%%%%%%%%%%%%%%%%%%%%%%%%%%
\chapter{2nd Paragraph\label{titlePage:2nd}}
\begin{center}
\quote{``Wherefore, it is an abridgment of the record of the people of Nephi, and also of the Lamanites -- Written to the Lamanites, who are a remnant of the house of Israel; and also to Jew and Gentile -- Written by way of commandment, and also by the spirit of prophecy and of revelation -- Written and sealed up, and hid up unto the Lord, that they might not be destroyed -- To come forth by the gift and power of God unto the interpretation thereof -- Sealed by the hand of Moroni, and hid up unto the Lord, to come forth in due time by way of the Gentile -- The interpretation thereof by the gift of God."}
\end{center}

\begin{table}[h!]
\centering
\label{table:titlePage2}
\begin{tabular*}{\textwidth}{c@{\extracolsep{\fill}}cc}
Speaker & Important Characters & Target Audience \\
\hline
\rule{0pt}{3ex}Mormon & People of Nephi; Lamanites; Jew; Gentile & Lamanites; Jews; Gentiles 
\end{tabular*}
\end{table}

\section{Definitions\label{titlePage:DFN2}}
\emph{Abridge}: \begin{itemize}
\item to shorten by omissions while retaining the basic contents
\item to reduce or lessen in duration, scope, authority, etc.; diminish; curtail
\item to deprive; cut off
\end{itemize}
\section{Principles\label{titlePage:principles2}}
\begin{itemize}
\item \emph{None} (yet)
\end{itemize}

\section{Comments\label{titlePage:comments2}}
This book was written because of the spirit of prophecy, which as we find in Revelation 19:10 is the testimony of Christ, but it was also written because it was a commandment.  This leads to the thought that as we develop a testimony of Christ, we will be commanded to write our own testimony so that others can learn from our own experience.

\section{Additional References\label{titlePage:references2}}
\begin{itemize}
\item Revelation 19:10
\end{itemize}
%%%%%%%%%%%%%%%%%%%%%%%%%%%%%%%%%%%%%%%%%%%%%%%%%%%%%%%%%%%%%%%%%%%%%%%%%%%%%%%%%%%%%%%%%%%
%%%%%%%%%%%%%%%%%%%%%%%%%%%%%%%%%%%%%%%%%%%%%%%%%%%%%%%%%%%%%%%%%%%%%%%%%%%%%%%%%%%%%%%%%%%

%%%%%%%%%%%%%%%%%%%%%%%%%%%%%%%%%%%%%%%%%%%%%%%%%%%%%%%%%%%%%%%%%%%%%%%%%%%%%%%%%%%%%%%%%%%
%%%%%%%%%%%%%%%%%%%%%%%%%%%%%%%%%%%%%%%%%%%%%%%%%%%%%%%%%%%%%%%%%%%%%%%%%%%%%%%%%%%%%%%%%%%
\chapter{3rd Paragraph\label{titlePage:3rd}}
\begin{center}
\quote{``An abridgment taken from the Book of Ether also, which is a record of the people of Jared, who were scattered at the time the Lord confounded the language of the people, when they were building a tower to get to heaven - Which is to shown unto the remnant of the House of Israel what great things the Lord hath done for their fathers; and that they may know the covenants of the Lord, that they are not cast off forever - And also to the convincing of the Jew and Gentile that JESUS is the CHRIST, the ETERNAL GOD, manifesting himself unto all nations - and now, if there are faults they are the mistakes of men; wherefore, condemn not the things of God, that ye may be found spotless at the judgment-seat of Christ."}
\end{center}

\begin{table}[h!]
\centering
\label{table:titlePage3}
\begin{tabular*}{\textwidth}{c @{\extracolsep{\fill}}cc}
Speaker & Important Characters & Target Audience \\
\hline
\rule{0pt}{3ex}Mormon & People of Jared; House of Israel; Jew; Gentile & Lamanites; Jews; Gentiles 
\end{tabular*}
\end{table}

\section{Definitions\label{titlePage:DFN3}}
\emph{Confound}: \begin{itemize}
\item to perplex or amaze, especially by sudden disturbance or surprise; bewilder; confuse
\item to throw into confusion or disorder
\item to thrown into increased confusion or disorder
\item to treat or regard erroneously as identical; mix or associate by mistake
\item to mingle so that the elements cannot be distinguished or separated
\item to damn (used in mild imprecations)
\item to contradict or refute
\end{itemize}
\section{Principles\label{titlePage:principles3}}
\begin{itemize}
\item \index{Christ}Jesus is the Christ
\end{itemize}

\section{Comments\label{titlePage_comments3}}
Including the abridgement of the Book of Ether is done to show that God is powerful, and watches out for those that serve Him.  Furthermore, this record gives another witness of the importance of the covenants of the Lord - without them, we are cast off forever.  The final statement in this paragraph is a warning to those that would find fault with the book.  Mormon readily acknowledges that he may have made a mistake, and implores those that read and ponder this book that they learn from his mistakes, rather than condemn the book because of his weakness.

Joseph Smith stated that ``The standard of truth has been erected; no unhallowed hand can stop the work from progressing; persecutions may rage, mobs may combine, armies may assemble, calumny may defame, but the truth of God will go forth boldly, nobly, and independent, till it has penetrated every continent, visited every clime, swept every country, and sounded in every ear, till the purposes of God shall be accomplished, and the Great Jehovah shall say the work is done."  We have a work to do, and this book is the way to do it!

\section{Additional References\label{titlePage:references3}}
\begin{itemize}
\item Mormon 8:17
\end{itemize}
%%%%%%%%%%%%%%%%%%%%%%%%%%%%%%%%%%%%%%%%%%%%%%%%%%%%%%%%%%%%%%%%%%%%%%%%%%%%%%%%%%%%%%%%%%%
%%%%%%%%%%%%%%%%%%%%%%%%%%%%%%%%%%%%%%%%%%%%%%%%%%%%%%%%%%%%%%%%%%%%%%%%%%%%%%%%%%%%%%%%%%%

%%%%%%%%%%%%%%%%%%%%%%%%%%%%%%%%%%%%%%%%%%%%%%%%%%%%%%%%%%%%%%%%%%%%%%%%%%%%%%%%%%%%%%%%%%%
%%%%%%%%%%%%%%%%%%%%%%%%%%%%%%%%%%%%%%%%%%%%%%%%%%%%%%%%%%%%%%%%%%%%%%%%%%%%%%%%%%%%%%%%%%%
\chapter{4th Paragraph\label{titlePage:4th}}
\begin{center}
\quote{``Translated by Joseph Smth, Jun."}
\end{center}

\begin{table}[h!]
\centering
\label{table:titlePage4}
\begin{tabular*}{\textwidth}{c @{\extracolsep{\fill}}cc}
Speaker & Important Characters & Target Audience \\
\hline
\rule{0pt}{3ex} Joseph Smith Junior & Joseph Smith Junior & The world 
\end{tabular*}
\end{table}

\section{Definitions\label{titlePage:DFN4}}
\emph{Translate}: \begin{itemize}
\item to turn from one language into another or from a foreign language into one's own
\item to change the form, condition, nature, etc., of; transform; convert
\item to explain in terms that can be more easily understood; interpret
\item to bear, carry, or move from one place, position, etc., to another; transfer
\item \emph{Mechanics} to cause (a body) to move without rotation or angular displacement
\item \emph{Computers} to convert (a program, data, code, etc.) from one form to another
\item \emph{Telegraphy} to retransmit or forward (a message), as by a relay.
\end{itemize}
\section{Principles\label{titlePage:principles4}}
\begin{itemize}
\item \index{test}\emph{None} (yet)
\end{itemize}

\section{Comments\label{titlePage:comments4}}
Joseph Smith Junior was called by God to translate the Book of Mormon - not write it.  

\section{Additional References\label{titlePage:references4}}
\begin{itemize}
\item Joseph Smith - History 1:67-68
\end{itemize}
%%%%%%%%%%%%%%%%%%%%%%%%%%%%%%%%%%%%%%%%%%%%%%%%%%%%%%%%%%%%%%%%%%%%%%%%%%%%%%%%%%%%%%%%%%%
%%%%%%%%%%%%%%%%%%%%%%%%%%%%%%%%%%%%%%%%%%%%%%%%%%%%%%%%%%%%%%%%%%%%%%%%%%%%%%%%%%%%%%%%%%%

\part{Introduction to the Book of Mormon\label{BoM:intro}}
%%%%%%%%%%%%%%%%%%%%%%%%%%%%%%%%%%%%%%%%%%%%%%%%%%%%%%%%%%%%%%%%%%%%%%%%%%%%%%%%%%%%%%%%%%%
%%%%%%%%%%%%%%%%%%%%%%%%%%%%%%%%%%%%%%%%%%%%%%%%%%%%%%%%%%%%%%%%%%%%%%%%%%%%%%%%%%%%%%%%%%%
\chapter{1st Paragraph\label{intro:1st}}
\begin{center}
\quote{``The Book of Mormon is a volume of holy scripture comparable to the Bible. It is a record of God's dealings with the ancient inhabitants of the Americas and contains, as does the Bible, the fulness of the everlasting gospel."}
\end{center}

\begin{table}[h!]
\centering
\label{table:intro1}
\begin{tabular*}{\textwidth}{c @{\extracolsep{\fill}}cc}
Speaker & Important Characters & Target Audience \\
\hline
\rule{0pt}{3ex} Bruce R. McKonkie & Ancient inhabitants of the Americas & The world 
\end{tabular*}
\end{table}

\section{Definitions\label{intro:DFN1}}
\emph{Comparable} 
\begin{itemize}
  \item capable of being compared; having features in common with something else to permit or suggest comparison
  \item worthy of comparison
  \item usable for comparison; similar
\end{itemize}
\emph{Fulness} (or \emph{Fullness}) 
\begin{itemize}
  \item completely filled; containing all that can be held; filled to utmost capacity
  \item complete; entire; maximum
  \item of the maximum size, amount, extent, volume, etc.
\end{itemize}
\emph{Everlasting} (as adjective)
\begin{itemize}
  \item lasting forever; eternal
  \item lasting or continuing for an indefinitely long time
  \item incessant; constantly recurring
  \item wearisome; tedious
\end{itemize}
\emph{Everlasting} (as noun)
\begin{itemize}
  \item eternal duration; eternity
  \item the Everlasting, God
  \item any of various plants that retain their shape or color when dried, as certain composite plants of the genera \emph{Helichrysum}, \emph{Gnaphalium}, and \emph{Helipterum}
\end{itemize}

\section{Principles\label{intro:principles1}}
\begin{itemize}
\item \index{Gospel} Fulness of the gospel
\end{itemize}

\section{Comments\label{intro:comments1}}
Just as we say in the eighth Article of Faith, ``We believe the Bible to be the word of God as far as it is translated correctly; we also believe the Book of Mormon to be the word of God."  The Bible does contain the fulness of the gospel - faith, repentance, baptism by immersion for the remission of sins, laying on of hands for the gift of the Holy Ghost, and enduring to the end.  We use the Book of Mormon as another witness of Jesus Christ and His gospel.  In my personal opinion, the Book of Mormon is clearer in its language in describing the gospel and what we must do in order to return to God again.  When we speak of the everlasting gospel, if we look at the definitions above, we are in fact saying that we are looking at God's gospel.  The use of the word \emph{comparable} is informative as well.  We do not say that the Bible and the Book of Mormon are \emph{the same}.  Rather, they share the same information, but do so with two completely different groups of people.  Note that the first definition of comparable mentions having features in common -- the implication is that not everything is the same.

\section{Additional References\label{intro:references1}}
\begin{itemize}
\item Articles of Faith 1:4, 8
\end{itemize}
%%%%%%%%%%%%%%%%%%%%%%%%%%%%%%%%%%%%%%%%%%%%%%%%%%%%%%%%%%%%%%%%%%%%%%%%%%%%%%%%%%%%%%%%%%%
%%%%%%%%%%%%%%%%%%%%%%%%%%%%%%%%%%%%%%%%%%%%%%%%%%%%%%%%%%%%%%%%%%%%%%%%%%%%%%%%%%%%%%%%%%%

\printindex
\end{document}
