\documentclass[12pt]{report}

\usepackage{cleveref}
\usepackage[margin=1in]{geometry}
\usepackage{makeidx}
\usepackage{chngcntr}
\usepackage{verbatim}
\usepackage{multirow}

\counterwithin*{chapter}{part}
\author{Jarin French}
\title{Book of Mormon Study}
\makeindex

%------------------------------------------------------------------------------------------
\iffalse
%%%%%%%%%%%%%%%%%%%%%%%%%%%%%%%%%%%%%%%%%%%%%%%%%%%%%%%%%%%%%%%%%%%%%%%%%%%%%%%%%%%%%%%%%%%
%%%%%%%%%%%%%%%%%%%%%%%%%%%%%%%%%%%%%%%%%%%%%%%%%%%%%%%%%%%%%%%%%%%%%%%%%%%%%%%%%%%%%%%%%%%
\section{1st Paragraph\label{example_chapter}}
\begin{center}
\begin{quote}
``"
\end{quote}
\end{center}

\begin{table}[h!]
\centering
\label{table:example_table}
\begin{tabular*}{\textwidth}{c @{\extracolsep{\fill}}cc}
Speaker & Important Characters & Target Audience \\
\hline
\rule{0pt}{3ex} --- & --- & --- 
\end{tabular*}
\end{table}

\subsection{Definitions\label{example_definition}}
\emph{Word}: \begin{itemize}
\item \emph{Definitions 1-n}
\end{itemize}
\subsection{Principles\label{example_principle}}
\begin{itemize}
\item \index{test}\emph{None} (yet)
\end{itemize}

\subsection{Comments\label{example_comment}}

\subsection{Additional References\label{example_references}}
\begin{itemize}
\item \emph{None} (yet)
\end{itemize}
%%%%%%%%%%%%%%%%%%%%%%%%%%%%%%%%%%%%%%%%%%%%%%%%%%%%%%%%%%%%%%%%%%%%%%%%%%%%%%%%%%%%%%%%%%%
%%%%%%%%%%%%%%%%%%%%%%%%%%%%%%%%%%%%%%%%%%%%%%%%%%%%%%%%%%%%%%%%%%%%%%%%%%%%%%%%%%%%%%%%%%%
\fi
%------------------------------------------------------------------------------------------

\begin{document}
\maketitle
\tableofcontents

\part{Introduction and Title Page\label{BoM:intro}}
\chapter{Title Page}
%%%%%%%%%%%%%%%%%%%%%%%%%%%%%%%%%%%%%%%%%%%%%%%%%%%%%%%%%%%%%%%%%%%%%%%%%%%%%%%%%%%%%%%%%%%
%%%%%%%%%%%%%%%%%%%%%%%%%%%%%%%%%%%%%%%%%%%%%%%%%%%%%%%%%%%%%%%%%%%%%%%%%%%%%%%%%%%%%%%%%%%
\section{1st Paragraph\label{titlePage:1st}}
\begin{center}
\begin{quote}
``The Book of Mormon: An account written by the hand of Mormon upon plates taken from the plates of Nephi."
\end{quote}
\end{center}

\begin{table}[h!]
\centering
\label{table:titlePage1}
\begin{tabular*}{\textwidth}{c @{\extracolsep{\fill}}cc}
Speaker & Important Characters & Target Audience \\
\hline
\rule{0pt}{3ex}Mormon & --- & The reader 
\end{tabular*}
\end{table}

\subsection{Definitions\label{titlePage:DFN1}}
\subsection{Principles\label{titlePage:principles1}}
\begin{itemize}
\item \index{Record Keeping}Record keeping
\end{itemize}

\subsection{Comments\label{titlePage:comments1}}
Mormon summarized the entirety of Lehi's descendents' history, and used the plates that Nephi started (approximately 1000 years previously).

\subsection{Additional References\label{titlePage:references1}}
\begin{itemize}
\item \emph{None} (yet)
\end{itemize}
%%%%%%%%%%%%%%%%%%%%%%%%%%%%%%%%%%%%%%%%%%%%%%%%%%%%%%%%%%%%%%%%%%%%%%%%%%%%%%%%%%%%%%%%%%%
%%%%%%%%%%%%%%%%%%%%%%%%%%%%%%%%%%%%%%%%%%%%%%%%%%%%%%%%%%%%%%%%%%%%%%%%%%%%%%%%%%%%%%%%%%%

%%%%%%%%%%%%%%%%%%%%%%%%%%%%%%%%%%%%%%%%%%%%%%%%%%%%%%%%%%%%%%%%%%%%%%%%%%%%%%%%%%%%%%%%%%%
%%%%%%%%%%%%%%%%%%%%%%%%%%%%%%%%%%%%%%%%%%%%%%%%%%%%%%%%%%%%%%%%%%%%%%%%%%%%%%%%%%%%%%%%%%%
\section{2nd Paragraph\label{titlePage:2nd}}
\begin{center}
\begin{quote}
``Wherefore, it is an abridgment of the record of the people of Nephi, and also of the Lamanites -- Written to the Lamanites, who are a remnant of the house of Israel; and also to Jew and Gentile -- Written by way of commandment, and also by the spirit of prophecy and of revelation -- Written and sealed up, and hid up unto the Lord, that they might not be destroyed -- To come forth by the gift and power of God unto the interpretation thereof -- Sealed by the hand of Moroni, and hid up unto the Lord, to come forth in due time by way of the Gentile -- The interpretation thereof by the gift of God."
\end{quote}
\end{center}

\begin{table}[h!]
\centering
\label{table:titlePage2}
\begin{tabular*}{\textwidth}{c@{\extracolsep{\fill}}cc}
Speaker & Important Characters & Target Audience \\
\hline
\rule{0pt}{3ex}Mormon & People of Nephi; Lamanites; Jew; Gentile & Lamanites; Jews; Gentiles 
\end{tabular*}
\end{table}

\subsection{Definitions\label{titlePage:DFN2}}
\emph{Abridge}: \begin{itemize}
\item to shorten by omissions while retaining the basic contents
\item to reduce or lessen in duration, scope, authority, etc.; diminish; curtail
\item to deprive; cut off
\end{itemize}
\subsection{Principles\label{titlePage:principles2}}
\begin{itemize}
\item \emph{None} (yet)
\end{itemize}

\subsection{Comments\label{titlePage:comments2}}
This book was written because of the spirit of prophecy, which as we find in Revelation 19:10 is the testimony of Christ, but it was also written because it was a commandment.  This leads to the thought that as we develop a testimony of Christ, we will be commanded to write our own testimony so that others can learn from our own experience.

\subsection{Additional References\label{titlePage:references2}}
\begin{itemize}
\item Revelation 19:10
\end{itemize}
%%%%%%%%%%%%%%%%%%%%%%%%%%%%%%%%%%%%%%%%%%%%%%%%%%%%%%%%%%%%%%%%%%%%%%%%%%%%%%%%%%%%%%%%%%%
%%%%%%%%%%%%%%%%%%%%%%%%%%%%%%%%%%%%%%%%%%%%%%%%%%%%%%%%%%%%%%%%%%%%%%%%%%%%%%%%%%%%%%%%%%%

%%%%%%%%%%%%%%%%%%%%%%%%%%%%%%%%%%%%%%%%%%%%%%%%%%%%%%%%%%%%%%%%%%%%%%%%%%%%%%%%%%%%%%%%%%%
%%%%%%%%%%%%%%%%%%%%%%%%%%%%%%%%%%%%%%%%%%%%%%%%%%%%%%%%%%%%%%%%%%%%%%%%%%%%%%%%%%%%%%%%%%%
\section{3rd Paragraph\label{titlePage:3rd}}
\begin{center}
\begin{quote}
``An abridgment taken from the Book of Ether also, which is a record of the people of Jared, who were scattered at the time the Lord confounded the language of the people, when they were building a tower to get to heaven - Which is to shown unto the remnant of the House of Israel what great things the Lord hath done for their fathers; and that they may know the covenants of the Lord, that they are not cast off forever - And also to the convincing of the Jew and Gentile that JESUS is the CHRIST, the ETERNAL GOD, manifesting himself unto all nations - and now, if there are faults they are the mistakes of men; wherefore, condemn not the things of God, that ye may be found spotless at the judgment-seat of Christ."
\end{quote}
\end{center}

\begin{table}[h!]
\centering
\label{table:titlePage3}
\begin{tabular*}{\textwidth}{c @{\extracolsep{\fill}}cc}
Speaker & Important Characters & Target Audience \\
\hline
\rule{0pt}{3ex}Mormon & People of Jared; House of Israel; Jew; Gentile & Lamanites; Jews; Gentiles 
\end{tabular*}
\end{table}

\subsection{Definitions\label{titlePage:DFN3}}
\emph{Confound}: \begin{itemize}
\item to perplex or amaze, especially by sudden disturbance or surprise; bewilder; confuse
\item to throw into confusion or disorder
\item to thrown into increased confusion or disorder
\item to treat or regard erroneously as identical; mix or associate by mistake
\item to mingle so that the elements cannot be distinguished or separated
\item to damn (used in mild imprecations)
\item to contradict or refute
\end{itemize}
\subsection{Principles\label{titlePage:principles3}}
\begin{itemize}
\item \index{Christ}Jesus is the Christ
\end{itemize}

\subsection{Comments\label{titlePage_comments3}}
Including the abridgement of the Book of Ether is done to show that God is powerful, and watches out for those that serve Him.  Furthermore, this record gives another witness of the importance of the covenants of the Lord - without them, we are cast off forever.  The final statement in this paragraph is a warning to those that would find fault with the book.  Mormon readily acknowledges that he may have made a mistake, and implores those that read and ponder this book that they learn from his mistakes, rather than condemn the book because of his weakness.

Joseph Smith stated that ``The standard of truth has been erected; no unhallowed hand can stop the work from progressing; persecutions may rage, mobs may combine, armies may assemble, calumny may defame, but the truth of God will go forth boldly, nobly, and independent, till it has penetrated every continent, visited every clime, swept every country, and sounded in every ear, till the purposes of God shall be accomplished, and the Great Jehovah shall say the work is done."  We have a work to do, and this book is the way to do it!

\subsection{Additional References\label{titlePage:references3}}
\begin{itemize}
\item Mormon 8:17
\end{itemize}
%%%%%%%%%%%%%%%%%%%%%%%%%%%%%%%%%%%%%%%%%%%%%%%%%%%%%%%%%%%%%%%%%%%%%%%%%%%%%%%%%%%%%%%%%%%
%%%%%%%%%%%%%%%%%%%%%%%%%%%%%%%%%%%%%%%%%%%%%%%%%%%%%%%%%%%%%%%%%%%%%%%%%%%%%%%%%%%%%%%%%%%

%%%%%%%%%%%%%%%%%%%%%%%%%%%%%%%%%%%%%%%%%%%%%%%%%%%%%%%%%%%%%%%%%%%%%%%%%%%%%%%%%%%%%%%%%%%
%%%%%%%%%%%%%%%%%%%%%%%%%%%%%%%%%%%%%%%%%%%%%%%%%%%%%%%%%%%%%%%%%%%%%%%%%%%%%%%%%%%%%%%%%%%
\section{4th Paragraph\label{titlePage:4th}}
\begin{center}
\begin{quote}
``Translated by Joseph Smth, Jun."
\end{quote}
\end{center}

\begin{table}[h!]
\centering
\label{table:titlePage4}
\begin{tabular*}{\textwidth}{c @{\extracolsep{\fill}}cc}
Speaker & Important Characters & Target Audience \\
\hline
\rule{0pt}{3ex} Joseph Smith Junior & Joseph Smith Junior & The reader 
\end{tabular*}
\end{table}

\subsection{Definitions\label{titlePage:DFN4}}
\emph{Translate}: \begin{itemize}
\item to turn from one language into another or from a foreign language into one's own
\item to change the form, condition, nature, etc., of; transform; convert
\item to explain in terms that can be more easily understood; interpret
\item to bear, carry, or move from one place, position, etc., to another; transfer
\item \emph{Mechanics} to cause (a body) to move without rotation or angular displacement
\item \emph{Computers} to convert (a program, data, code, etc.) from one form to another
\item \emph{Telegraphy} to retransmit or forward (a message), as by a relay.
\end{itemize}
\subsection{Principles\label{titlePage:principles4}}
\begin{itemize}
\item \emph{None} (yet)
\end{itemize}

\subsection{Comments\label{titlePage:comments4}}
Joseph Smith Junior was called by God to translate the Book of Mormon - not write it.  

\subsection{Additional References\label{titlePage:references4}}
\begin{itemize}
\item Joseph Smith - History 1:67-68
\end{itemize}
%%%%%%%%%%%%%%%%%%%%%%%%%%%%%%%%%%%%%%%%%%%%%%%%%%%%%%%%%%%%%%%%%%%%%%%%%%%%%%%%%%%%%%%%%%%
%%%%%%%%%%%%%%%%%%%%%%%%%%%%%%%%%%%%%%%%%%%%%%%%%%%%%%%%%%%%%%%%%%%%%%%%%%%%%%%%%%%%%%%%%%%

\chapter{Introduction}
%%%%%%%%%%%%%%%%%%%%%%%%%%%%%%%%%%%%%%%%%%%%%%%%%%%%%%%%%%%%%%%%%%%%%%%%%%%%%%%%%%%%%%%%%%%
%%%%%%%%%%%%%%%%%%%%%%%%%%%%%%%%%%%%%%%%%%%%%%%%%%%%%%%%%%%%%%%%%%%%%%%%%%%%%%%%%%%%%%%%%%%
\section{1st Paragraph\label{intro:1st}}
\begin{center}
\begin{quote}
``The Book of Mormon is a volume of holy scripture comparable to the Bible. It is a record of God's dealings with the ancient inhabitants of the Americas and contains the fulness of the everlasting gospel."
\end{quote}
\end{center}

\begin{table}[h!]
\centering
\label{table:intro1}
\begin{tabular*}{\textwidth}{c @{\extracolsep{\fill}}cc}
Speaker & Important Characters & Target Audience \\
\hline
\rule{0pt}{3ex} --- & Ancient inhabitants of the Americas & The reader 
\end{tabular*}
\end{table}

\subsection{Definitions\label{intro:DFN1}}
\emph{Comparable}: 
\begin{itemize}
  \item capable of being compared; having features in common with something else to permit or suggest comparison
  \item worthy of comparison
  \item usable for comparison; similar
\end{itemize}
\emph{Fulness} (or \emph{Fullness}): 
\begin{itemize}
  \item completely filled; containing all that can be held; filled to utmost capacity
  \item complete; entire; maximum
  \item of the maximum size, amount, extent, volume, etc.
\end{itemize}
\emph{Everlasting} (as adjective):
\begin{itemize}
  \item lasting forever; eternal
  \item lasting or continuing for an indefinitely long time
  \item incessant; constantly recurring
  \item wearisome; tedious
\end{itemize}
\emph{Everlasting} (as noun):
\begin{itemize}
  \item eternal duration; eternity
  \item the Everlasting, God
  \item any of various plants that retain their shape or color when dried, as certain composite plants of the genera \emph{Helichrysum}, \emph{Gnaphalium}, and \emph{Helipterum}
\end{itemize}

\subsection{Principles\label{intro:principles1}}
\begin{itemize}
\item \index{Gospel} Fulness of the gospel
\end{itemize}

\subsection{Comments\label{intro:comments1}}
Just as we say in the eighth Article of Faith, ``We believe the Bible to be the word of God as far as it is translated correctly; we also believe the Book of Mormon to be the word of God."  The Bible does contain the fulness of the gospel - faith, repentance, baptism by immersion for the remission of sins, laying on of hands for the gift of the Holy Ghost, and enduring to the end.  We use the Book of Mormon as another witness of Jesus Christ and His gospel.  In my personal opinion, the Book of Mormon is clearer in its language in describing the gospel and what we must do in order to return to God again.  When we speak of the everlasting gospel, if we look at the definitions above, we are in fact saying that we are looking at God's gospel.  The use of the word \emph{comparable} is informative as well.  We do not say that the Bible and the Book of Mormon are \emph{the same}.  Rather, they share the same information, but do so with two completely different groups of people.  Note that the first definition of comparable mentions having features in common -- the implication is that not everything is the same.

Another comment on this paragraph is that in recent years the phrase ``as does the Bible" was removed from the introduction.  I do not think that this removal was done to imply that the Bible does not contain the fulness of the gospel, but rather that the introduction is just a starting point for those who are beginning to read the Book of Mormon.  Thus, the introduction focuses on the words contained in the Book of Mormon, without detracting from the importance of the Bible.

\subsection{Additional References\label{intro:references1}}
\begin{itemize}
\item Articles of Faith 1:4, 8
\end{itemize}
%%%%%%%%%%%%%%%%%%%%%%%%%%%%%%%%%%%%%%%%%%%%%%%%%%%%%%%%%%%%%%%%%%%%%%%%%%%%%%%%%%%%%%%%%%%
%%%%%%%%%%%%%%%%%%%%%%%%%%%%%%%%%%%%%%%%%%%%%%%%%%%%%%%%%%%%%%%%%%%%%%%%%%%%%%%%%%%%%%%%%%%

%%%%%%%%%%%%%%%%%%%%%%%%%%%%%%%%%%%%%%%%%%%%%%%%%%%%%%%%%%%%%%%%%%%%%%%%%%%%%%%%%%%%%%%%%%%
%%%%%%%%%%%%%%%%%%%%%%%%%%%%%%%%%%%%%%%%%%%%%%%%%%%%%%%%%%%%%%%%%%%%%%%%%%%%%%%%%%%%%%%%%%%
\section{2nd Paragraph\label{intro:2nd}}
\begin{center}
\begin{quote}
``The book was written by many ancient prophets by the spirit of prophecy and revelation.  Their words, written on gold plates, were quoted and abridged by a prophet-historian named Mormon.  The record gives an account of two civilizations.  One came from Jerusalem in 600 \scriptsize B.C. \normalsize and afterward separated into two nations, known as the Nephites and the Lamanites.  The other came much earlier when the Lord confounded the tongues at the Tower of Babel.  This group is known as the Jaredites.  After thousands of years, all were destroyed except the Lamanites, and they are among the ancestors of the American Indians."
\end{quote}
\end{center}

\begin{table}[h!]
\centering
\label{table:intro2}
\begin{tabular*}{\textwidth}{c @{\extracolsep{\fill}}cc}
Speaker & Important Characters & Target Audience \\
\hline
\rule{0pt}{3ex} --- & Mormon; Nephites; Lamanites; Jaredites & The reader 
\end{tabular*}
\end{table}

\subsection{Definitions\label{intro:DFN2}}
\emph{Historian}:
\begin{itemize}
\item an expert in history; authority on history
\item a writer of history; chronicler
\item from Middle French \emph{historien}, from Latin \emph{historia}. As ``writer of history in the higher sense" (distinguished from a mere annalist or chronicler), from 1530s
\end{itemize} 

\subsection{Principles\label{intro:principles2}}
\begin{itemize}
\item \emph{None} (yet)
\end{itemize}

\subsection{Comments\label{intro:comments2}}
This paragraph once again reaffirms that the Book of Mormon was written by prophets by revelation.  This book did not come out of a whim like many  of the novels and stories we read today.  This book was written as a commandment of God, and as such requires (and should \emph{command}) our deepest attentions and studies.  Note that this particular paragraph was updated in recent years.  The original paragraph read, at the end ``... they are among the principal ancestors of the American Indians."  Why this was changed, I do not know, but one thought that I have on that matter is that the Nephites, Lamanites, and Jaredites may not have been the only ones to settle on the American continent. While we may not have the records of other civilizations, that does not preclude them from having living and intermingling with the groups.  After all, there are the lost ten tribes of Israel, and there is a possibility that some of those lost tribes found their way to the Americas and are also among the ancestors of the American Indians.  Something that should stick out as a big warning to everyone is the phrase, ``all were destroyed except the Lamanites."  Having read the Book of Mormon many times, and knowing how both civilizations were destroyed, this scares me especially because of the parallels I see in the United States today.  While I am not sure that the United States alone was the home of these civilizations, it seems quite possible to me that the entirety of both North and South America were well traveled by these groups, and that the commandment of God that those who dwell in the land shall serve Him applies across the board.  Thus, it seems to me, that if we do not repent, there is a strong possibility that destruction will come our way from other groups.  I have to be careful in saying this though, because I do not mean to imply that the other countries in North and South America are wicked (as could be implied from the United States being destroyed by them similar to the Nephites).  That is certainly not my place to judge, especially as I have absolutely no idea the spiritual state of those peoples.  Rather, I mean to say that the Lord in His justice will not wait much longer for His children to repent before wars begin to break out -- not in this land alone, but across the world.

\subsection{Additional References\label{intro:references2}}
\begin{itemize}
\item Anthony W. Ivins, in Conference Report, Apr. 1929, 15
\end{itemize}
%%%%%%%%%%%%%%%%%%%%%%%%%%%%%%%%%%%%%%%%%%%%%%%%%%%%%%%%%%%%%%%%%%%%%%%%%%%%%%%%%%%%%%%%%%%
%%%%%%%%%%%%%%%%%%%%%%%%%%%%%%%%%%%%%%%%%%%%%%%%%%%%%%%%%%%%%%%%%%%%%%%%%%%%%%%%%%%%%%%%%%%

%%%%%%%%%%%%%%%%%%%%%%%%%%%%%%%%%%%%%%%%%%%%%%%%%%%%%%%%%%%%%%%%%%%%%%%%%%%%%%%%%%%%%%%%%%%
%%%%%%%%%%%%%%%%%%%%%%%%%%%%%%%%%%%%%%%%%%%%%%%%%%%%%%%%%%%%%%%%%%%%%%%%%%%%%%%%%%%%%%%%%%%
\section{3rd Paragraph\label{intro:3rd}}
\begin{center}
\begin{quote}
``The crowning event recorded in the Book of Mormon is the personal ministry of the Lord Jesus Christ among the Nephites soon after his resurrection.  It puts forth the doctrines of the gospel, outlines the plan of salvation, and tells men what they must do to gain peace in this life and eternal salvation in the life to come."
\end{quote}
\end{center}

\begin{table}[h!]
\centering
\label{table:intro3}
\begin{tabular*}{\textwidth}{c @{\extracolsep{\fill}}cc}
Speaker & Important Characters & Target Audience \\
\hline
\rule{0pt}{3ex} --- & Jesus Christ; Nephites & The reader 
\end{tabular*}
\end{table}

\subsection{Definitions\label{intro:DFN3}}
\emph{Crowning}: \begin{itemize}
\item representing a level of surpassing achievement, attainment, etc.
\item forming or providing a crown, top, or summit
\item late 12c., from Old French \emph{coroner}, from \emph{corone} Related: \emph{Crowned}; \emph{crowning}. The latter in its sense of "that makes complete" is from 1650s.
\item (as \emph{crown}) the top or highest part of anything, as of a hat or a mountain
\item (as \emph{crown}) the distinction that comes from a great achievement.
\end{itemize}
\emph{Personal}: \begin{itemize}
\item of, relating to, or coming as from a particular person; individual; private
\item relating to, directed to, or intended for a particular person
\item intended for use by one person
\item referring or directed to a particular person in a disparaging or offensive sense or manner, usually involving character, behavior, appearance, etc.
\item making personal remarks or attacks
\item done, carried out, held, etc., in person
\item pertaining to or characteristic of a person or self-conscious being
\end{itemize}

\subsection{Principles\label{intro:principles3}}
\begin{itemize}
\item \index{Christ}Christ's personal ministry
\end{itemize}

\subsection{Comments\label{intro:comments3}}
Something that has always meant a lot to me is just how individually focused Christ is.  Throughout the Bible and the Book of Mormon, Christ shows us what it means to truly be selfless.  He shows such an infinite capacity for love that has always soothed my soul and driven me to work to become better, and more worthy of His love.  

This paragraph emphasizes that Christ is the focal point of this book.  His doctrine, His gospel, and the plan that He carried out are explained in greater detail in this book than in any other book.  It only makes sense that of all the books that can be studied, this one alone will bring a man closer to God than any other book, because it truly is ``the most correct of any book."

\subsection{Additional References\label{intro:references3}}
\begin{itemize}
\item \emph{None} (yet)
\end{itemize}
%%%%%%%%%%%%%%%%%%%%%%%%%%%%%%%%%%%%%%%%%%%%%%%%%%%%%%%%%%%%%%%%%%%%%%%%%%%%%%%%%%%%%%%%%%%
%%%%%%%%%%%%%%%%%%%%%%%%%%%%%%%%%%%%%%%%%%%%%%%%%%%%%%%%%%%%%%%%%%%%%%%%%%%%%%%%%%%%%%%%%%%

%%%%%%%%%%%%%%%%%%%%%%%%%%%%%%%%%%%%%%%%%%%%%%%%%%%%%%%%%%%%%%%%%%%%%%%%%%%%%%%%%%%%%%%%%%%
%%%%%%%%%%%%%%%%%%%%%%%%%%%%%%%%%%%%%%%%%%%%%%%%%%%%%%%%%%%%%%%%%%%%%%%%%%%%%%%%%%%%%%%%%%%
\section{4th Paragraph\label{intro:4th}}
\begin{center}
\begin{quote}
``After Mormon completed his writings, he delivered the account to his son Moroni, who added a few words of his own and hid up the plates in the hill Cumorah.  On September 21, 1823, the same Moroni, then a glorified, resurrected being, appeared to the Prophet Joseph Smith and instructed him relative to the ancient record and its destined translation into the English language."
\end{quote}
\end{center}

\begin{table}[h!]
\centering
\label{table:intro4}
\begin{tabular*}{\textwidth}{c @{\extracolsep{\fill}}cc}
Speaker & Important Characters & Target Audience \\
\hline
\rule{0pt}{3ex} --- & Mormon; Moroni; Joseph Smith & The reader 
\end{tabular*}
\end{table}

\subsection{Definitions\label{intro:DFN4}}
\emph{Destined}: \begin{itemize}
\item bound for a certain destination
\item ordained, appointed, or predetermined to be or do something
\item liable, planning, or intending to be or do something
\end{itemize}
\subsection{Principles\label{intro:principles4}}
\begin{itemize}
\item \index{Resurrection}Resurrection
\end{itemize}

\subsection{Comments\label{intro:comments4}}
This paragraph covers a lot of ground in such a short space -- from Moroni receiving the plates (approximately 400 \scriptsize A.D. \normalsize) to 1823, a span of nearly a millennium and a half.  The use of the word \emph{destined} is interesting to me.  While this introduction was written after the translation of the Book of Mormon into English, it is quite clear in the Book of Mormon itself that the authors knew that it would be translated into another language.

Something that has given me pause is the idea that Moroni was a glorified, \emph{resurrected} being.  Keep in mind that at the time of Moroni's death, Christ had already started the first resurrection nearly 400 years previous.  Alma indicates that there could be one resurrection and a final resurrection, a continuous resurrection, or a series of resurrections (see Alma 40).  If there were only two resurrections, that would indicate that both resurrections have already happened (the first being with Christ, the second being some time after the death of Moroni).  This does not sit well with me, so we look at the other two possibilities.  For the continuous resurrection idea, this seems slightly more plausible.  We know that when Christ was resurrected, many of the dead rose as well (see Matthew 27:52).  If that first resurrection began a continuous cycle of resurrection, this makes much more sense.  There could be an indeterminate amount of time between death and resurrection, as President Joseph F. Smith seems to indicate that Joseph Smith (who had died some 64 years previous) was still in the spirit world, and with others of the last dispensation, ``looked upon the long absence of their spirits from their bodies as a bondage." (see D\&C 138(:50)).  This bring us to the idea of a series of resurrections.  This idea stems from the fact that people simply do not experience death at the same time.  In a way, this could be viewed as a parody on the continuous resurrection idea, just with more spacing between resurrection `events.'  One thing that needs to be considered is that we have absolutely \emph{no} idea what the ordinance of resurrection entails (see \Cref{intro:references4} for further study).  Because of this, we can only speculate, but further study by faith and learning may provide further answers.

\subsection{Additional References\label{intro:references4}}
\begin{itemize}
\item See Hoskisson, P. Y., \emph{What's in a Name? The Name \emph{Cumorah}}, \emph{Journal of Book of Mormon Studies}, 13/1-2(2004):158-60, 174-75.
\item Alma 40
\item Matthew 27:52
\item D\&C 138 (esp. vs 50)
\item Matthew, R.J. \emph{Resurrection}, April 1991 \emph{Ensign}.
\item http://www.templestudy.com/2008/08/13/many-more-ordinances-including-resurrection/
\end{itemize}
%%%%%%%%%%%%%%%%%%%%%%%%%%%%%%%%%%%%%%%%%%%%%%%%%%%%%%%%%%%%%%%%%%%%%%%%%%%%%%%%%%%%%%%%%%%
%%%%%%%%%%%%%%%%%%%%%%%%%%%%%%%%%%%%%%%%%%%%%%%%%%%%%%%%%%%%%%%%%%%%%%%%%%%%%%%%%%%%%%%%%%%

%%%%%%%%%%%%%%%%%%%%%%%%%%%%%%%%%%%%%%%%%%%%%%%%%%%%%%%%%%%%%%%%%%%%%%%%%%%%%%%%%%%%%%%%%%%
%%%%%%%%%%%%%%%%%%%%%%%%%%%%%%%%%%%%%%%%%%%%%%%%%%%%%%%%%%%%%%%%%%%%%%%%%%%%%%%%%%%%%%%%%%%
\section{5th Paragraph\label{intro:5th}}
\begin{center}
\begin{quote}
``In due course the plates were delivered to Joseph Smith, who translated them by the gift and power of God.  The record is now published in many languages as a new and additional witness that Jesus Christ is the Son of the living God and that all who will come unto him and obey the laws and ordinances of his gospel may be saved."
\end{quote}
\end{center}

\begin{table}[h!]
\centering
\label{table:intro5}
\begin{tabular*}{\textwidth}{c @{\extracolsep{\fill}}cc}
Speaker & Important Characters & Target Audience \\
\hline
\rule{0pt}{3ex} --- & Joseph Smith; Jesus Christ & The reader 
\end{tabular*}
\end{table}

\subsection{Definitions\label{intro:DFN5}}
\emph{Additional}: \begin{itemize}
\item added; more; supplementary
\end{itemize}
\subsection{Principles\label{intro:principles5}}
\begin{itemize}
\item \index{Gospel} The Gospel of Jesus Christ
\end{itemize}

\subsection{Comments\label{intro:comments5}}
Again, an emphasis on the idea that Joseph Smith \emph{translated} the Book of Mormon through the gift and power of God.

At the time of this writing the Book of Mormon has been translated into over 110 different languages (see May 2015 \emph{Ensign} article).  As the gospel continues to spread, and as more and more people share their God-given gifts of language, the Book of Mormon will continue to spread throughout the world to bring everyone to a knowledge that Jesus Christ is the Savior of the world!

\subsection{Additional References\label{intro:references5}}
\begin{itemize}
\item See \Cref{titlePage:4th}
\item \emph{Book of Mormon in 110 Languages}, May 2015 \emph{Ensign}
\end{itemize}
%%%%%%%%%%%%%%%%%%%%%%%%%%%%%%%%%%%%%%%%%%%%%%%%%%%%%%%%%%%%%%%%%%%%%%%%%%%%%%%%%%%%%%%%%%%
%%%%%%%%%%%%%%%%%%%%%%%%%%%%%%%%%%%%%%%%%%%%%%%%%%%%%%%%%%%%%%%%%%%%%%%%%%%%%%%%%%%%%%%%%%%

%%%%%%%%%%%%%%%%%%%%%%%%%%%%%%%%%%%%%%%%%%%%%%%%%%%%%%%%%%%%%%%%%%%%%%%%%%%%%%%%%%%%%%%%%%%
%%%%%%%%%%%%%%%%%%%%%%%%%%%%%%%%%%%%%%%%%%%%%%%%%%%%%%%%%%%%%%%%%%%%%%%%%%%%%%%%%%%%%%%%%%%
\section{6th Paragraph\label{intro:6th}}
\begin{center}
\begin{quote}
``Concerning this record the Prophet Joseph Smith said: `I told the brethren that the Book of Mormon was the most correct of any book on earth, and the keystone of our religion, and a man would get nearer to God by abiding by its precepts, than by any other book.'"
\end{quote}
\end{center}

\begin{table}[h!]
\centering
\label{table:intro6}
\begin{tabular*}{\textwidth}{c @{\extracolsep{\fill}}cc}
Speaker & Important Characters & Target Audience \\
\hline
\rule{0pt}{3ex} --- & Joseph Smith & The reader 
\end{tabular*}
\end{table}

\subsection{Definitions\label{intro:DFN6}}
\emph{Keystone}: \begin{itemize}
\item the wedge-shaped piece at the summit of an arch, regarded as holding the other pieces in place.
\item something on which associated things depend
\end{itemize}
\emph{Abiding}:  \begin{itemize}
\item continuing without change; enduring; steadfast
\item to continue in a particular condition, attitude, relationship, etc.; last
\item to endure, sustain, or withstand without yielding or submitting
\item to wait for; await
\item to act in accord with
\item to submit to; agree to
\item to remain steadfast or faithful to; keep
\end{itemize}
\emph{Precept}: \begin{itemize}
\item a commandment or direction given as a rule of action or conduct
\item an injunction as to moral conduct; maxim
\item a procedural directive or rule, as for the performance of some technical operation
\end{itemize}
\subsection{Principles\label{intro:principles6}}
\begin{itemize}
\item \index{Keystone}The Book of Mormon is the keystone of our religion.
\end{itemize}

\subsection{Comments\label{intro:comments6}}
The idea of the Book of Mormon being the keystone of our religion is an important one.  Without the Book of Mormon, we have absolutely no claim that we are Christ's church restored to the earth.  Without the Book of Mormon, we have absolutely no claim that we have the power of God within our church in the priesthood.  We cannot claim to receive revelation, we cannot claim to have a modern prophet, and we cannot claim any of the knowledge that we have is true, if we do not have the Book of Mormon.  The Book of Mormon testifies in its purity of the simple truths of the gospel of Jesus Christ.  Because of its truth and divinity, we have what we have -- prophets, revelation, the priesthood, and a whole host of wonderful blessings that I really need to spend more time thinking about.  This book is \emph{precious}!

\subsection{Additional References\label{intro:references6}}
\begin{itemize}
\item Benson, E. T. \emph{The Book of Mormon -- Keystone of Our Religion}, October 1986 General Conference
\end{itemize}
%%%%%%%%%%%%%%%%%%%%%%%%%%%%%%%%%%%%%%%%%%%%%%%%%%%%%%%%%%%%%%%%%%%%%%%%%%%%%%%%%%%%%%%%%%%
%%%%%%%%%%%%%%%%%%%%%%%%%%%%%%%%%%%%%%%%%%%%%%%%%%%%%%%%%%%%%%%%%%%%%%%%%%%%%%%%%%%%%%%%%%%

%%%%%%%%%%%%%%%%%%%%%%%%%%%%%%%%%%%%%%%%%%%%%%%%%%%%%%%%%%%%%%%%%%%%%%%%%%%%%%%%%%%%%%%%%%%
%%%%%%%%%%%%%%%%%%%%%%%%%%%%%%%%%%%%%%%%%%%%%%%%%%%%%%%%%%%%%%%%%%%%%%%%%%%%%%%%%%%%%%%%%%%
\section{7th Paragraph\label{intro:7th}}
\begin{center}
\begin{quote}
``In addition to Joseph Smith, the Lord provided for eleven others to see the gold plates for themselves and to be special witnesses of the truth and divinity of the Book of Mormon.  Their written testimonies are included herewith as `The Testimony of Three Witnesses' and `The Testimony of Eight Witnesses.'"
\end{quote}
\end{center}

\begin{table}[h!]
\centering
\label{table:intro7}
\begin{tabular*}{\textwidth}{c @{\extracolsep{\fill}}cc}
Speaker & Important Characters & Target Audience \\
\hline
\rule{0pt}{3ex} --- & Joseph Smith; Three Witnesses; Eight Witnesses & The reader 
\end{tabular*}
\end{table}

\subsection{Definitions\label{intro:DFN7}}
\emph{Special}: \begin{itemize}
\item of a distinct or particular kind or character
\item being a particular one; particular, individual, or certain
\item pertaining or peculiar to a particular person, thing, instance, etc.; distinctive; unique
\item having a specific or particular function, purpose, etc.
\item distinguished or different from what is ordinary or usual
\item extraordinary; exceptional, as in amount or degree; especial
\item being such in an exceptional degree; particularly valued
\end{itemize}
\subsection{Principles\label{intro:principles7}}
\begin{itemize}
\item \index{Witness}Law of Witnesses
\end{itemize}

\subsection{Comments\label{intro:comments7}}
Throughout the history of mankind, God has operated under the Law of Witnesses, which is in essence that at least two testimonies must come forth to establish the truth of something (see Deuteronomy 19:15, 2 Corinthians 13:1 and 1 Timothy 5:19).  The Book of Mormon is a part of this law as it, along with the Bible, testifies of the divinity of Christ.  The law is that two or three witnesses must establish the truth, so the Book of Mormon itself needs the support of this law.  This support is found in the form of the testimonies of the prophet Joseph Smith (1), the testimonies of the Three Witnesses (2), and the testimonies of the Eight Witnesses (3).  By these testimonies is the truth and divinity of the Book of Mormon established.  Any further testimonies lend further credence to the original statement, but do not change the impact.  However, I would point out that our own testimonies can act as a witness of sorts to those that are seeking the truth.  If we seek to help those around us to come to know God in a deeper way, we can plant the seed of faith with our testimony.  As they hear the testimonies of the missionaries, that can be a second witness of the truth to them.  Furthermore, as they gain their own witness from their own study of the Book of Mormon, they experience their own example of the Law of Witnesses.

Something of interest to me is that part of the testimonies of these three sets of witnesses relates to the actual existence of the gold plates.  To me, this seems like a non-issue.  If God had commanded Joseph Smith to write down these words, seemingly pulling them out of thin air, I suppose that could make it more difficult to believe.  But the words themselves offer the challenge (as shown in \Cref{intro:8th}), which in essence tells us to not take the words of those who have testified before, but to become a witness ourselves.  

\subsection{Additional References\label{intro:references7}}
\begin{itemize}
\item Ether 5:2-4
\item 2 Nephi 11:3
\item 2 Nephi 27:12-13
\item Doctrine and Covenants 17
\item Deuteronomy 19:15
\item 2 Corinthians 13:1
\item 1 Timothy 5:19
\item Encyclopedia of Mormonism, \emph{Witnesses, Law of}
\item Steven C. Harper, "Evaluating the Book of Mormon Witnesses" in \emph{Religious Educator} 11, no. 2 (2010): 37-50.
\end{itemize}
%%%%%%%%%%%%%%%%%%%%%%%%%%%%%%%%%%%%%%%%%%%%%%%%%%%%%%%%%%%%%%%%%%%%%%%%%%%%%%%%%%%%%%%%%%%
%%%%%%%%%%%%%%%%%%%%%%%%%%%%%%%%%%%%%%%%%%%%%%%%%%%%%%%%%%%%%%%%%%%%%%%%%%%%%%%%%%%%%%%%%%%

%%%%%%%%%%%%%%%%%%%%%%%%%%%%%%%%%%%%%%%%%%%%%%%%%%%%%%%%%%%%%%%%%%%%%%%%%%%%%%%%%%%%%%%%%%%
%%%%%%%%%%%%%%%%%%%%%%%%%%%%%%%%%%%%%%%%%%%%%%%%%%%%%%%%%%%%%%%%%%%%%%%%%%%%%%%%%%%%%%%%%%%
\section{8th Paragraph\label{intro:8th}}
\begin{center}
\begin{quote}
``We invite all men everywhere to read the Book of Mormon, to ponder in their hearts the message it contains, and then to ask God, the Eternal Father, in the name of Christ if the book is true.  Those who pursue this course and ask in faith will gain a testimony of its truth and divinity by the power of the Holy Ghost. (See Moroni 10:3-5.)"
\end{quote}
\end{center}

\begin{table}[h!]
\centering
\label{table:intro8}
\begin{tabular*}{\textwidth}{c @{\extracolsep{\fill}}cc}
Speaker & Important Characters & Target Audience \\
\hline
\rule{0pt}{3ex} --- & All men; God; Christ; Holy Ghost & The reader 
\end{tabular*}
\end{table}

\subsection{Definitions\label{intro:DFN8}}
\emph{Ponder}: \begin{itemize}
\item to consider something deeply and thoroughly; meditate (often followed by over or upon)
\item to weigh carefully in the mind; consider thoughtfully
\item to estimate the worth of, to appraise
\end{itemize}
\emph{Pursue}: \begin{itemize}
\item to strive to gain; seek to attain or accomplish (an end, object, purpose, etc.)
\item to proceed in accordance with (a method, plan, etc.)
\item to carry on or continue (a course of action, a train of thought, an inquiry, studies, etc.)
\item to continue
\end{itemize}
\subsection{Principles\label{intro:principles8}}
\begin{itemize}
\item \index{Prayer}Prayer
\end{itemize}

\subsection{Comments\label{intro:comments8}}
This book has a powerful promise.  It promises that if we ask God, in the name of Christ, if the book is true, then we will know by the power of the Holy Ghost that it is true.  This book encourages us to test it, to actually hold it up to the light it professes to have, and determine if it is true, or if it is simply a good story.  The promise is that its truth will be revealed to you. The requirement is that you simply put in the effort to know.  You can't test something if you don't have the necessary knowledge - a test is a way to determine the difference between what you know and what is (taught to be) right.

\subsection{Additional References\label{intro:references8}}
\begin{itemize}
\item Moroni 10:3-5
\end{itemize}
%%%%%%%%%%%%%%%%%%%%%%%%%%%%%%%%%%%%%%%%%%%%%%%%%%%%%%%%%%%%%%%%%%%%%%%%%%%%%%%%%%%%%%%%%%%
%%%%%%%%%%%%%%%%%%%%%%%%%%%%%%%%%%%%%%%%%%%%%%%%%%%%%%%%%%%%%%%%%%%%%%%%%%%%%%%%%%%%%%%%%%%

%%%%%%%%%%%%%%%%%%%%%%%%%%%%%%%%%%%%%%%%%%%%%%%%%%%%%%%%%%%%%%%%%%%%%%%%%%%%%%%%%%%%%%%%%%%
%%%%%%%%%%%%%%%%%%%%%%%%%%%%%%%%%%%%%%%%%%%%%%%%%%%%%%%%%%%%%%%%%%%%%%%%%%%%%%%%%%%%%%%%%%%
\section{9th Paragraph\label{intro:9th}}
\begin{center}
\begin{quote}
``Those who gain this divine witness from the Holy Spirit will also come to know by the same power that Jesus Christ is the Savior of the world, that Joseph Smith is his revelator and prophet in these last days, and that The Church of Jesus Christ of Latter-day Saints is the Lord's kingdom once again established on the earth, preparatory to the second coming of the Messiah."
\end{quote}
\end{center}

\begin{table}[h!]
\centering
\label{table:intro9}
\begin{tabular*}{\textwidth}{c @{\extracolsep{\fill}}cc}
Speaker & Important Characters & Target Audience \\
\hline
\rule{0pt}{3ex} --- & Holy Spirit; Jesus Christ; Joseph Smith & The reader 
\end{tabular*}
\end{table}

\subsection{Definitions\label{intro:DFN9}}
\emph{Divine}: \begin{itemize}
\item of or relating to a god, especially the Supreme Being
\item addressed, appropriated, or devoted to God or a god; religious; sacred
\item proceeding from God or a god
\item godlike; characteristic of or befitting a deity
\item heavenly; celestial
\item extremely good; unusually lovely
\end{itemize}
\emph{Revelator}: \begin{itemize}
\item a person who makes a revelation
\end{itemize}
\emph{Preparatory}: \begin{itemize}
\item serving or designed to prepare
\item preliminary; introductory
\item of or relating to training that prepares for more advanced education
\end{itemize}
\subsection{Principles\label{intro:principles9}}
\begin{itemize}
\item \index{Witness}Witness from God
\end{itemize}

\subsection{Comments\label{intro:comment9}}
God teaches us in a way that helps us best to understand.  If we have learned something from Him in a certain way (i.e. a burning of the bosom, or words spoken into our minds), it is a safe assumption that further light and knowledge from Him will come similarly.  The scriptures state that God is not a changeable being (see Moroni 8:18), so He will not suddenly try a method on us that won't work.  That being said, God has known us much longer than we currently know ourselves.  We existed with Him before the foundation of the world (see Jeremiah 1:5), and since we cannot remember that time in our life, God knows us intimately, and much better than we know ourselves.  Because of this knowledge, God knows \emph{exactly} how to teach us.  There may be ways that we have to prepare ourselves for, but God doesn't `experiment' with His children.  After all, this existence is a test for us as His children to see whom we will follow: God, or Satan.

It is important to note that the same feelings and testimony that we gain regarding the truthfulness of the Book of Mormon will also apply to Christ being our Savior, Joseph Smith being Christ's prophet on the earth in the last days, and the Church of Jesus Christ of Latter-day Saints being the Lord's established kingdom on the earth.  In much the same way, the Holy Ghost will testify of the truthfulness of the Bible, and by extension that Christ is our Savior, that prophets were called in those days, and that Christ's kingdom had been established on the earth (and here I will add that His kingdom was taken from the earth due to wickedness).  Both the Bible and the Book of Mormon warn of the things to come in the last days.  We must prepare now for the coming of our Lord and Savior, Jesus Christ.  We must be found worthy and prepared at His coming.  Only God knows exactly when Christ will return, but until then we must vigilantly watch and pray, and prepare our lamps for the Bridegroom.

\subsection{Additional References\label{intro:references9}}
\begin{itemize}
\item Moroni 8:18
\item Jeremiah 1:5
\end{itemize}
%%%%%%%%%%%%%%%%%%%%%%%%%%%%%%%%%%%%%%%%%%%%%%%%%%%%%%%%%%%%%%%%%%%%%%%%%%%%%%%%%%%%%%%%%%%
%%%%%%%%%%%%%%%%%%%%%%%%%%%%%%%%%%%%%%%%%%%%%%%%%%%%%%%%%%%%%%%%%%%%%%%%%%%%%%%%%%%%%%%%%%%

\chapter{The Testimony of Three Witnesses}
%%%%%%%%%%%%%%%%%%%%%%%%%%%%%%%%%%%%%%%%%%%%%%%%%%%%%%%%%%%%%%%%%%%%%%%%%%%%%%%%%%%%%%%%%%%
%%%%%%%%%%%%%%%%%%%%%%%%%%%%%%%%%%%%%%%%%%%%%%%%%%%%%%%%%%%%%%%%%%%%%%%%%%%%%%%%%%%%%%%%%%%
\section{1st Paragraph\label{3witness}}
\begin{center}
\begin{quote}
``Be it known unto all nations, kindreds, tongues, and people, unto whom this work shall come: That we, through the grace of God the Father and our Lord Jesus Christ, have seen the plates which contain this record, which is a record of the people of Nephi, and also of the Lamanites, their brethren, and also of the people of Jared, who came from the tower which hath been spoken.  And we also know that they have been translated by the gift and power of God, for his voice hath declared it unto us; wherefore we know of a surety that the work is true.  And we also testify that we have seen the engravings which are upon the plates; and they have been shown unto us by the power of God, and not of man.  And we declare with words of soberness, that an angel of God came down from heaven, and he brought and laid before our eyes, that we beheld and saw the plates, and the engravings thereon; and we know that it is by the grace of God the Father, and our Lord Jesus Christ, that we beheld and bear record that these things are true.  And it is marvelous in our eyes.  Nevertheless, the voice of the Lord commanded us that we should bear record of it; wherefore, to be obedient unto the commandments of God, we bear testimony of these things. And we know that if we are faithful in Christ, we shall rid our garments of the blood of all men, and be found spotless before the judgment-seat of Christ, and shall dwell with him eternally in the heavens.  And the honor be to the Father, and to the Son, and to the Holy Ghost, which is one God.  Amen."
\end{quote}
\end{center}

\begin{table}[h!]
\centering
\label{table:example_table}
\begin{tabular*}{\textwidth}{c @{\extracolsep{\fill}}cc}
Speaker & Important Characters & Target Audience \\
\hline
\rule{0pt}{3ex} Oliver Cowdery & God & \multirow{3}{*}{The reader} \\
David Whitmer & Jesus Christ \\
Martin Harris & Holy Ghost
\end{tabular*}
\end{table}

\subsection{Definitions\label{3witness:DFN}}
\emph{Word}: \begin{itemize}
\item \emph{Definitions 1-n}
\end{itemize}
\subsection{Principles\label{3witness:principles}}
\begin{itemize}
\item \index{test}\emph{None} (yet)
\end{itemize}

\subsection{Comments\label{3witness:comments}}

\subsection{Additional References\label{3witness:references}}
\begin{itemize}
\item \emph{None} (yet)
\end{itemize}
%%%%%%%%%%%%%%%%%%%%%%%%%%%%%%%%%%%%%%%%%%%%%%%%%%%%%%%%%%%%%%%%%%%%%%%%%%%%%%%%%%%%%%%%%%%
%%%%%%%%%%%%%%%%%%%%%%%%%%%%%%%%%%%%%%%%%%%%%%%%%%%%%%%%%%%%%%%%%%%%%%%%%%%%%%%%%%%%%%%%%%%
\printindex
\end{document}
