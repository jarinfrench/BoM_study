\documentclass[12pt]{report}

\usepackage{makeidx}
\usepackage{chngcntr}
\usepackage{verbatim}
\usepackage{multirow}
\usepackage[margin=1in]{geometry}
\usepackage{hyperref}
\usepackage{cleveref}

\counterwithin*{chapter}{part}
\setcounter{secnumdepth}{5}
\author{Jarin French}
\title{Book of Mormon Study}
\makeindex

%------------------------------------------------------------------------------------------
\iffalse
%%%%%%%%%%%%%%%%%%%%%%%%%%%%%%%%%%%%%%%%%%%%%%%%%%%%%%%%%%%%%%%%%%%%%%%%%%%%%%%%%%%%%%%%%%%
%%%%%%%%%%%%%%%%%%%%%%%%%%%%%%%%%%%%%%%%%%%%%%%%%%%%%%%%%%%%%%%%%%%%%%%%%%%%%%%%%%%%%%%%%%%
\section{1st Paragraph\label{example_chapter}}
\begin{center}
\begin{quote}
``"
\end{quote}
\end{center}

\begin{table}[h!]
\centering
\label{table:example_table}
\begin{tabular*}{\textwidth}{l @{\extracolsep{\fill}}cc}
Speaker & Important Characters & Target Audience \\
\hline
\rule{0pt}{3ex}--- & --- & --- 
\end{tabular*}
\end{table}

\subsection{Definitions\label{example_definition}}
\emph{Word}: \begin{itemize}
\item \emph{Definitions 1-n}
\end{itemize}

\subsection{Principles and Tags\label{example_principle}}
\begin{itemize}
\item \index{}\emph{None} (yet)
\end{itemize}

\subsection{Comments\label{example_comment}}

\subsection{Additional References\label{example_references}}
\begin{itemize}
\item \emph{None} (yet)
\end{itemize}
%%%%%%%%%%%%%%%%%%%%%%%%%%%%%%%%%%%%%%%%%%%%%%%%%%%%%%%%%%%%%%%%%%%%%%%%%%%%%%%%%%%%%%%%%%%
%%%%%%%%%%%%%%%%%%%%%%%%%%%%%%%%%%%%%%%%%%%%%%%%%%%%%%%%%%%%%%%%%%%%%%%%%%%%%%%%%%%%%%%%%%%
\fi
%------------------------------------------------------------------------------------------

\begin{document}
\maketitle
\tableofcontents

\part{Introduction and Title Page\label{BoM:intro}}
\chapter{Title Page\label{chapter:titlePage}}
%%%%%%%%%%%%%%%%%%%%%%%%%%%%%%%%%%%%%%%%%%%%%%%%%%%%%%%%%%%%%%%%%%%%%%%%%%%%%%%%%%%%%%%%%%%
%%%%%%%%%%%%%%%%%%%%%%%%%%%%%%%%%%%%%%%%%%%%%%%%%%%%%%%%%%%%%%%%%%%%%%%%%%%%%%%%%%%%%%%%%%%
\section{1st Paragraph\label{titlePage:1st}}
\begin{center}
\begin{quote}
``The Book of Mormon: An account written by the hand of Mormon upon plates taken from the plates of Nephi."
\end{quote}
\end{center}

\begin{table}[h!]
\centering
\label{table:titlePage1}
\begin{tabular*}{\textwidth}{l @{\extracolsep{\fill}}cc}
Speaker & Important Characters & Target Audience \\
\hline
\rule{0pt}{3ex}Mormon & --- & The reader 
\end{tabular*}
\end{table}

\subsection{Definitions\label{titlePage:DFN1}}
\emph{NOTE: Unless otherwise specified, the definitions contained in these sections are from \href{http://www.dictionary.com}{dictionary.com}.}
\subsection{Principles and Tags\label{titlePage:principles1}}
\begin{itemize}
\item \index{Record Keeping}Record keeping
\end{itemize}

\subsection{Comments\label{titlePage:comments1}}
Mormon summarized the entirety of Lehi's descendents' history, and used the plates that Nephi started (approximately 1000 years previously).

\subsection{Additional References\label{titlePage:references1}}
\begin{itemize}
\item \emph{None} (yet)
\end{itemize}
%%%%%%%%%%%%%%%%%%%%%%%%%%%%%%%%%%%%%%%%%%%%%%%%%%%%%%%%%%%%%%%%%%%%%%%%%%%%%%%%%%%%%%%%%%%
%%%%%%%%%%%%%%%%%%%%%%%%%%%%%%%%%%%%%%%%%%%%%%%%%%%%%%%%%%%%%%%%%%%%%%%%%%%%%%%%%%%%%%%%%%%

%%%%%%%%%%%%%%%%%%%%%%%%%%%%%%%%%%%%%%%%%%%%%%%%%%%%%%%%%%%%%%%%%%%%%%%%%%%%%%%%%%%%%%%%%%%
%%%%%%%%%%%%%%%%%%%%%%%%%%%%%%%%%%%%%%%%%%%%%%%%%%%%%%%%%%%%%%%%%%%%%%%%%%%%%%%%%%%%%%%%%%%
\section{2nd Paragraph\label{titlePage:2nd}}
\begin{center}
\begin{quote}
``Wherefore, it is an abridgment of the record of the people of Nephi, and also of the Lamanites -- Written to the Lamanites, who are a remnant of the house of Israel; and also to Jew and Gentile -- Written by way of commandment, and also by the spirit of prophecy and of revelation -- Written and sealed up, and hid up unto the Lord, that they might not be destroyed -- To come forth by the gift and power of God unto the interpretation thereof -- Sealed by the hand of Moroni, and hid up unto the Lord, to come forth in due time by way of the Gentile -- The interpretation thereof by the gift of God."
\end{quote}
\end{center}

\begin{table}[h!]
\centering
\label{table:titlePage2}
\begin{tabular*}{\textwidth}{l @{\extracolsep{\fill}}cc}
Speaker & Important Characters & Target Audience \\
\hline
\rule{0pt}{3ex}Mormon & People of Nephi; Lamanites; Jew; Gentile & Lamanites; Jews; Gentiles 
\end{tabular*}
\end{table}

\subsection{Definitions\label{titlePage:DFN2}}
\emph{Abridge}: \begin{itemize}
\item to shorten by omissions while retaining the basic contents
\item to reduce or lessen in duration, scope, authority, etc.; diminish; curtail
\item to deprive; cut off
\end{itemize}
\subsection{Principles and Tags\label{titlePage:principles2}}
\begin{itemize}
\item \index{Sealing Power}Sealing Power
\end{itemize}

\subsection{Comments\label{titlePage:comments2}}
This book was written because of the spirit of prophecy, which as we find in Revelation 19:10 is the testimony of Christ, but it was also written because it was a commandment.  This leads to the thought that as we develop a testimony of Christ, we will be commanded to write our own testimony so that others can learn from our own experience.

\subsection{Additional References\label{titlePage:references2}}
\begin{itemize}
\item Revelation 19:10
\end{itemize}
%%%%%%%%%%%%%%%%%%%%%%%%%%%%%%%%%%%%%%%%%%%%%%%%%%%%%%%%%%%%%%%%%%%%%%%%%%%%%%%%%%%%%%%%%%%
%%%%%%%%%%%%%%%%%%%%%%%%%%%%%%%%%%%%%%%%%%%%%%%%%%%%%%%%%%%%%%%%%%%%%%%%%%%%%%%%%%%%%%%%%%%

%%%%%%%%%%%%%%%%%%%%%%%%%%%%%%%%%%%%%%%%%%%%%%%%%%%%%%%%%%%%%%%%%%%%%%%%%%%%%%%%%%%%%%%%%%%
%%%%%%%%%%%%%%%%%%%%%%%%%%%%%%%%%%%%%%%%%%%%%%%%%%%%%%%%%%%%%%%%%%%%%%%%%%%%%%%%%%%%%%%%%%%
\section{3rd Paragraph\label{titlePage:3rd}}
\begin{center}
\begin{quote}
``An abridgment taken from the Book of Ether also, which is a record of the people of Jared, who were scattered at the time the Lord confounded the language of the people, when they were building a tower to get to heaven - Which is to shown unto the remnant of the House of Israel what great things the Lord hath done for their fathers; and that they may know the covenants of the Lord, that they are not cast off forever - And also to the convincing of the Jew and Gentile that JESUS is the CHRIST, the ETERNAL GOD, manifesting himself unto all nations - and now, if there are faults they are the mistakes of men; wherefore, condemn not the things of God, that ye may be found spotless at the judgment-seat of Christ."
\end{quote}
\end{center}

\begin{table}[h!]
\centering
\label{table:titlePage3}
\begin{tabular*}{\textwidth}{l @{\extracolsep{\fill}}cc}
Speaker & Important Characters & Target Audience \\
\hline
\rule{0pt}{3ex}Mormon & People of Jared; House of Israel; Jew; Gentile & Lamanites; Jews; Gentiles 
\end{tabular*}
\end{table}

\subsection{Definitions\label{titlePage:DFN3}}
\emph{Confound}: \begin{itemize}
\item to perplex or amaze, especially by sudden disturbance or surprise; bewilder; confuse
\item to throw into confusion or disorder
\item to thrown into increased confusion or disorder
\item to treat or regard erroneously as identical; mix or associate by mistake
\item to mingle so that the elements cannot be distinguished or separated
\item to damn (used in mild imprecations)
\item to contradict or refute
\end{itemize}
\subsection{Principles and Tags\label{titlePage:principles3}}
\begin{itemize}
\item \index{Christ}Jesus is the Christ
\item \index{Remember}Remember
\item \index{If-Then}If-Then
\end{itemize}

\subsection{Comments\label{titlePage_comments3}}
Including the abridgment of the Book of Ether is done to show that God is powerful, and watches out for those that serve Him.  Furthermore, this record gives another witness of the importance of the covenants of the Lord - without them, we are cast off forever.  The final statement in this paragraph is a warning to those that would find fault with the book.  Mormon readily acknowledges that he may have made a mistake, and implores those that read and ponder this book that they learn from his mistakes, rather than condemn the book because of his weakness.

Joseph Smith stated that ``The standard of truth has been erected; no unhallowed hand can stop the work from progressing; persecutions may rage, mobs may combine, armies may assemble, calumny may defame, but the truth of God will go forth boldly, nobly, and independent, till it has penetrated every continent, visited every clime, swept every country, and sounded in every ear, till the purposes of God shall be accomplished, and the Great Jehovah shall say the work is done."  We have a work to do, and this book is the way to do it!

\subsection{Additional References\label{titlePage:references3}}
\begin{itemize}
\item Mormon 8:17
\end{itemize}
%%%%%%%%%%%%%%%%%%%%%%%%%%%%%%%%%%%%%%%%%%%%%%%%%%%%%%%%%%%%%%%%%%%%%%%%%%%%%%%%%%%%%%%%%%%
%%%%%%%%%%%%%%%%%%%%%%%%%%%%%%%%%%%%%%%%%%%%%%%%%%%%%%%%%%%%%%%%%%%%%%%%%%%%%%%%%%%%%%%%%%%

%%%%%%%%%%%%%%%%%%%%%%%%%%%%%%%%%%%%%%%%%%%%%%%%%%%%%%%%%%%%%%%%%%%%%%%%%%%%%%%%%%%%%%%%%%%
%%%%%%%%%%%%%%%%%%%%%%%%%%%%%%%%%%%%%%%%%%%%%%%%%%%%%%%%%%%%%%%%%%%%%%%%%%%%%%%%%%%%%%%%%%%
\section{4th Paragraph\label{titlePage:4th}}
\begin{center}
\begin{quote}
``Translated by Joseph Smith, Jun."
\end{quote}
\end{center}

\begin{table}[h!]
\centering
\label{table:titlePage4}
\begin{tabular*}{\textwidth}{l @{\extracolsep{\fill}}cc}
Speaker & Important Characters & Target Audience \\
\hline
\rule{0pt}{3ex}Joseph Smith Junior & Joseph Smith Junior & The reader 
\end{tabular*}
\end{table}

\subsection{Definitions\label{titlePage:DFN4}}
\emph{Translate}: \begin{itemize}
\item to turn from one language into another or from a foreign language into one's own
\item to change the form, condition, nature, etc., of; transform; convert
\item to explain in terms that can be more easily understood; interpret
\item to bear, carry, or move from one place, position, etc., to another; transfer
\item \emph{Mechanics}: to cause (a body) to move without rotation or angular displacement
\item \emph{Computers}: to convert (a program, data, code, etc.) from one form to another
\item \emph{Telegraphy}: to retransmit or forward (a message), as by a relay.
\end{itemize}
\subsection{Principles and Tags\label{titlePage:principles4}}
\begin{itemize}
\item \emph{None} (yet)
\end{itemize}

\subsection{Comments\label{titlePage:comments4}}
Joseph Smith Junior was called by God to translate the Book of Mormon - not write it.  

\subsection{Additional References\label{titlePage:references4}}
\begin{itemize}
\item Joseph Smith - History 1:67-68
\end{itemize}
%%%%%%%%%%%%%%%%%%%%%%%%%%%%%%%%%%%%%%%%%%%%%%%%%%%%%%%%%%%%%%%%%%%%%%%%%%%%%%%%%%%%%%%%%%%
%%%%%%%%%%%%%%%%%%%%%%%%%%%%%%%%%%%%%%%%%%%%%%%%%%%%%%%%%%%%%%%%%%%%%%%%%%%%%%%%%%%%%%%%%%%

\section{General Comments on \Cref{chapter:titlePage}}
An important note about the entirety of the title page described in this chapter is that it was taken directly from the Book of Mormon (with the exception of \Cref{titlePage:4th}).  I believe that these words were written directly by Mormon or Moroni.  

\chapter{Introduction\label{chapter:intro}}
%%%%%%%%%%%%%%%%%%%%%%%%%%%%%%%%%%%%%%%%%%%%%%%%%%%%%%%%%%%%%%%%%%%%%%%%%%%%%%%%%%%%%%%%%%%
%%%%%%%%%%%%%%%%%%%%%%%%%%%%%%%%%%%%%%%%%%%%%%%%%%%%%%%%%%%%%%%%%%%%%%%%%%%%%%%%%%%%%%%%%%%
\section{1st Paragraph\label{intro:1st}}
\begin{center}
\begin{quote}
``The Book of Mormon is a volume of holy scripture comparable to the Bible. It is a record of God's dealings with the ancient inhabitants of the Americas and contains the fulness of the everlasting gospel."
\end{quote}
\end{center}

\begin{table}[h!]
\centering
\label{table:intro1}
\begin{tabular*}{\textwidth}{c @{\extracolsep{\fill}}cc}
Speaker & Important Characters & Target Audience \\
\hline
\rule{0pt}{3ex}--- & Ancient inhabitants of the Americas & The reader 
\end{tabular*}
\end{table}

\subsection{Definitions\label{intro:DFN1}}
\emph{Comparable}: 
\begin{itemize}
  \item capable of being compared; having features in common with something else to permit or suggest comparison
  \item worthy of comparison
  \item usable for comparison; similar
\end{itemize}
\emph{Fulness} (or \emph{Fullness}): 
\begin{itemize}
  \item completely filled; containing all that can be held; filled to utmost capacity
  \item complete; entire; maximum
  \item of the maximum size, amount, extent, volume, etc.
\end{itemize}
\emph{Everlasting} (as adjective):
\begin{itemize}
  \item lasting forever; eternal
  \item lasting or continuing for an indefinitely long time
  \item incessant; constantly recurring
  \item wearisome; tedious
\end{itemize}
\emph{Everlasting} (as noun):
\begin{itemize}
  \item eternal duration; eternity
  \item the Everlasting, God
  \item any of various plants that retain their shape or color when dried, as certain composite plants of the genera \emph{Helichrysum}, \emph{Gnaphalium}, and \emph{Helipterum}
\end{itemize}

\subsection{Principles and Tags\label{intro:principles1}}
\begin{itemize}
\item \index{Gospel} Fulness of the gospel
\end{itemize}

\subsection{Comments\label{intro:comments1}}
Just as we say in the eighth Article of Faith, ``We believe the Bible to be the word of God as far as it is translated correctly; we also believe the Book of Mormon to be the word of God."  The Bible does contain the fulness of the gospel - faith, repentance, baptism by immersion for the remission of sins, laying on of hands for the gift of the Holy Ghost, and enduring to the end.  We use the Book of Mormon as another witness of Jesus Christ and His gospel.  In my personal opinion, the Book of Mormon is clearer in its language in describing the gospel and what we must do in order to return to God again.  When we speak of the everlasting gospel, if we look at the definitions above, we are in fact saying that we are looking at God's gospel.  The use of the word \emph{comparable} is informative as well.  We do not say that the Bible and the Book of Mormon are \emph{the same}.  Rather, they share the same information, but do so with two completely different groups of people.  Note that the first definition of comparable mentions having features in common -- the implication is that not everything is the same.

Another comment on this paragraph is that in recent years the phrase ``as does the Bible" was removed from the introduction.  I do not think that this removal was done to imply that the Bible does not contain the fulness of the gospel, but rather that the introduction is just a starting point for those who are beginning to read the Book of Mormon.  Thus, the introduction focuses on the words contained in the Book of Mormon, without detracting from the importance of the Bible.

\subsection{Additional References\label{intro:references1}}
\begin{itemize}
\item Articles of Faith 1:4, 8
\end{itemize}
%%%%%%%%%%%%%%%%%%%%%%%%%%%%%%%%%%%%%%%%%%%%%%%%%%%%%%%%%%%%%%%%%%%%%%%%%%%%%%%%%%%%%%%%%%%
%%%%%%%%%%%%%%%%%%%%%%%%%%%%%%%%%%%%%%%%%%%%%%%%%%%%%%%%%%%%%%%%%%%%%%%%%%%%%%%%%%%%%%%%%%%

%%%%%%%%%%%%%%%%%%%%%%%%%%%%%%%%%%%%%%%%%%%%%%%%%%%%%%%%%%%%%%%%%%%%%%%%%%%%%%%%%%%%%%%%%%%
%%%%%%%%%%%%%%%%%%%%%%%%%%%%%%%%%%%%%%%%%%%%%%%%%%%%%%%%%%%%%%%%%%%%%%%%%%%%%%%%%%%%%%%%%%%
\section{2nd Paragraph\label{intro:2nd}}
\begin{center}
\begin{quote}
``The book was written by many ancient prophets by the spirit of prophecy and revelation.  Their words, written on gold plates, were quoted and abridged by a prophet-historian named Mormon.  The record gives an account of two civilizations.  One came from Jerusalem in 600 \scriptsize B.C. \normalsize and afterward separated into two nations, known as the Nephites and the Lamanites.  The other came much earlier when the Lord confounded the tongues at the Tower of Babel.  This group is known as the Jaredites.  After thousands of years, all were destroyed except the Lamanites, and they are among the ancestors of the American Indians."
\end{quote}
\end{center}

\begin{table}[h!]
\centering
\label{table:intro2}
\begin{tabular*}{\textwidth}{c @{\extracolsep{\fill}}cc}
Speaker & Important Characters & Target Audience \\
\hline
\rule{0pt}{3ex}--- & Mormon; Nephites; Lamanites; Jaredites & The reader 
\end{tabular*}
\end{table}

\subsection{Definitions\label{intro:DFN2}}
\emph{Historian}:
\begin{itemize}
\item an expert in history; authority on history
\item a writer of history; chronicler
\item from Middle French \emph{historien}, from Latin \emph{historia}. As ``writer of history in the higher sense" (distinguished from a mere annalist or chronicler), from 1530s
\end{itemize} 

\subsection{Principles and Tags\label{intro:principles2}}
\begin{itemize}
\item \emph{None} (yet)
\end{itemize}

\subsection{Comments\label{intro:comments2}}
This paragraph once again reaffirms that the Book of Mormon was written by prophets by revelation.  This book did not come out of a whim like many  of the novels and stories we read today.  This book was written as a commandment of God, and as such requires (and should \emph{command}) our deepest attentions and studies.  Note that this particular paragraph was updated in recent years.  The original paragraph read, at the end ``... they are among the principal ancestors of the American Indians."  Why this was changed, I do not know, but one thought that I have on that matter is that the Nephites, Lamanites, and Jaredites may not have been the only ones to settle on the American continent. While we may not have the records of other civilizations, that does not preclude them from having living and intermingling with the groups.  After all, there are the lost ten tribes of Israel, and there is a possibility that some of those lost tribes found their way to the Americas and are also among the ancestors of the American Indians.  Something that should stick out as a big warning to everyone is the phrase, ``all were destroyed except the Lamanites."  Having read the Book of Mormon many times, and knowing how both civilizations were destroyed, this scares me especially because of the parallels I see in the United States today.  While I am not sure that the United States alone was the home of these civilizations, it seems quite possible to me that the entirety of both North and South America were well traveled by these groups, and that the commandment of God that those who dwell in the land shall serve Him applies across the board.  Thus, it seems to me, that if we do not repent, there is a strong possibility that destruction will come our way from other groups.  I have to be careful in saying this though, because I do not mean to imply that the other countries in North and South America are wicked (as could be implied from the United States being destroyed by them similar to the Nephites).  That is certainly not my place to judge, especially as I have absolutely no idea the spiritual state of those peoples.  Rather, I mean to say that the Lord in His justice will not wait much longer for His children to repent before wars begin to break out -- not in this land alone, but across the world.

\subsection{Additional References\label{intro:references2}}
\begin{itemize}
\item Anthony W. Ivins, in Conference Report, Apr. 1929, 15
\end{itemize}
%%%%%%%%%%%%%%%%%%%%%%%%%%%%%%%%%%%%%%%%%%%%%%%%%%%%%%%%%%%%%%%%%%%%%%%%%%%%%%%%%%%%%%%%%%%
%%%%%%%%%%%%%%%%%%%%%%%%%%%%%%%%%%%%%%%%%%%%%%%%%%%%%%%%%%%%%%%%%%%%%%%%%%%%%%%%%%%%%%%%%%%

%%%%%%%%%%%%%%%%%%%%%%%%%%%%%%%%%%%%%%%%%%%%%%%%%%%%%%%%%%%%%%%%%%%%%%%%%%%%%%%%%%%%%%%%%%%
%%%%%%%%%%%%%%%%%%%%%%%%%%%%%%%%%%%%%%%%%%%%%%%%%%%%%%%%%%%%%%%%%%%%%%%%%%%%%%%%%%%%%%%%%%%
\section{3rd Paragraph\label{intro:3rd}}
\begin{center}
\begin{quote}
``The crowning event recorded in the Book of Mormon is the personal ministry of the Lord Jesus Christ among the Nephites soon after his resurrection.  It puts forth the doctrines of the gospel, outlines the plan of salvation, and tells men what they must do to gain peace in this life and eternal salvation in the life to come."
\end{quote}
\end{center}

\begin{table}[h!]
\centering
\label{table:intro3}
\begin{tabular*}{\textwidth}{c @{\extracolsep{\fill}}cc}
Speaker & Important Characters & Target Audience \\
\hline
\rule{0pt}{3ex}--- & Jesus Christ; Nephites & The reader 
\end{tabular*}
\end{table}

\subsection{Definitions\label{intro:DFN3}}
\emph{Crowning}: \begin{itemize}
\item representing a level of surpassing achievement, attainment, etc.
\item forming or providing a crown, top, or summit
\item late 12c., from Old French \emph{coroner}, from \emph{corone} Related: \emph{Crowned}; \emph{crowning}. The latter in its sense of "that makes complete" is from 1650s.
\item (as \emph{crown}) the top or highest part of anything, as of a hat or a mountain
\item (as \emph{crown}) the distinction that comes from a great achievement.
\end{itemize}
\emph{Personal}: \begin{itemize}
\item of, relating to, or coming as from a particular person; individual; private
\item relating to, directed to, or intended for a particular person
\item intended for use by one person
\item referring or directed to a particular person in a disparaging or offensive sense or manner, usually involving character, behavior, appearance, etc.
\item making personal remarks or attacks
\item done, carried out, held, etc., in person
\item pertaining to or characteristic of a person or self-conscious being
\end{itemize}

\subsection{Principles and Tags\label{intro:principles3}}
\begin{itemize}
\item \index{Christ}Christ's personal ministry
\end{itemize}

\subsection{Comments\label{intro:comments3}}
Something that has always meant a lot to me is just how individually focused Christ is.  Throughout the Bible and the Book of Mormon, Christ shows us what it means to truly be selfless.  He shows such an infinite capacity for love that has always soothed my soul and driven me to work to become better, and more worthy of His love.  

This paragraph emphasizes that Christ is the focal point of this book.  His doctrine, His gospel, and the plan that He carried out are explained in greater detail in this book than in any other book.  It only makes sense that of all the books that can be studied, this one alone will bring a man closer to God than any other book, because it truly is ``the most correct of any book."

\subsection{Additional References\label{intro:references3}}
\begin{itemize}
\item \emph{None} (yet)
\end{itemize}
%%%%%%%%%%%%%%%%%%%%%%%%%%%%%%%%%%%%%%%%%%%%%%%%%%%%%%%%%%%%%%%%%%%%%%%%%%%%%%%%%%%%%%%%%%%
%%%%%%%%%%%%%%%%%%%%%%%%%%%%%%%%%%%%%%%%%%%%%%%%%%%%%%%%%%%%%%%%%%%%%%%%%%%%%%%%%%%%%%%%%%%

%%%%%%%%%%%%%%%%%%%%%%%%%%%%%%%%%%%%%%%%%%%%%%%%%%%%%%%%%%%%%%%%%%%%%%%%%%%%%%%%%%%%%%%%%%%
%%%%%%%%%%%%%%%%%%%%%%%%%%%%%%%%%%%%%%%%%%%%%%%%%%%%%%%%%%%%%%%%%%%%%%%%%%%%%%%%%%%%%%%%%%%
\section{4th Paragraph\label{intro:4th}}
\begin{center}
\begin{quote}
``After Mormon completed his writings, he delivered the account to his son Moroni, who added a few words of his own and hid up the plates in the hill Cumorah.  On September 21, 1823, the same Moroni, then a glorified, resurrected being, appeared to the Prophet Joseph Smith and instructed him relative to the ancient record and its destined translation into the English language."
\end{quote}
\end{center}

\begin{table}[h!]
\centering
\label{table:intro4}
\begin{tabular*}{\textwidth}{c @{\extracolsep{\fill}}cc}
Speaker & Important Characters & Target Audience \\
\hline
\rule{0pt}{3ex}--- & Mormon; Moroni; Joseph Smith & The reader 
\end{tabular*}
\end{table}

\subsection{Definitions\label{intro:DFN4}}
\emph{Destined}: \begin{itemize}
\item bound for a certain destination
\item ordained, appointed, or predetermined to be or do something
\item liable, planning, or intending to be or do something
\end{itemize}
\subsection{Principles and Tags\label{intro:principles4}}
\begin{itemize}
\item \index{Resurrection}Resurrection
\end{itemize}

\subsection{Comments\label{intro:comments4}}
This paragraph covers a lot of ground in such a short space -- from Moroni receiving the plates (approximately 400 \scriptsize A.D. \normalsize) to 1823, a span of nearly a millennium and a half.  The use of the word \emph{destined} is interesting to me.  While this introduction was written after the translation of the Book of Mormon into English, it is quite clear in the Book of Mormon itself that the authors knew that it would be translated into another language.

Something that has given me pause is the idea that Moroni was a glorified, \emph{resurrected} being.  Keep in mind that at the time of Moroni's death, Christ had already started the first resurrection nearly 400 years previous.  Alma indicates that there could be one resurrection and a final resurrection, a continuous resurrection, or a series of resurrections (see Alma 40).  If there were only two resurrections, that would indicate that both resurrections have already happened (the first being with Christ, the second being some time after the death of Moroni).  This does not sit well with me, so we look at the other two possibilities.  For the continuous resurrection idea, this seems slightly more plausible.  We know that when Christ was resurrected, many of the dead rose as well (see Matthew 27:52).  If that first resurrection began a continuous cycle of resurrection, this makes much more sense.  There could be an indeterminate amount of time between death and resurrection, as President Joseph F. Smith seems to indicate that Joseph Smith (who had died some 64 years previous) was still in the spirit world, and with others of the last dispensation, ``looked upon the long absence of their spirits from their bodies as a bondage." (see D\&C 138(:50)).  This bring us to the idea of a series of resurrections.  This idea stems from the fact that people simply do not experience death at the same time.  In a way, this could be viewed as a parody on the continuous resurrection idea, just with more spacing between resurrection `events.'  One thing that needs to be considered is that we have absolutely \emph{no} idea what the ordinance of resurrection entails (see \Cref{intro:references4} for further study).  Because of this, we can only speculate, but further study by faith and learning may provide further answers.

\subsection{Additional References\label{intro:references4}}
\begin{itemize}
\item See Hoskisson, P. Y., \emph{What's in a Name? The Name \emph{Cumorah}}, \emph{Journal of Book of Mormon Studies}, 13/1-2(2004):158-60, 174-75.
\item Alma 40
\item Matthew 27:52
\item D\&C 138 (esp. vs 50)
\item Matthew, R.J. \emph{Resurrection}, April 1991 \emph{Ensign}.
\item http://www.templestudy.com/2008/08/13/many-more-ordinances-including-resurrection/
\end{itemize}
%%%%%%%%%%%%%%%%%%%%%%%%%%%%%%%%%%%%%%%%%%%%%%%%%%%%%%%%%%%%%%%%%%%%%%%%%%%%%%%%%%%%%%%%%%%
%%%%%%%%%%%%%%%%%%%%%%%%%%%%%%%%%%%%%%%%%%%%%%%%%%%%%%%%%%%%%%%%%%%%%%%%%%%%%%%%%%%%%%%%%%%

%%%%%%%%%%%%%%%%%%%%%%%%%%%%%%%%%%%%%%%%%%%%%%%%%%%%%%%%%%%%%%%%%%%%%%%%%%%%%%%%%%%%%%%%%%%
%%%%%%%%%%%%%%%%%%%%%%%%%%%%%%%%%%%%%%%%%%%%%%%%%%%%%%%%%%%%%%%%%%%%%%%%%%%%%%%%%%%%%%%%%%%
\section{5th Paragraph\label{intro:5th}}
\begin{center}
\begin{quote}
``In due course the plates were delivered to Joseph Smith, who translated them by the gift and power of God.  The record is now published in many languages as a new and additional witness that Jesus Christ is the Son of the living God and that all who will come unto him and obey the laws and ordinances of his gospel may be saved."
\end{quote}
\end{center}

\begin{table}[h!]
\centering
\label{table:intro5}
\begin{tabular*}{\textwidth}{c @{\extracolsep{\fill}}cc}
Speaker & Important Characters & Target Audience \\
\hline
\rule{0pt}{3ex}--- & Joseph Smith; Jesus Christ & The reader 
\end{tabular*}
\end{table}

\subsection{Definitions\label{intro:DFN5}}
\emph{Additional}: \begin{itemize}
\item added; more; supplementary
\end{itemize}
\subsection{Principles and Tags\label{intro:principles5}}
\begin{itemize}
\item \index{Gospel} The Gospel of Jesus Christ
\end{itemize}

\subsection{Comments\label{intro:comments5}}
Again, an emphasis on the idea that Joseph Smith \emph{translated} the Book of Mormon through the gift and power of God.

At the time of this writing the Book of Mormon has been translated into over 110 different languages (see May 2015 \emph{Ensign} article).  As the gospel continues to spread, and as more and more people share their God-given gifts of language, the Book of Mormon will continue to spread throughout the world to bring everyone to a knowledge that Jesus Christ is the Savior of the world!

\subsection{Additional References\label{intro:references5}}
\begin{itemize}
\item See \Cref{titlePage:4th}
\item \emph{Book of Mormon in 110 Languages}, May 2015 \emph{Ensign}
\end{itemize}
%%%%%%%%%%%%%%%%%%%%%%%%%%%%%%%%%%%%%%%%%%%%%%%%%%%%%%%%%%%%%%%%%%%%%%%%%%%%%%%%%%%%%%%%%%%
%%%%%%%%%%%%%%%%%%%%%%%%%%%%%%%%%%%%%%%%%%%%%%%%%%%%%%%%%%%%%%%%%%%%%%%%%%%%%%%%%%%%%%%%%%%

%%%%%%%%%%%%%%%%%%%%%%%%%%%%%%%%%%%%%%%%%%%%%%%%%%%%%%%%%%%%%%%%%%%%%%%%%%%%%%%%%%%%%%%%%%%
%%%%%%%%%%%%%%%%%%%%%%%%%%%%%%%%%%%%%%%%%%%%%%%%%%%%%%%%%%%%%%%%%%%%%%%%%%%%%%%%%%%%%%%%%%%
\section{6th Paragraph\label{intro:6th}}
\begin{center}
\begin{quote}
``Concerning this record the Prophet Joseph Smith said: `I told the brethren that the Book of Mormon was the most correct of any book on earth, and the keystone of our religion, and a man would get nearer to God by abiding by its precepts, than by any other book.'"
\end{quote}
\end{center}

\begin{table}[h!]
\centering
\label{table:intro6}
\begin{tabular*}{\textwidth}{c @{\extracolsep{\fill}}cc}
Speaker & Important Characters & Target Audience \\
\hline
\rule{0pt}{3ex}--- & Joseph Smith & The reader 
\end{tabular*}
\end{table}

\subsection{Definitions\label{intro:DFN6}}
\emph{Keystone}: \begin{itemize}
\item the wedge-shaped piece at the summit of an arch, regarded as holding the other pieces in place.
\item something on which associated things depend
\end{itemize}
\emph{Abiding}:  \begin{itemize}
\item continuing without change; enduring; steadfast
\item to continue in a particular condition, attitude, relationship, etc.; last
\item to endure, sustain, or withstand without yielding or submitting
\item to wait for; await
\item to act in accord with
\item to submit to; agree to
\item to remain steadfast or faithful to; keep
\end{itemize}
\emph{Precept}: \begin{itemize}
\item a commandment or direction given as a rule of action or conduct
\item an injunction as to moral conduct; maxim
\item a procedural directive or rule, as for the performance of some technical operation
\end{itemize}
\subsection{Principles and Tags\label{intro:principles6}}
\begin{itemize}
\item \index{Keystone}The Book of Mormon is the keystone of our religion.
\end{itemize}

\subsection{Comments\label{intro:comments6}}
The idea of the Book of Mormon being the keystone of our religion is an important one.  Without the Book of Mormon, we have absolutely no claim that we are Christ's church restored to the earth.  Without the Book of Mormon, we have absolutely no claim that we have the power of God within our church in the priesthood.  We cannot claim to receive revelation, we cannot claim to have a modern prophet, and we cannot claim any of the knowledge that we have is true, if we do not have the Book of Mormon.  The Book of Mormon testifies in its purity of the simple truths of the gospel of Jesus Christ.  Because of its truth and divinity, we have what we have -- prophets, revelation, the priesthood, and a whole host of wonderful blessings that I really need to spend more time thinking about.  This book is \emph{precious}!

\subsection{Additional References\label{intro:references6}}
\begin{itemize}
\item Benson, E. T. \emph{The Book of Mormon -- Keystone of Our Religion}, October 1986 General Conference
\end{itemize}
%%%%%%%%%%%%%%%%%%%%%%%%%%%%%%%%%%%%%%%%%%%%%%%%%%%%%%%%%%%%%%%%%%%%%%%%%%%%%%%%%%%%%%%%%%%
%%%%%%%%%%%%%%%%%%%%%%%%%%%%%%%%%%%%%%%%%%%%%%%%%%%%%%%%%%%%%%%%%%%%%%%%%%%%%%%%%%%%%%%%%%%

%%%%%%%%%%%%%%%%%%%%%%%%%%%%%%%%%%%%%%%%%%%%%%%%%%%%%%%%%%%%%%%%%%%%%%%%%%%%%%%%%%%%%%%%%%%
%%%%%%%%%%%%%%%%%%%%%%%%%%%%%%%%%%%%%%%%%%%%%%%%%%%%%%%%%%%%%%%%%%%%%%%%%%%%%%%%%%%%%%%%%%%
\section{7th Paragraph\label{intro:7th}}
\begin{center}
\begin{quote}
``In addition to Joseph Smith, the Lord provided for eleven others to see the gold plates for themselves and to be special witnesses of the truth and divinity of the Book of Mormon.  Their written testimonies are included herewith as `The Testimony of Three Witnesses' and `The Testimony of Eight Witnesses.'"
\end{quote}
\end{center}

\begin{table}[h!]
\centering
\label{table:intro7}
\begin{tabular*}{\textwidth}{c @{\extracolsep{\fill}}cc}
Speaker & Important Characters & Target Audience \\
\hline
\rule{0pt}{3ex}--- & Joseph Smith; Three Witnesses; Eight Witnesses & The reader 
\end{tabular*}
\end{table}

\subsection{Definitions\label{intro:DFN7}}
\emph{Special}: \begin{itemize}
\item of a distinct or particular kind or character
\item being a particular one; particular, individual, or certain
\item pertaining or peculiar to a particular person, thing, instance, etc.; distinctive; unique
\item having a specific or particular function, purpose, etc.
\item distinguished or different from what is ordinary or usual
\item extraordinary; exceptional, as in amount or degree; especial
\item being such in an exceptional degree; particularly valued
\end{itemize}
\subsection{Principles and Tags\label{intro:principles7}}
\begin{itemize}
\item \index{Witness}Law of Witnesses
\item \index{Testimony}Testimony of the Three and Eight Witnesses, as well as Joseph Smith
\end{itemize}

\subsection{Comments\label{intro:comments7}}
Throughout the history of mankind, God has operated under the Law of Witnesses, which is in essence that at least two testimonies must come forth to establish the truth of something (see Deuteronomy 19:15, 2 Corinthians 13:1 and 1 Timothy 5:19).  The Book of Mormon is a part of this law as it, along with the Bible, testifies of the divinity of Christ.  The law is that two or three witnesses must establish the truth, so the Book of Mormon itself needs the support of this law.  This support is found in the form of the testimonies of the prophet Joseph Smith (1), the testimonies of the Three Witnesses (2), and the testimonies of the Eight Witnesses (3).  By these testimonies is the truth and divinity of the Book of Mormon established.  Any further testimonies lend further credence to the original statement, but do not change the impact.  However, I would point out that our own testimonies can act as a witness of sorts to those that are seeking the truth.  If we seek to help those around us to come to know God in a deeper way, we can plant the seed of faith with our testimony.  As they hear the testimonies of the missionaries, that can be a second witness of the truth to them.  Furthermore, as they gain their own witness from their own study of the Book of Mormon, they experience their own example of the Law of Witnesses.

Something of interest to me is that part of the testimonies of these three sets of witnesses relates to the actual existence of the gold plates.  To me, this seems like a non-issue.  If God had commanded Joseph Smith to write down these words, seemingly pulling them out of thin air, I suppose that could make it more difficult to believe.  But the words themselves offer the challenge (as shown in \Cref{intro:8th}), which in essence tells us to not take the words of those who have testified before, but to become a witness ourselves.  

\subsection{Additional References\label{intro:references7}}
\begin{itemize}
\item Ether 5:2-4
\item 2 Nephi 11:3
\item 2 Nephi 27:12-13
\item Doctrine and Covenants 17
\item Deuteronomy 19:15
\item 2 Corinthians 13:1
\item 1 Timothy 5:19
\item Encyclopedia of Mormonism, \emph{Witnesses, Law of}
\item Steven C. Harper, "Evaluating the Book of Mormon Witnesses" in \emph{Religious Educator} 11, no. 2 (2010): 37-50.
\end{itemize}
%%%%%%%%%%%%%%%%%%%%%%%%%%%%%%%%%%%%%%%%%%%%%%%%%%%%%%%%%%%%%%%%%%%%%%%%%%%%%%%%%%%%%%%%%%%
%%%%%%%%%%%%%%%%%%%%%%%%%%%%%%%%%%%%%%%%%%%%%%%%%%%%%%%%%%%%%%%%%%%%%%%%%%%%%%%%%%%%%%%%%%%

%%%%%%%%%%%%%%%%%%%%%%%%%%%%%%%%%%%%%%%%%%%%%%%%%%%%%%%%%%%%%%%%%%%%%%%%%%%%%%%%%%%%%%%%%%%
%%%%%%%%%%%%%%%%%%%%%%%%%%%%%%%%%%%%%%%%%%%%%%%%%%%%%%%%%%%%%%%%%%%%%%%%%%%%%%%%%%%%%%%%%%%
\section{8th Paragraph\label{intro:8th}}
\begin{center}
\begin{quote}
``We invite all men everywhere to read the Book of Mormon, to ponder in their hearts the message it contains, and then to ask God, the Eternal Father, in the name of Christ if the book is true.  Those who pursue this course and ask in faith will gain a testimony of its truth and divinity by the power of the Holy Ghost. (See Moroni 10:3-5.)"
\end{quote}
\end{center}

\begin{table}[h!]
\centering
\label{table:intro8}
\begin{tabular*}{\textwidth}{c @{\extracolsep{\fill}}cc}
Speaker & Important Characters & Target Audience \\
\hline
\rule{0pt}{3ex}--- & All men; God; Christ; Holy Ghost & The reader 
\end{tabular*}
\end{table}

\subsection{Definitions\label{intro:DFN8}}
\emph{Ponder}: \begin{itemize}
\item to consider something deeply and thoroughly; meditate (often followed by over or upon)
\item to weigh carefully in the mind; consider thoughtfully
\item to estimate the worth of, to appraise
\end{itemize}
\emph{Pursue}: \begin{itemize}
\item to strive to gain; seek to attain or accomplish (an end, object, purpose, etc.)
\item to proceed in accordance with (a method, plan, etc.)
\item to carry on or continue (a course of action, a train of thought, an inquiry, studies, etc.)
\item to continue
\end{itemize}
\subsection{Principles and Tags\label{intro:principles8}}
\begin{itemize}
\item \index{Prayer}Prayer
\item \index{Ask and Receive}Ask and ye shall receive
\item \index{If-Then}If-Then
\end{itemize}

\subsection{Comments\label{intro:comments8}}
This book has a powerful promise.  It promises that if we ask God, in the name of Christ, if the book is true, then we will know by the power of the Holy Ghost that it is true.  This book encourages us to test it, to actually hold it up to the light it professes to have, and determine if it is true, or if it is simply a good story.  The promise is that its truth will be revealed to you. The requirement is that you simply put in the effort to know.  You can't test something if you don't have the necessary knowledge - a test is a way to determine the difference between what you know and what is (taught to be) right.

\subsection{Additional References\label{intro:references8}}
\begin{itemize}
\item Moroni 10:3-5
\end{itemize}
%%%%%%%%%%%%%%%%%%%%%%%%%%%%%%%%%%%%%%%%%%%%%%%%%%%%%%%%%%%%%%%%%%%%%%%%%%%%%%%%%%%%%%%%%%%
%%%%%%%%%%%%%%%%%%%%%%%%%%%%%%%%%%%%%%%%%%%%%%%%%%%%%%%%%%%%%%%%%%%%%%%%%%%%%%%%%%%%%%%%%%%

%%%%%%%%%%%%%%%%%%%%%%%%%%%%%%%%%%%%%%%%%%%%%%%%%%%%%%%%%%%%%%%%%%%%%%%%%%%%%%%%%%%%%%%%%%%
%%%%%%%%%%%%%%%%%%%%%%%%%%%%%%%%%%%%%%%%%%%%%%%%%%%%%%%%%%%%%%%%%%%%%%%%%%%%%%%%%%%%%%%%%%%
\section{9th Paragraph\label{intro:9th}}
\begin{center}
\begin{quote}
``Those who gain this divine witness from the Holy Spirit will also come to know by the same power that Jesus Christ is the Savior of the world, that Joseph Smith is his revelator and prophet in these last days, and that The Church of Jesus Christ of Latter-day Saints is the Lord's kingdom once again established on the earth, preparatory to the second coming of the Messiah."
\end{quote}
\end{center}

\begin{table}[h!]
\centering
\label{table:intro9}
\begin{tabular*}{\textwidth}{c @{\extracolsep{\fill}}cc}
Speaker & Important Characters & Target Audience \\
\hline
\rule{0pt}{3ex}--- & Holy Spirit; Jesus Christ; Joseph Smith & The reader 
\end{tabular*}
\end{table}

\subsection{Definitions\label{intro:DFN9}}
\emph{Divine}: \begin{itemize}
\item of or relating to a god, especially the Supreme Being
\item addressed, appropriated, or devoted to God or a god; religious; sacred
\item proceeding from God or a god
\item godlike; characteristic of or befitting a deity
\item heavenly; celestial
\item extremely good; unusually lovely
\end{itemize}
\emph{Revelator}: \begin{itemize}
\item a person who makes a revelation
\end{itemize}
\emph{Preparatory}: \begin{itemize}
\item serving or designed to prepare
\item preliminary; introductory
\item of or relating to training that prepares for more advanced education
\end{itemize}
\subsection{Principles and Tags\label{intro:principles9}}
\begin{itemize}
\item \index{Witness}Witness from God
\item \index{Second Coming}Second Coming
\end{itemize}

\subsection{Comments\label{intro:comment9}}
God teaches us in a way that helps us best to understand.  If we have learned something from Him in a certain way (i.e. a burning of the bosom, or words spoken into our minds), it is a safe assumption that further light and knowledge from Him will come similarly.  The scriptures state that God is not a changeable being (see Moroni 8:18), so He will not suddenly try a method on us that won't work.  That being said, God has known us much longer than we currently know ourselves.  We existed with Him before the foundation of the world (see Jeremiah 1:5), and since we cannot remember that time in our life, God knows us intimately, and much better than we know ourselves.  Because of this knowledge, God knows \emph{exactly} how to teach us.  There may be ways that we have to prepare ourselves for, but God doesn't `experiment' with His children.  After all, this existence is a test for us as His children to see whom we will follow: God, or Satan.

It is important to note that the same feelings and testimony that we gain regarding the truthfulness of the Book of Mormon will also apply to Christ being our Savior, Joseph Smith being Christ's prophet on the earth in the last days, and the Church of Jesus Christ of Latter-day Saints being the Lord's established kingdom on the earth.  In much the same way, the Holy Ghost will testify of the truthfulness of the Bible, and by extension that Christ is our Savior, that prophets were called in those days, and that Christ's kingdom had been established on the earth (and here I will add that His kingdom was taken from the earth due to wickedness).  Both the Bible and the Book of Mormon warn of the things to come in the last days.  We must prepare now for the coming of our Lord and Savior, Jesus Christ.  We must be found worthy and prepared at His coming.  Only God knows exactly when Christ will return, but until then we must vigilantly watch and pray, and prepare our lamps for the Bridegroom.

\subsection{Additional References\label{intro:references9}}
\begin{itemize}
\item Moroni 8:18
\item Jeremiah 1:5
\end{itemize}
%%%%%%%%%%%%%%%%%%%%%%%%%%%%%%%%%%%%%%%%%%%%%%%%%%%%%%%%%%%%%%%%%%%%%%%%%%%%%%%%%%%%%%%%%%%
%%%%%%%%%%%%%%%%%%%%%%%%%%%%%%%%%%%%%%%%%%%%%%%%%%%%%%%%%%%%%%%%%%%%%%%%%%%%%%%%%%%%%%%%%%%

\section{General Comments on \Cref{chapter:intro}}
I too add my witness to that of Joseph Smith -- This book truly is the most correct of any book.  I have read it.  I study it, and learn from it every time I read.  I have asked God about it's truthfulness, and I have felt His confirming witness of its truth.  I know that if you truly desire to know if this book is the truth, and if this book does in fact come from God, you can ask Him who is the source of all truth.  God will not lead His children astray, because His love for each individual is all-encompassing.  He will answer, if you ask in sincerity of heart.

\chapter{Testimonies of the Witnesses\label{chapter:testimonies}}
%%%%%%%%%%%%%%%%%%%%%%%%%%%%%%%%%%%%%%%%%%%%%%%%%%%%%%%%%%%%%%%%%%%%%%%%%%%%%%%%%%%%%%%%%%%
%%%%%%%%%%%%%%%%%%%%%%%%%%%%%%%%%%%%%%%%%%%%%%%%%%%%%%%%%%%%%%%%%%%%%%%%%%%%%%%%%%%%%%%%%%%
\section{Three Witnesses\label{3witness}}
\begin{center}
\begin{quote}
``Be it known unto all nations, kindreds, tongues, and people, unto whom this work shall come: That we, through the grace of God the Father and our Lord Jesus Christ, have seen the plates which contain this record, which is a record of the people of Nephi, and also of the Lamanites, their brethren, and also of the people of Jared, who came from the tower which hath been spoken.  And we also know that they have been translated by the gift and power of God, for his voice hath declared it unto us; wherefore we know of a surety that the work is true.  And we also testify that we have seen the engravings which are upon the plates; and they have been shown unto us by the power of God, and not of man.  And we declare with words of soberness, that an angel of God came down from heaven, and he brought and laid before our eyes, that we beheld and saw the plates, and the engravings thereon; and we know that it is by the grace of God the Father, and our Lord Jesus Christ, that we beheld and bear record that these things are true.  And it is marvelous in our eyes.  Nevertheless, the voice of the Lord commanded us that we should bear record of it; wherefore, to be obedient unto the commandments of God, we bear testimony of these things. And we know that if we are faithful in Christ, we shall rid our garments of the blood of all men, and be found spotless before the judgment-seat of Christ, and shall dwell with him eternally in the heavens.  And the honor be to the Father, and to the Son, and to the Holy Ghost, which is one God.  Amen."
\end{quote}
\end{center}

\begin{table}[h!]
\centering
\label{table:3witness}
\begin{tabular*}{\textwidth}{l @{\extracolsep{\fill}}cc}
Speaker & Important Characters & Target Audience \\
\hline
\rule{0pt}{3ex}Oliver Cowdery & God & \multirow{3}{*}{The reader} \\
David Whitmer & Jesus Christ \\
Martin Harris & Holy Ghost
\end{tabular*}
\end{table}

\subsection{Principles and Tags\label{3witness:principles}}
\begin{itemize}
\item \index{Witness}Witness of the Book of Mormon
\item \index{If-Then}If-Then
\item \index{Testimony}Testimony
\end{itemize}

\subsection{Comments\label{3witness:comments}}
Something that is interesting to me is looking into the subsequent events that occurred to the three witnesses.  After having the incredible experience which they explained, some years later all of them fell away from the church.  Some of them did come back, but it astounds me that they did fall away.  Even more astonishing is the fact that despite their falling away, they never did deny the testimony written here (see \Cref{intro:references7}).  These three men, who saw an angel, who viewed the plates upon which were written the history of the Nephites and the Lamanites, as well as an abridgment of the history of the Jaredites, fell away from the church that they knew to be true.  This stands as a warning to me of the need for constant vigilance in keeping my faith strong.

An important point about this witness: the physical senses with which these men testify of the truth are sight and sound - \emph{seen} the engravings... \emph{shown} unto us... laid before our \emph{eyes} ... been \emph{spoken}... \emph{voice} hath declared.  They received a commandment from God that they should bear record of the Book of Mormon, and they obediently kept that commandment.  This is important!  If they had not kept that commandment, they likely would have been destroyed, for ``no unhallowed hand can stop the work from progressing."  That said, that is my \emph{opinion}, and should not be taken as fact.

\subsection{Additional References\label{3witness:references}}
\begin{itemize}
\item \emph{None} (yet)
\end{itemize}
%%%%%%%%%%%%%%%%%%%%%%%%%%%%%%%%%%%%%%%%%%%%%%%%%%%%%%%%%%%%%%%%%%%%%%%%%%%%%%%%%%%%%%%%%%%
%%%%%%%%%%%%%%%%%%%%%%%%%%%%%%%%%%%%%%%%%%%%%%%%%%%%%%%%%%%%%%%%%%%%%%%%%%%%%%%%%%%%%%%%%%%

%%%%%%%%%%%%%%%%%%%%%%%%%%%%%%%%%%%%%%%%%%%%%%%%%%%%%%%%%%%%%%%%%%%%%%%%%%%%%%%%%%%%%%%%%%%
%%%%%%%%%%%%%%%%%%%%%%%%%%%%%%%%%%%%%%%%%%%%%%%%%%%%%%%%%%%%%%%%%%%%%%%%%%%%%%%%%%%%%%%%%%%
\section{Eight Witnesses\label{8witness}}
\begin{center}
\begin{quote}
``Be it known unto all nations, kindreds, tongues, and people, unto whom this work shall come: That Joseph Smith, Jun., the translator of this work, has shown unto us the plates of which hath been spoken, which have the appearance of gold; and as many of the leaves as the said Smith has translated we did handle with our hands and we also saw the engravings thereon, all of which has the appearance of ancient work, and of curious workmanship.  And this we bear record with words of soberness, that the said Smith has shown unto us, for we have seen and hefted, and know of a surety that the said Smith has got the plates of which we have spoken.  And we give our names unto the world, to witness unto the world that which we have seen.  And we lie not, God bearing witness of it"
\end{quote}
\end{center}

\begin{table}[h!]
\centering
\label{table:8witness}
\begin{tabular*}{\textwidth}{l @{\extracolsep{\fill}}cc}
Speaker & Important Characters & Target Audience \\
\hline
\rule{0pt}{3ex}Christian Whitmer & \multirow{8}{*}{Joseph Smith, Junior} & \multirow{8}{*}{The reader} \\ 
Jacob Whitmer \\
Peter Whitmer, Jun. \\
John Whitmer \\
Hiram Page \\
Joseph Smith, Sen. \\
Hyrum Smith \\
Samuel H. Smith
\end{tabular*}
\end{table}

\subsection{Definitions\label{8witness:DFN}}
\emph{Curious}: \begin{itemize}
\item eager to learn or know; inquisitive
\item prying; meddlesome
\item arousing or exciting speculation, interest, or attention through being inexplicable or highly unusual; odd; strange
\item \emph{Archaic}:
\begin{itemize}
\item made or prepared skillfully
\item done with painstaking accuracy or attention to detail
\item careful; fastidious
\item marked by intricacy or subtlety
\end{itemize}
\end{itemize}
\emph{Soberness}: \begin{itemize}
\item not intoxicated or drunk
\item habitually temperate, especially in the use of liquor
\item quiet or sedate in demeanor, as persons
\item marked by seriousness, gravity, solemnity, etc., as of demeanor, speech, etc.
\item subdued in tone, as color; not colorful or showy, as clothes
\item free from excess, extravagance, or exaggeration
\item showing self-control
\end{itemize}
\emph{Hefted}: \begin{itemize}
\item to test the weight of by lifting and balancing
\item to heave; hoist
\end{itemize}
\subsection{Principles and Tags\label{8witness:principles}}
\begin{itemize}
\item \index{Testimony}Testimony
\end{itemize}

\subsection{Comments\label{8witness:comments}}
This witness is slightly different from the witness given by the Three Witnesses.  This testimony explicitly states that they held the plates with their hands, and felt it's physical form.  The witness given by the Three says they \emph{saw} the plates as shown unto them by an angel.  This is a way to verify the witnesses, as one could be said to have had a spiritual experience, while they others had a physical experience.

Important points about this witness: the witness is based on the physical senses of sight and touch, different from the witness given by the Three Witnesses.  The words used are: \emph{shown} unto us... \emph{handle} with our hands... we also \emph{saw}... \emph{shown} unto us... we have \emph{seen} and \emph{hefted}.

\subsection{Additional References\label{8witness:references}}
\begin{itemize}
\item \emph{None} (yet)
\end{itemize}
%%%%%%%%%%%%%%%%%%%%%%%%%%%%%%%%%%%%%%%%%%%%%%%%%%%%%%%%%%%%%%%%%%%%%%%%%%%%%%%%%%%%%%%%%%%
%%%%%%%%%%%%%%%%%%%%%%%%%%%%%%%%%%%%%%%%%%%%%%%%%%%%%%%%%%%%%%%%%%%%%%%%%%%%%%%%%%%%%%%%%%%

%%%%%%%%%%%%%%%%%%%%%%%%%%%%%%%%%%%%%%%%%%%%%%%%%%%%%%%%%%%%%%%%%%%%%%%%%%%%%%%%%%%%%%%%%%%
%%%%%%%%%%%%%%%%%%%%%%%%%%%%%%%%%%%%%%%%%%%%%%%%%%%%%%%%%%%%%%%%%%%%%%%%%%%%%%%%%%%%%%%%%%%
\section{Testimony of Joseph Smith\label{JStestimony}}
\subsection{1st paragraph\label{js:1st}}
\begin{center}
\begin{quote}
``The Prophet Joseph Smith's own words about the coming forth of the Book of Mormon are:"
\end{quote}
\end{center}

\begin{table}[h!]
\centering
\label{table:js1}
\begin{tabular*}{\textwidth}{l @{\extracolsep{\fill}}cc}
Speaker & Important Characters & Target Audience \\
\hline
\rule{0pt}{3ex}--- & Joseph Smith & The reader 
\end{tabular*}
\end{table}

\subsubsection{Principles and Tags\label{js:principles1}}
\begin{itemize}
\item \index{}\emph{None} (yet)
\end{itemize}

\subsubsection{Comments\label{js:comments1}}
Not much to say here, other than what follows is the first-hand account of Joseph Smith's experience with the coming forth of the Book of Mormon.

\subsubsection{Additional References\label{js:references1}}
\begin{itemize}
\item \emph{None} (yet)
\end{itemize}
%%%%%%%%%%%%%%%%%%%%%%%%%%%%%%%%%%%%%%%%%%%%%%%%%%%%%%%%%%%%%%%%%%%%%%%%%%%%%%%%%%%%%%%%%%%
%%%%%%%%%%%%%%%%%%%%%%%%%%%%%%%%%%%%%%%%%%%%%%%%%%%%%%%%%%%%%%%%%%%%%%%%%%%%%%%%%%%%%%%%%%%

%%%%%%%%%%%%%%%%%%%%%%%%%%%%%%%%%%%%%%%%%%%%%%%%%%%%%%%%%%%%%%%%%%%%%%%%%%%%%%%%%%%%%%%%%%%
%%%%%%%%%%%%%%%%%%%%%%%%%%%%%%%%%%%%%%%%%%%%%%%%%%%%%%%%%%%%%%%%%%%%%%%%%%%%%%%%%%%%%%%%%%%
\subsection{2nd paragraph\label{js:2nd}}
\begin{center}
\begin{quote}
``On the evening of the... twenty-first of September [1823]... I betook myself to prayer and supplication to Almighty God...."
\end{quote}
\end{center}

\begin{table}[h!]
\centering
\label{table:js2}
\begin{tabular*}{\textwidth}{l @{\extracolsep{\fill}}cc}
Speaker & Important Characters & Target Audience \\
\hline
\rule{0pt}{3ex}Joseph Smith & Joseph Smith; God & The reader 
\end{tabular*}
\end{table}

\subsubsection{Definitions\label{js:DFN2}}
\emph{Betook}: \begin{itemize}
\item to cause to go (usually used reflexively)
\item \emph{Archaic}: to resort or have recourse to.
\end{itemize}
\emph{Supplication}: \begin{itemize}
\item an act or instance of supplicating; humble prayer, entreaty, or petition
\end{itemize}

\subsubsection{Principles and Tags\label{js:principles2}}
\begin{itemize}
\item \index{Prayer}Prayer
\item \index{Record Keeping}Record Keeping
\end{itemize}

\subsubsection{Comments\label{js:comments2}}
If supplication is `humble prayer,' this sentence reads that Joseph Smith turned to prayer and humble prayer to God.  This indicates a gradation of sorts in the types of prayers we can offer.  Prayer, in and of itself, is the act of communicating with God.  The Bible Dictionary definition of prayer states that ``Prayer is the act by which the will of the Father and the will of the child are brought into correspondence with each other." Simple prayer (the first prayer that Joseph Smith indicates) may be less focused on aligning our will with God, but rather more focused on saying things that are going on.  Humble prayer on the other hand evokes an idea of submissiveness, which aligns well with the idea given in the Bible Dictionary definition.

There is also a subtle reminder of the importance of keeping a personal record.  I don't know when Joseph Smith wrote this particular testimony, but it seems to me like he may have referenced a journal to pull out the specific date.  It is absolutely possible that he remembered the day though.

\subsubsection{Additional References\label{js:references2}}
\begin{itemize}
\item See \href{https://byustudies.byu.edu/history-of-the-church}{History of the Church of Jesus Christ of Latter-day Saints}
\item Joseph Smith -- History 1:29
\end{itemize}
%%%%%%%%%%%%%%%%%%%%%%%%%%%%%%%%%%%%%%%%%%%%%%%%%%%%%%%%%%%%%%%%%%%%%%%%%%%%%%%%%%%%%%%%%%%
%%%%%%%%%%%%%%%%%%%%%%%%%%%%%%%%%%%%%%%%%%%%%%%%%%%%%%%%%%%%%%%%%%%%%%%%%%%%%%%%%%%%%%%%%%%

%%%%%%%%%%%%%%%%%%%%%%%%%%%%%%%%%%%%%%%%%%%%%%%%%%%%%%%%%%%%%%%%%%%%%%%%%%%%%%%%%%%%%%%%%%%
%%%%%%%%%%%%%%%%%%%%%%%%%%%%%%%%%%%%%%%%%%%%%%%%%%%%%%%%%%%%%%%%%%%%%%%%%%%%%%%%%%%%%%%%%%%
\subsection{3rd paragraph\label{js:3rd}}
\begin{center}
\begin{quote}
``While I was thus in the act of calling upon God, I discovered a light appearing in my room, which continued to increase until the room was lighter than at noonday, when immediately a personage appeared at my bedside, standing in the air, for his feet did not touch the floor."
\end{quote}
\end{center}

\begin{table}[h!]
\centering
\label{table:js3}
\begin{tabular*}{\textwidth}{l @{\extracolsep{\fill}}cc}
Speaker & Important Characters & Target Audience \\
\hline
\rule{0pt}{3ex}Joseph Smith & --- & The reader 
\end{tabular*}
\end{table}

\subsubsection{Definitions\label{js:DFN3}}
\emph{Personage}: \begin{itemize}
\item a person of distinction or importance.
\item any person.
\item a character in a play, story, etc.
\item from late Middle English word meaning body or image (statue, portrait) of a person
\end{itemize}

\subsubsection{Principles and Tags\label{js:principles3}}
\begin{itemize}
\item \index{Ministry of Angels}Ministry of Angels
\end{itemize}

\subsubsection{Comments\label{js:comments3}}
This is an interesting experience.  A young man, about 17 years old, decides to pray to God.  In so doing, he notices that his room is becoming brighter (whether or not his eyes were open at this point or not doesn't really matter, as light can be sensed even with eyes closed).  He may have continued his prayer, perhaps thinking that a family member may have been coming to check on him.  Imagine his surprise as the light continues to grow brighter, making it less likely that it's a lamp or candle.  Opening his eyes, perhaps he needs to shield his eyes from the light that continues to grow brighter, until his room was brighter than at noon.  As noon is generally seen as the height of the sun's travel across the sky, his room must have been painfully bright.  At this point of intense light, seemingly out of nowhere a personage appears in the air.  In today's culture, we might simply dismiss the experience as one of ``aliens" or hallucinogenics.  Some may even try to compare this experience to that of a magic show, where sleight of hand and trickery creates the intended effects. Joseph Smith may have been simply awestruck for a moment as he processed what he was seeing (see \Cref{js:5th}). If I had this experience, I don't know what I would  be thinking.  Joseph has had the unique experience of having God and Jesus Christ \emph{personally} answer his prayers.  While I too have had answers to prayers, none have been so dramatic as the experience of Joseph Smith.  Perhaps part of the reason that these experiences are not talked about is because of the thoughts mentioned above.  I almost feel that if I had this experience, I would (rather irreverently) check to make sure it wasn't just some trick, or image.  We live in a world that has access to rather remarkable technology that can imitate many things, even sacred, spiritual things.  Thus it requires our utmost attention, and the spirit of discernment, to know what is of God, and what is not.

\subsubsection{Additional References\label{js:references3}}
\begin{itemize}
\item See \href{https://byustudies.byu.edu/history-of-the-church}{History of the Church of Jesus Christ of Latter-day Saints}
\item Joseph Smith -- History 1:30
\end{itemize}
%%%%%%%%%%%%%%%%%%%%%%%%%%%%%%%%%%%%%%%%%%%%%%%%%%%%%%%%%%%%%%%%%%%%%%%%%%%%%%%%%%%%%%%%%%%
%%%%%%%%%%%%%%%%%%%%%%%%%%%%%%%%%%%%%%%%%%%%%%%%%%%%%%%%%%%%%%%%%%%%%%%%%%%%%%%%%%%%%%%%%%%

%%%%%%%%%%%%%%%%%%%%%%%%%%%%%%%%%%%%%%%%%%%%%%%%%%%%%%%%%%%%%%%%%%%%%%%%%%%%%%%%%%%%%%%%%%%
%%%%%%%%%%%%%%%%%%%%%%%%%%%%%%%%%%%%%%%%%%%%%%%%%%%%%%%%%%%%%%%%%%%%%%%%%%%%%%%%%%%%%%%%%%%
\subsection{4th Paragraph\label{js:4th}}
\begin{center}
\begin{quote}
``He had on a loose robe of most exquisite whiteness.  It was a whiteness beyond anything earthly I had ever seen; nor do I believe that any earthly thing could be made to appear so exceedingly white and brilliant.  His hands were naked, and his arms also, a little above the wrists; so, also, were his feet naked, as were his legs, a little above the ankles.  His head and neck were also bare.  I could discover that he had no other clothing on but this robe, as it was open, so that I could see into his bosom."
\end{quote}
\end{center}

\begin{table}[h!]
\centering
\label{table:js4}
\begin{tabular*}{\textwidth}{l @{\extracolsep{\fill}}cc}
Speaker & Important Characters & Target Audience \\
\hline
\rule{0pt}{3ex}Joseph Smith & --- & The reader 
\end{tabular*}
\end{table}

\subsubsection{Definitions\label{js:DFN4}}
\emph{Exquisite}: \begin{itemize}
\item of special beauty or charm, or rare and appealing excellence, as a face, a flower, coloring, music, or poetry.
\item extraordinarily fine or admirable; consummate
\item intense; acute, or keen, as pleasure or pain
\item of rare excellence of production or execution, as works of art or workmanship
\item keenly or delicately sensitive or responsive
\item of particular refinement or elegance, as taste, manners, etc., or persons
\item carefully sought out, chosen, ascertained, devised, etc
\end{itemize}
\emph{Bosom}: \begin{itemize}
\item the breast of a human being
\item the breasts of a woman
\item the part of a garment that covers the breast
\item the breast, conceived of as the center of feelings or emotions
\item something likened to the human breast
\item a state of enclosing intimacy; warm closeness
\end{itemize}

\subsubsection{Principles and Tags\label{js:principles4}}
\begin{itemize}
\item \index{}\emph{None} (yet)
\end{itemize}

\subsubsection{Comments\label{js:comments4}}
An interesting reference to read on this particular experience can be found in the references below (\Cref{js:references4}).  There seems to be a decent effort to be neutral, but there also seems to be a tendency (in my opinion) to focus on criticisms, rather than supports.  That being said, something that is interesting to me is the apparel of the visiting angle.  As a member of the church, I have a better understanding of sacred apparel than those who are not, and I am intrigued at the differences between what a typical, faithful Latter-day Saint member wears, and what this visiting personage from God wears.  The garments worn by members of the church are meant to shield and protect the wearer from Satan, insofar as the individual is faithful.  I feel that I can safely assume that angels of God would have similar protections.  Joseph Smith describes the clothing as white, which according to Western Oregon University symbolizes purity, virginity, innocence and birth, and clarifies that it is ``beyond anything earthly [he] had ever seen."  Perhaps the light itself is the protection, as Satan is sometimes referred to as ``The Prince of Darkness."

\subsubsection{Additional References\label{js:references4}}
\begin{itemize}
\item See \url{http://www.mormonthink.com/moroniweb.htm}
\item See \href{https://www.wou.edu/wp/exhibits/files/2015/07/christianity.pdf}{Color Symbolism in Christianity}
\item Joseph Smith -- History 1:31
\end{itemize}
%%%%%%%%%%%%%%%%%%%%%%%%%%%%%%%%%%%%%%%%%%%%%%%%%%%%%%%%%%%%%%%%%%%%%%%%%%%%%%%%%%%%%%%%%%%
%%%%%%%%%%%%%%%%%%%%%%%%%%%%%%%%%%%%%%%%%%%%%%%%%%%%%%%%%%%%%%%%%%%%%%%%%%%%%%%%%%%%%%%%%%%

%%%%%%%%%%%%%%%%%%%%%%%%%%%%%%%%%%%%%%%%%%%%%%%%%%%%%%%%%%%%%%%%%%%%%%%%%%%%%%%%%%%%%%%%%%%
%%%%%%%%%%%%%%%%%%%%%%%%%%%%%%%%%%%%%%%%%%%%%%%%%%%%%%%%%%%%%%%%%%%%%%%%%%%%%%%%%%%%%%%%%%%
\subsection{5th Paragraph\label{js:5th}}
\begin{center}
\begin{quote}
``Not only was his robe exceedingly white, but his whole person was glorious beyond description, and his countenance truly like lightning.  The room was exceedingly light, but not so very bright as immediately around his person.  When I first looked upon him, I was afraid; but the fear soon left me."
\end{quote}
\end{center}

\begin{table}[h!]
\centering
\label{table:js5}
\begin{tabular*}{\textwidth}{l @{\extracolsep{\fill}}cc}
Speaker & Important Characters & Target Audience \\
\hline
\rule{0pt}{3ex}Joseph Smith & --- & The reader 
\end{tabular*}
\end{table}

\subsubsection{Definitions\label{js:DFN5}}
\emph{Glorious}: \begin{itemize}
\item delightful; wonderful; completely enjoyable
\item conferring glory
\item full of glory; entitled to great renown
\item brilliantly beautiful or magnificent; splendid
\end{itemize}
\emph{Countenance}: \begin{itemize}
\item appearance, especially the look or expression of the face
\item the face; visage
\item calm facial expression; composure
\item approval or favor; encouragement; moral support
\end{itemize}

\subsubsection{Principles and Tags\label{js:principles5}}
\begin{itemize}
\item \index{}\emph{None} (yet)
\end{itemize}

\subsubsection{Comments\label{js:comments5}}
I find it interesting that the gradient of light intensity was negative as Joseph looked closer at the angel.  Joseph Smith says that the ``room was exceedingly light, but not so very bright as immediately around his person."  I could be reading this incorrectly, but it seems like it was less bright around the angel than in the room in general.  Perhaps this was simply a courtesy to Joseph Smith?  If this was truly a spiritual experience, rather than a physical experience, then why would there be a decrease of light?  That being said, having a countenance like lightning is already very bright.  While there doesn't seem to be any hard data on the intensity of light from a lightning strike, lightning is known to be very bright (see \Cref{js:references5} below).  There also doesn't seem to be any hard facts on what intensity of light the human eye can withstand without damage.  Regardless of the mechanism(s), we do know that God has all power, so if Joseph Smith saw what he said he saw, and possibly could have experienced physical disability from it, God could have prevented/healed any possible physical damage.

I appreciate the honesty of Joseph Smith when he said he was initially afraid.  I would be confused if he wasn't afraid - bright lights, ignored laws of physics, and unearthly whiteness all could and would be concerning to a human mind.  I wonder what my response would be to an experience like this?

\subsubsection{Additional References\label{js:references5}}
\begin{itemize}
\item See \href{http://academlib.com/5778/education/bright_lightning}{How Bright is Lightning?}
\item Article on effect of bright light on the eyes \href{https://www.quora.com/What-is-the-maximum-light-intensity-that-a-human-eye-can-withstand-without-being-damaged}{here}.
\item Joseph Smith -- History 1:32
\end{itemize}
%%%%%%%%%%%%%%%%%%%%%%%%%%%%%%%%%%%%%%%%%%%%%%%%%%%%%%%%%%%%%%%%%%%%%%%%%%%%%%%%%%%%%%%%%%%
%%%%%%%%%%%%%%%%%%%%%%%%%%%%%%%%%%%%%%%%%%%%%%%%%%%%%%%%%%%%%%%%%%%%%%%%%%%%%%%%%%%%%%%%%%%

%%%%%%%%%%%%%%%%%%%%%%%%%%%%%%%%%%%%%%%%%%%%%%%%%%%%%%%%%%%%%%%%%%%%%%%%%%%%%%%%%%%%%%%%%%%
%%%%%%%%%%%%%%%%%%%%%%%%%%%%%%%%%%%%%%%%%%%%%%%%%%%%%%%%%%%%%%%%%%%%%%%%%%%%%%%%%%%%%%%%%%%
\subsection{6th Paragraph\label{js:6th}}
\begin{center}
\begin{quote}
``He called me by name, and said unto me that he was a messenger sent from the presence of God to me, and that his name was Moroni; that God had a work for me to do; and that my name should be had for good and evil among all nations, kindreds, and tongues, or that it should be both good and evil spoken of among all people."
\end{quote}
\end{center}

\begin{table}[h!]
\centering
\label{table:js6}
\begin{tabular*}{\textwidth}{l @{\extracolsep{\fill}}cc}
Speaker & Important Characters & Target Audience \\
\hline
\rule{0pt}{3ex}Joseph Smith & Moroni & The reader 
\end{tabular*}
\end{table}

\subsubsection{Principles and Tags\label{js:principles6}}
\begin{itemize}
\item \index{Persecution}Persecution of the Righteous
\item \index{Personal God}Personal God
\end{itemize}

\subsubsection{Comments\label{js:comments6}}
Something significant here is the first five words: Joseph Smith was called \emph{by name}.  He had had no prior experiences with said angel (at least that he recorded), and identified that he was initially afraid of this heavenly being.  It is only after getting Joseph Smith's attention that the messenger states that he is sent from God, and that his name is Moroni.  He then proceeds to tell Joseph Smith of the work that he will be required to do as he goes forward, and what sort of results he can expect.  Doctrine and Covenants 121 states that ``many are called, but few are chosen... because their hearts are set so much upon the things of this world, and aspire to the honors of men, that they do not learn that... the powers of heaven cannot be controlled nor handled only upon the principles of righteousness."  From this we can infer that Joseph Smith's heart was \emph{not} set on the things of this world; that he did \emph{not} aspire to the honors of men; and that he \emph{had} learned (at least in some degree) that that powers of heaven were given based on principles of righteousness.

Moroni further explains, and quite clearly, that Joseph's life will not be easy.  Good and evil have been spoken of him, and will continue to be spoken of him for the rest of the existence of the earth (and possibly beyond that). So many people who have gained a testimony thank him as the prophet and seer that brought about the Restoration of the gospel of Jesus Christ.  Many other people cry that Joseph Smith was a fraud, and created a church for his own gain.  My own personal testimony is that Joseph Smith was God's chosen prophet.  The Spirit of God has confirmed to me that this is the case many times over, and I have absolute confidence that this will continue throughout my life.

\subsubsection{Additional References\label{js:references6}}
\begin{itemize}
\item D\&C 121:34
\item Joseph Smith -- History 1:33
\end{itemize}
%%%%%%%%%%%%%%%%%%%%%%%%%%%%%%%%%%%%%%%%%%%%%%%%%%%%%%%%%%%%%%%%%%%%%%%%%%%%%%%%%%%%%%%%%%%
%%%%%%%%%%%%%%%%%%%%%%%%%%%%%%%%%%%%%%%%%%%%%%%%%%%%%%%%%%%%%%%%%%%%%%%%%%%%%%%%%%%%%%%%%%%

%%%%%%%%%%%%%%%%%%%%%%%%%%%%%%%%%%%%%%%%%%%%%%%%%%%%%%%%%%%%%%%%%%%%%%%%%%%%%%%%%%%%%%%%%%%
%%%%%%%%%%%%%%%%%%%%%%%%%%%%%%%%%%%%%%%%%%%%%%%%%%%%%%%%%%%%%%%%%%%%%%%%%%%%%%%%%%%%%%%%%%%
\subsection{7th Paragraph\label{js:7th}}
\begin{center}
\begin{quote}
``He said there was a book deposited, written upon gold plates, giving an account of the former inhabitants of this continent, and the source from whence they sprang.  He also said that the fulness of the everlasting Gospel was contained in it, as delivered by the Savior to the ancient inhabitants;"
\end{quote}
\end{center}

\begin{table}[h!]
\centering
\label{table:js7}
\begin{tabular*}{\textwidth}{l @{\extracolsep{\fill}}cc}
Speaker & Important Characters & Target Audience \\
\hline
\rule{0pt}{3ex}Joseph Smith & --- & The reader 
\end{tabular*}
\end{table}

\subsubsection{Principles and Tags\label{js:principles7}}
\begin{itemize}
\item \index{}\emph{None} (yet)
\end{itemize}

\subsubsection{Comments\label{js:comments7}}
Various scholars of the practices of ancient civilizations have affirmed that such a practice as writing on metallic plates existed in such a time as the Book of Mormon is purported to take place.  As more discoveries come to light, more and more validation is found for the authenticity of the Book of Mormon.

How many histories of various peoples do we not have access to (yet)?  As we read in the Bible, God is no respecter of persons, and we also know that God loves all of His children.  Who is to say that God did not command various peoples across the ages to write of their experiences?  Fortunately, we have at least the records contained in the Bible, and a second witness in the Book of Mormon.  Beyond that, we have the teachings of the modern prophets and apostles in the form of the Doctrine and Covenants.  Messages from General Conference are stored as far back as 1971 on \url{lds.org}, and BYU has documents from messages given since around 1850 at \url{lds-general-conference.org/}.  There is so much to study and learn from these combined resources, and that prospect is exciting!

\subsubsection{Additional References\label{js:references7}}
\begin{itemize}
\item See \href{http://www.deseretnews.com/article/705371752/Ancient-gold-plates-in-Mesoamerica.html}{Ancient Gold Plates in Mesoamerica}
\item See \href{http://www.jefflindsay.com/bme10.shtml}{Book of Mormon Nuggets}
\item See \href{https://www.lds.org/general-conference/conferences?lang=eng}{Conference Talks Since 1971}
\item See \href{http://www.lds-general-conference.org/}{Conference Talks Since 1850}
\item Joseph Smith -- History 1:34
\end{itemize}
%%%%%%%%%%%%%%%%%%%%%%%%%%%%%%%%%%%%%%%%%%%%%%%%%%%%%%%%%%%%%%%%%%%%%%%%%%%%%%%%%%%%%%%%%%%
%%%%%%%%%%%%%%%%%%%%%%%%%%%%%%%%%%%%%%%%%%%%%%%%%%%%%%%%%%%%%%%%%%%%%%%%%%%%%%%%%%%%%%%%%%%

%%%%%%%%%%%%%%%%%%%%%%%%%%%%%%%%%%%%%%%%%%%%%%%%%%%%%%%%%%%%%%%%%%%%%%%%%%%%%%%%%%%%%%%%%%%
%%%%%%%%%%%%%%%%%%%%%%%%%%%%%%%%%%%%%%%%%%%%%%%%%%%%%%%%%%%%%%%%%%%%%%%%%%%%%%%%%%%%%%%%%%%
\subsection{8th Paragraph\label{js:8th}}
\begin{center}
\begin{quote}
``Also that there were two stones in silver bows -- and these stones, fastened to a breastplate, constituted what is called the Urim and Thummim -- deposited with the plates; and the possession and use of these stones were what constituted \emph{Seers} in ancient or former times; and that God had prepared them for the purpose of translating the book."
\end{quote}
\end{center}

\begin{table}[h!]
\centering
\label{table:js8}
\begin{tabular*}{\textwidth}{l @{\extracolsep{\fill}}cc}
Speaker & Important Characters & Target Audience \\
\hline
\rule{0pt}{3ex}Joseph Smith & --- & The reader 
\end{tabular*}
\end{table}

\subsubsection{Definitions\label{js:DFN8}}
\emph{Breastplate}: \begin{itemize}
\item a piece of plate armor partially or completely covering the front of the torso: used by itself or as part of a cuirass
\item the part of the harness that runs across the chest of a saddle horse
\item \emph{Judaism}:\begin{itemize}
	\item a square, richly embroidered vestment ornamented with 12 precious stones, each inscribed with the name of one of the 12 tribes of Israel, secured to the ephod of the high priest and worn on the chest. Ex. 28:15–28
	\item a rectangular ornament, typically of silver, suspended by a chain over the front of a scroll of the Torah
	\end{itemize}
\item a plate opposite the chuck end of a breast drill against which the operator's chest is placed
\end{itemize}
\emph{Constituted}: \begin{itemize}
\item to compose; form
\item to appoint to an office or function; make or create
\item to establish (laws, an institution, etc.)
\item to give legal form to (an assembly, court, etc.)
\item to create or be tantamount to
\item \emph{Archaic}: to set or place
\end{itemize}

\subsubsection{Principles and Tags\label{js:principles8}}
\begin{itemize}
\item \index{}\emph{None} (yet)
\end{itemize}

\subsubsection{Comments\label{js:comments8}}
I've never thought much on what it means to have access to a Urim and Thummim.  The first article below seems to indicate (and various other sources seem to agree) that the Urim and Thummim was a lot-based type of revelation, where sticks or stones, or other similar items were tossed, and the resulting layout determined an answer.  The literal meaning is uncertain, but a good consensus is ``lights and perfections."  From the LDS Bible Dictionary we learn that after the earth has been celestialized, that it will become a Urim and Thummim to its inhabitants.  It seems that the purpose of each Urim and Thummim is decided by the Lord, as Joseph Smith is told here that this particular Urim and Thummim was prepared by the Lord for the purpose of translating the Book of Mormon.  Other uses of the Urim and Thummim in the Old Testament seemed more focused on the `yes or no' answers (according to the article).

\subsubsection{Additional References\label{js:references8}}
\begin{itemize}
\item See \href{https://bible.org/question/how-did-urim-and-thummim-function}{Urim and Thummim Function}
\item See \href{https://www.lds.org/scriptures/bd/urim-and-thummim}{LDS Bible Dictionary: Urim and Thummim}
\item Joseph Smith -- History 1:35
\end{itemize}
%%%%%%%%%%%%%%%%%%%%%%%%%%%%%%%%%%%%%%%%%%%%%%%%%%%%%%%%%%%%%%%%%%%%%%%%%%%%%%%%%%%%%%%%%%%
%%%%%%%%%%%%%%%%%%%%%%%%%%%%%%%%%%%%%%%%%%%%%%%%%%%%%%%%%%%%%%%%%%%%%%%%%%%%%%%%%%%%%%%%%%%

%%%%%%%%%%%%%%%%%%%%%%%%%%%%%%%%%%%%%%%%%%%%%%%%%%%%%%%%%%%%%%%%%%%%%%%%%%%%%%%%%%%%%%%%%%%
%%%%%%%%%%%%%%%%%%%%%%%%%%%%%%%%%%%%%%%%%%%%%%%%%%%%%%%%%%%%%%%%%%%%%%%%%%%%%%%%%%%%%%%%%%%
\subsection{9th Paragraph\label{js:9th}}
\begin{center}
\begin{quote}
``Again, he told me, that when I got those plates of which he had spoken -- for the time that they should be obtained was not yet fulfilled -- I should not show them to any person; neither the breastplate with the Urim and Thummim; only to those to whom I should be commanded to show them; if I did I should be destroyed.  While he was conversing with me about the plates, the vision was opened to my mind that I could see the place where the plates were deposited, and that so clearly and distinctly that I knew the place again when I visited it."
\end{quote}
\end{center}

\begin{table}[h!]
\centering
\label{table:js9}
\begin{tabular*}{\textwidth}{l @{\extracolsep{\fill}}cc}
Speaker & Important Characters & Target Audience \\
\hline
\rule{0pt}{3ex}Joseph Smith & --- & The reader 
\end{tabular*}
\end{table}

\subsubsection{Principles and Tags\label{js:principles9}}
\begin{itemize}
\item \index{If-Then}If-Then
\end{itemize}

\subsubsection{Comments\label{js:comments9}}
I have a testimony that God can certainly plant images in our minds that are very clear.  It is absolutely no surprise to me that Joseph Smith saw the exact location of where the plates were buried, and distinctly enough that he could easily identify it when he went to get the plates.  Something that some people find particularly difficult about the ``gold Bible" that Joseph claimed to have is that so few people actually saw it.  I feel safe in pointing to this and saying that if God didn't want others to see it, He made sure it was so.  Self-preservation is a natural instinct, and I think it's important to point out that there would be more eternal consequences than simply being struck dead if Joseph Smith had shown the plates to those who were not meant to see the plates.  One question I have for those who question the Book of Mormon's authenticity: Why did Joseph Smith wait?  If he knew exactly where the plates were located (and we find out later that he went to the same spot four years in a row), what was the point in waiting for them?  Based on the sources below, I feel I can conclude that there was still much for Joseph to learn in preparation for obtaining a record endorsed by God.  

\subsubsection{Additional References\label{js:references9}}
\begin{itemize}
\item Joseph Smith -- History 1:42
\item See \href{https://www.lds.org/manual/the-pearl-of-great-price-student-manual/joseph-smith-history?lang=eng}{Pearl of Great Price Student Manual}, Joseph Smith's first visit to the Hill Cumorah.
\item See \href{https://www.lds.org/manual/teachings-joseph-smith/chapter-4?lang=eng}{Teachings of the Prophet Joseph Smith -- LDS Manual}
\item See \href{https://www.lds.org/ensign/1992/01/moroni-joseph-smiths-tutor?lang=eng}{Moroni - Joseph Smith's Tutor}
\end{itemize}
%%%%%%%%%%%%%%%%%%%%%%%%%%%%%%%%%%%%%%%%%%%%%%%%%%%%%%%%%%%%%%%%%%%%%%%%%%%%%%%%%%%%%%%%%%%
%%%%%%%%%%%%%%%%%%%%%%%%%%%%%%%%%%%%%%%%%%%%%%%%%%%%%%%%%%%%%%%%%%%%%%%%%%%%%%%%%%%%%%%%%%%

%%%%%%%%%%%%%%%%%%%%%%%%%%%%%%%%%%%%%%%%%%%%%%%%%%%%%%%%%%%%%%%%%%%%%%%%%%%%%%%%%%%%%%%%%%%
%%%%%%%%%%%%%%%%%%%%%%%%%%%%%%%%%%%%%%%%%%%%%%%%%%%%%%%%%%%%%%%%%%%%%%%%%%%%%%%%%%%%%%%%%%%
\subsection{10th Paragraph\label{js:10th}}
\begin{center}
\begin{quote}
``After this communication, I saw the light in the room begin to gather immediately around the person of him who had been speaking to me, and it continued to do so, until the room was again left dark, except just around him, when instantly I saw, as it were, a conduit open right up into heaven, and he ascended until he entirely disappeared, and the room was left as it had been before this heavenly light had made its appearance."
\end{quote}
\end{center}

\begin{table}[h!]
\centering
\label{table:js10}
\begin{tabular*}{\textwidth}{l @{\extracolsep{\fill}}cc}
Speaker & Important Characters & Target Audience \\
\hline
\rule{0pt}{3ex}Joseph Smith & --- & The reader 
\end{tabular*}
\end{table}

\subsubsection{Definitions\label{js:DFN10}}
\emph{Conduit}: \begin{itemize}
\item a pipe, tube, or the like, for conveying water or other fluid
\item a similar natural passage
\item \emph{Electricity}: a structure containing one or more ducts
\item \emph{Archaic}: a fountain
\end{itemize}

\subsubsection{Principles and Tags\label{js:principles10}}
\begin{itemize}
\item \index{}\emph{None} (yet)
\end{itemize}

\subsubsection{Comments\label{js:comments10}}
It's interesting to me the amount of control over light that this heavenly being had.  Perhaps it was not light at all?  Perhaps light was just the best way that Joseph Smith could describe it.  The way that Joseph Smith describes this experience, it seems like Joseph never actually spoke \emph{to} Moroni.  Rather, it seems that Moroni gave a lecture to Joseph Smith, and upon completion of the lecture, left the student to ponder on the message.

\subsubsection{Additional References\label{js:reference10}}
\begin{itemize}
\item Joseph Smith -- History 1:43
\end{itemize}
%%%%%%%%%%%%%%%%%%%%%%%%%%%%%%%%%%%%%%%%%%%%%%%%%%%%%%%%%%%%%%%%%%%%%%%%%%%%%%%%%%%%%%%%%%%
%%%%%%%%%%%%%%%%%%%%%%%%%%%%%%%%%%%%%%%%%%%%%%%%%%%%%%%%%%%%%%%%%%%%%%%%%%%%%%%%%%%%%%%%%%%

%%%%%%%%%%%%%%%%%%%%%%%%%%%%%%%%%%%%%%%%%%%%%%%%%%%%%%%%%%%%%%%%%%%%%%%%%%%%%%%%%%%%%%%%%%%
%%%%%%%%%%%%%%%%%%%%%%%%%%%%%%%%%%%%%%%%%%%%%%%%%%%%%%%%%%%%%%%%%%%%%%%%%%%%%%%%%%%%%%%%%%%
\subsection{11th Paragraph\label{js:11th}}
\begin{center}
\begin{quote}
``I lay musing on the singularity of the scene, and marveling greatly at what had been told to me by the extraordinary messenger; when, in the midst of my meditation, I suddenly discovered that my room was again beginning to get lighted, and in an instant, as it were, the same heavenly messenger was again by my bedside."
\end{quote}
\end{center}

\begin{table}[h!]
\centering
\label{table:js11}
\begin{tabular*}{\textwidth}{l @{\extracolsep{\fill}}cc}
Speaker & Important Characters & Target Audience \\
\hline
\rule{0pt}{3ex}Joseph Smith & --- & The reader 
\end{tabular*}
\end{table}

\subsubsection{Definitions\label{js:DFN11}}
\emph{Musing}: \begin{itemize}
\item absorbed in thought; meditative
\end{itemize}
\emph{Singularity}: \begin{itemize}
\item the state, fact, or quality of being singular
\item a singular, unusual, or unique quality; peculiarity
\item \emph{Mathematics}: singular point
\item \emph{Astronomy}: (in general relativity) the mathematical representation of a black hole
\end{itemize}
\emph{Extraordinary}: \begin{itemize}
\item beyond what is usual, ordinary, regular, or established
\item exceptional in character, amount, extent, degree, etc.; noteworthy; remarkable
\item (of an official, employee, etc.) outside of or additional to the ordinary staff; having a special, often temporary task or responsibility
\end{itemize}
\emph{Midst}: \begin{itemize}
\item the position of anything surrounded by other things or parts, or occurring in the middle of a period of time, course of action, etc. (usually preceded by the)
\item the middle point, part, or stage (usually preceded by the)
\end{itemize}
\emph{As it were}: \begin{itemize}
\item a parenthetic phrase used to indicate that a word or statement is perhaps not formally exact though practically right
\end{itemize}

\subsubsection{Principles and Tags\label{js:principles11}}
\begin{itemize}
\item \index{}\emph{None} (yet)
\end{itemize}

\subsubsection{Comments\label{js:comments11}}
As a side note, this paragraph contains a lot of alliteration.

I believe it is instructive to note that the previous spiritual experience led to another one, \emph{after} Joseph had ``lay musing on the singularity of the scene, and marveling greatly at what had been told" to him.  How often do we ponder on the spiritual messages we receive?  How often do we recognize the marvelous events and principles and doctrines that we hear?  Joseph seems to have learned from a young age that there are things that require a different sort of attention than what the world might give.  The world (it seems) now has a focus on instant judgments.  The world has fallen into the trap of requiring instant gratification.  Joseph demonstrates that instant gratification is not something that will lead to spiritual experiences - he had to put forth effort in pondering and musing before the result appeared before him.  Granted, the way the story is told it seems like it happened very quickly, but we have to remember that Joseph Smith had at least thought about things once or twice throughout the course of three or four years before coming to the Lord for further light.

\subsubsection{Additional References\label{js:reference11}}
\begin{itemize}
\item Joseph Smith -- History 1:44
\end{itemize}
%%%%%%%%%%%%%%%%%%%%%%%%%%%%%%%%%%%%%%%%%%%%%%%%%%%%%%%%%%%%%%%%%%%%%%%%%%%%%%%%%%%%%%%%%%%
%%%%%%%%%%%%%%%%%%%%%%%%%%%%%%%%%%%%%%%%%%%%%%%%%%%%%%%%%%%%%%%%%%%%%%%%%%%%%%%%%%%%%%%%%%%

%%%%%%%%%%%%%%%%%%%%%%%%%%%%%%%%%%%%%%%%%%%%%%%%%%%%%%%%%%%%%%%%%%%%%%%%%%%%%%%%%%%%%%%%%%%
%%%%%%%%%%%%%%%%%%%%%%%%%%%%%%%%%%%%%%%%%%%%%%%%%%%%%%%%%%%%%%%%%%%%%%%%%%%%%%%%%%%%%%%%%%%
\subsection{12th Paragraph\label{js:12th}}
\begin{center}
\begin{quote}
``He commenced, and again related the very same things which he had done at his first visit, without the least variation; which having done, he informed me of great judgments which were coming upon the earth, with great desolations by famine, sword, and pestilence; and that these grievous judgments would come on the earth in this generation.  Having related these things, he again ascended as he had done before."
\end{quote}
\end{center}

\begin{table}[h!]
\centering
\label{table:js12}
\begin{tabular*}{\textwidth}{l @{\extracolsep{\fill}}cc}
Speaker & Important Characters & Target Audience \\
\hline
\rule{0pt}{3ex}Joseph Smith & --- & The reader 
\end{tabular*}
\end{table}

\subsubsection{Definitions\label{js:DFN12}}
\emph{Commence}: \begin{itemize}
\item to begin; start
\end{itemize}
\emph{Desolation}: \begin{itemize}
\item an act or instance of desolating
\item the state of being desolated
\item devastation; ruin
\item depopulation
\item dreariness; barrenness
\item deprivation of companionship; loneliness
\item sorrow; grief; woe
\end{itemize}
\emph{Famine}: \begin{itemize}
\item extreme and general scarcity of food, as in a country or a large geographical area
\item any extreme and general scarcity
\item extreme hunger; starvation
\end{itemize}
\emph{Pestilence}: \begin{itemize}
\item a deadly or virulent epidemic disease
\item bubonic plague
\item something that is considered harmful, destructive, or evil
\end{itemize}
\emph{Grievous}: \begin{itemize}
\item causing grief or great sorrow
\item flagrant; outrageous; atrocious
\item full of or expressing grief; sorrowful
\item burdensome or oppressive
\item causing great pain or suffering
\end{itemize}
\emph{Generation}: \begin{itemize}
\item the entire body of individuals born and living at about the same time
\item the term of years, roughly 30 among human beings, accepted as the average period between the birth of parents and the birth of their offspring
\item a group of individuals, most of whom are the same approximate age, having similar ideas, problems, attitudes, etc
\end{itemize}

\subsubsection{Principles and Tags\label{js:principles12}}
\begin{itemize}
\item \index{War}War
\end{itemize}

\subsubsection{Comments\label{js:comments12}}
I wonder how much detail Joseph Smith received of these ``grievous judgments" that were to come.  I also wonder how far into the future these foretellings were.  Various people have been hung up on the word ``generation," some focusing on the (seeming) modern definition of a few decades, where-as some apologists focus on a longer length of time, citing Biblical references referring to a group of people over the course of an extended length of time.  I think the focus on such a word is a bit narrow-minded.  The point here is that Joseph Smith was made aware of unfortunate and disastrous events that would occur in the (relatively) near future.  Joseph Smith makes it clear here that the details of such events were not important.  The important things that occur are: 1) that the \emph{same} heavenly being appeared to him and 2) gave him the \emph{same} message, which having done so 3) brought additional details of the world he lived in to his attention.  Today, we have much of the same information available to us.  God has commanded the leaders of His church to bring us the same messages year after year, and while doing so will often bring additional details of the world we live in to our attention.  While we may not experience visits from heavenly beings, all that is required is that we have faith, and we will be able have spiritual experiences that we presently do not now enjoy.

\subsubsection{Additional References\label{js:reference12}}
\begin{itemize}
\item Joseph Smith -- History 1:45
\end{itemize}
%%%%%%%%%%%%%%%%%%%%%%%%%%%%%%%%%%%%%%%%%%%%%%%%%%%%%%%%%%%%%%%%%%%%%%%%%%%%%%%%%%%%%%%%%%%
%%%%%%%%%%%%%%%%%%%%%%%%%%%%%%%%%%%%%%%%%%%%%%%%%%%%%%%%%%%%%%%%%%%%%%%%%%%%%%%%%%%%%%%%%%%

%%%%%%%%%%%%%%%%%%%%%%%%%%%%%%%%%%%%%%%%%%%%%%%%%%%%%%%%%%%%%%%%%%%%%%%%%%%%%%%%%%%%%%%%%%%
%%%%%%%%%%%%%%%%%%%%%%%%%%%%%%%%%%%%%%%%%%%%%%%%%%%%%%%%%%%%%%%%%%%%%%%%%%%%%%%%%%%%%%%%%%%
\subsection{13th Paragraph\label{js:13th}}
\begin{center}
\begin{quote}
``By this time, so deep were the impressions made on my mind, that sleep had fled from my eyes, and I lay overwhelmed in astonishment at what I had both seen and heard.  But what was my surprise when again I beheld the same messenger at my bedside, and heard him rehearse or repeat over again to me the same things as before; and added a caution to me, telling me that Satan would try to tempt me (in consequence of the indigent circumstances of my father's family), to get the plates for the purpose of getting rich.  This he forbade me, saying that I must have no other object in view in getting the plates but to glorify God, and must not be influenced by any other motive than that of building His kingdom; otherwise I could not get them."
\end{quote}
\end{center}

\begin{table}[h!]
\centering
\label{table:js13}
\begin{tabular*}{\textwidth}{l @{\extracolsep{\fill}}cc}
Speaker & Important Characters & Target Audience \\
\hline
\rule{0pt}{3ex}Joseph Smith & --- & The reader 
\end{tabular*}
\end{table}

\subsubsection{Definitions\label{js:DFN13}}
\emph{Indigent}: \begin{itemize}
\item lacking food, clothing, and other necessities of life because of poverty; needy; poor; impoverished
\item \emph{Archaic}:\begin{itemize}
  \item deficient in what is requisite
  \item destitute (usually followed by of)
  \end{itemize}
\end{itemize}

\subsubsection{Principles and Tags\label{js:principles13}}
\begin{itemize}
\item \index{Temptation}Temptation
\end{itemize}

\subsubsection{Comments\label{js:comments13}}
Joseph indicates what a singular experience this is.  He is so overwhelmed with what he has experienced, that he cannot sleep.  His mind is probably in overdrive, analyzing and striving to understand (as much as a 17 or 18 year-old boy can) what he has experienced.  And then it happens \emph{again}.  Everything he has heard up to this point is repeated, emphasizing the importance of the message (again). Then, he is given a strong warning.  The way Joseph Smith words his experience, it isn't clear if the parenthetical section is worded by Moroni or himself.  Regardless, it is instructive to look at how the principles behind the temptation here.  We do know from Joseph Smith's later experience (see \Cref{js:20th}) that the temptation succeeded, so what happened?  Joseph or Moroni says that Satan would tempt Joseph Smith \emph{because of the poor circumstances of his family} to use the plates to become rich.  It is very easy to think that this is a good goal - what child wouldn't want to help their family with money issues?  I think especially in those times it was easier to consider this a normal thing, as many children helped their parents earn the wages of the day.  The subtlety here though, is that the motive is completely selfish.  Up to this point in Joseph's life, he has lived in relative comfort.  Certainly there have been challenges, but there does not seem to be any indication that Joseph Smith wanted for anything.  We look at his upbringing now in the 21st century and think that he was an underprivileged individual, but there is not much of a focus on that in Joseph's testimony.  I think this is another testament to the truth of this work.  Joseph Smith could have made this a success story that focused on how little he started with,  and comparing it to the growth it had seen by the time he was murdered.  As it stands, members of the Church of Jesus Christ of Latter-day Saints don't focus on what Joseph did not have.  We rather focus on the things that Joseph Smith \emph{did}.  Joseph Smith prayed in a grove of trees; saw God the Father, and His Son, Jesus Christ; translated the Book of Mormon from plates that had the appearance of gold; and ultimately established Christ's church \emph{back} on the earth, under the direction of Jesus Christ Himself.  But I digress.  The motive is selfish in the sense that it would only bring about the ``glory" of Joseph Smith and (perhaps) his family.  The glory of God and His kingdom would be ignored.  Satan's goal is to undermine all of God's work.  If he had succeeded in convincing Joseph Smith to simply sell the plates, God's work would have been frustrated.  If Satan had succeeded in taking the plates away from Joseph Smith, God's work would have been frustrated.  But God is in the details of our lives, and is working hard to make sure that we can succeed and make it through this probationary state.  Thus He has protected and preserved the sacred record called The Book of Mormon so that we can have it in our hands today.  There is still much in the way of sacred records that we have yet to receive, so we need to prepare ourselves so that we can be worthy of those additional gifts.

\subsubsection{Additional References\label{js:referencces13}}
\begin{itemize}
\item Joseph Smith -- History 1:46
\end{itemize}
%%%%%%%%%%%%%%%%%%%%%%%%%%%%%%%%%%%%%%%%%%%%%%%%%%%%%%%%%%%%%%%%%%%%%%%%%%%%%%%%%%%%%%%%%%%
%%%%%%%%%%%%%%%%%%%%%%%%%%%%%%%%%%%%%%%%%%%%%%%%%%%%%%%%%%%%%%%%%%%%%%%%%%%%%%%%%%%%%%%%%%%

%%%%%%%%%%%%%%%%%%%%%%%%%%%%%%%%%%%%%%%%%%%%%%%%%%%%%%%%%%%%%%%%%%%%%%%%%%%%%%%%%%%%%%%%%%%
%%%%%%%%%%%%%%%%%%%%%%%%%%%%%%%%%%%%%%%%%%%%%%%%%%%%%%%%%%%%%%%%%%%%%%%%%%%%%%%%%%%%%%%%%%%
\subsection{14th Paragraph\label{js:14th}}
\begin{center}
\begin{quote}
``After this third visit, he again ascended into heaven as before, and I was again left to ponder on the strangeness of what I had just experienced; when almost immediately after the heavenly messenger had ascended from me the third time, the cock crowed, and I found that day was approaching, so that our interviews must have occupied the whole of that night."
\end{quote}
\end{center}

\begin{table}[h!]
\centering
\label{table:js14}
\begin{tabular*}{\textwidth}{l @{\extracolsep{\fill}}cc}
Speaker & Important Characters & Target Audience \\
\hline
\rule{0pt}{3ex}Joseph Smith & --- & The reader 
\end{tabular*}
\end{table}

\subsubsection{Principles and Tags\label{js:principles14}}
\begin{itemize}
\item \index{Teaching}Teaching
\end{itemize}

\subsubsection{Comments\label{js:comments14}}
Have you ever stayed up all night to finish a book you just couldn't put down?  Or even to play a favorite video game?  What is the longest lecture you have attended?  Now combine the two, and that seems to be what Joseph Smith has experienced here.  Joseph Smith is not very clear on the details of how much dialogue was had between them, but it seems to me that Moroni is here to act as a lecturer, or one who is responsible for teaching a student.  Good teachers can tell when a student is ready to learn, often able to bring in important ideas that relate to the student in special ways.  Moroni had been sent by the Master Teacher to Joseph Smith \emph{after} Joseph Smith had asked a question.  A question is an excellent indicator of the readiness of the student to learn.  Thus, because of Joseph Smith's inquisitiveness, he was able to learn of the Book of Mormon, and his role in bringing it forth to the world.

Now consider just how tired Joseph Smith must have been by this point.  If you have ever stayed up all night to do something, you know something of what Joseph must have experienced.  Even after a full night of rest, my own eyes struggle to stay open sometimes.  Despite the exhaustion that no doubt was plaguing him, we find out (See \Cref{js:15th}) that he gets up to go to work, stopping only after his concerned father asks him to.

\subsubsection{Additional References\label{js:references14}}
\begin{itemize}
\item Joseph Smith -- History 1:47
\end{itemize}
%%%%%%%%%%%%%%%%%%%%%%%%%%%%%%%%%%%%%%%%%%%%%%%%%%%%%%%%%%%%%%%%%%%%%%%%%%%%%%%%%%%%%%%%%%%
%%%%%%%%%%%%%%%%%%%%%%%%%%%%%%%%%%%%%%%%%%%%%%%%%%%%%%%%%%%%%%%%%%%%%%%%%%%%%%%%%%%%%%%%%%%

%%%%%%%%%%%%%%%%%%%%%%%%%%%%%%%%%%%%%%%%%%%%%%%%%%%%%%%%%%%%%%%%%%%%%%%%%%%%%%%%%%%%%%%%%%%
%%%%%%%%%%%%%%%%%%%%%%%%%%%%%%%%%%%%%%%%%%%%%%%%%%%%%%%%%%%%%%%%%%%%%%%%%%%%%%%%%%%%%%%%%%%
\subsection{15th Paragraph\label{js:15th}}
\begin{center}
\begin{quote}
``I shortly after arose from my bed, and, as usual, went to the necessary labors of the day; but, in attempting to work as at other times, I found my strength so exhausted as to render me entirely unable.  My father, who was laboring along with me, discovered something to be wrong with me, and told me to go home.  I started with the intention of going to the house; but, in attempting to cross the fence out of the filed where we were, my strength entirely failed me, and I fell helpless on the ground, and for a time was quite unconscious of anything."
\end{quote}
\end{center}

\begin{table}[h!]
\centering
\label{table:js15}
\begin{tabular*}{\textwidth}{l @{\extracolsep{\fill}}cc}
Speaker & Important Characters & Target Audience \\
\hline
\rule{0pt}{3ex}Joseph Smith & --- & The reader 
\end{tabular*}
\end{table}

\subsubsection{Principles and Tags\label{js:principles15}}
\begin{itemize}
\item \index{}\emph{None} (yet)
\end{itemize}

\subsubsection{Comments\label{js:comments15}}
I think there are two forces at work here that contribute to Joseph's exhaustion.  The obvious one is the physical exhaustion from having no sleep.  One night of no sleep (according to \href{https://www.sciencealert.com/here-s-what-happens-to-your-body-when-you-stay-up-all-night}{this study}) can cause a host of problems, exhaustion being one of them.  As most people have at least had a relatively sleepless night, it seems easy to relate to the exhaustion here.  The other contributing factor (I believe) to Joseph's exhaustion is from the spiritual exertion.  In a sense, Joseph Smith had a very intense spiritual workout throughout the course of the night.  We read of Moses's experience after being in the presence of God in Moses 1, where he states that ``as [Moses] was left unto himself, he fell unto to the earth.  And it came to pass that it was for the space of many hours before Moses did again receive his natural strength like unto man; and he said unto himself: Now, for this cause I know that man is nothing, which thing I never had supposed."  Spiritual stamina comes in much the same way physical stamina does - continued effort and work to develop it.  The difficult is that as we live in a mortal world, it becomes easy to focus purely on what we can physically experience, making it difficult for us to develop the necessary spiritual stamina necessary to experience such things on a regular basis.

\subsubsection{Additional References\label{js:references15}}
\begin{itemize}
\item See \href{https://www.sciencealert.com/here-s-what-happens-to-your-body-when-you-stay-up-all-night}{Effect of an All-nighter.}
\item Moses 1:9-10
\item Joseph Smith -- History 1:48
\end{itemize}
%%%%%%%%%%%%%%%%%%%%%%%%%%%%%%%%%%%%%%%%%%%%%%%%%%%%%%%%%%%%%%%%%%%%%%%%%%%%%%%%%%%%%%%%%%%
%%%%%%%%%%%%%%%%%%%%%%%%%%%%%%%%%%%%%%%%%%%%%%%%%%%%%%%%%%%%%%%%%%%%%%%%%%%%%%%%%%%%%%%%%%%

%%%%%%%%%%%%%%%%%%%%%%%%%%%%%%%%%%%%%%%%%%%%%%%%%%%%%%%%%%%%%%%%%%%%%%%%%%%%%%%%%%%%%%%%%%%
%%%%%%%%%%%%%%%%%%%%%%%%%%%%%%%%%%%%%%%%%%%%%%%%%%%%%%%%%%%%%%%%%%%%%%%%%%%%%%%%%%%%%%%%%%%
\subsection{16th Paragraph\label{js:16th}}
\begin{center}
\begin{quote}
``The first thing that I can recollect was a voice speaking unto me, calling me by name.  I looked up, and beheld the same messenger standing over my head, surrounded by light as before.  He then again related unto me all that he had related to me the previous night, and commanded me to go to my father and tell him of the vision and commandments which I had received."
\end{quote}
\end{center}

\begin{table}[h!]
\centering
\label{table:js16}
\begin{tabular*}{\textwidth}{l @{\extracolsep{\fill}}cc}
Speaker & Important Characters & Target Audience \\
\hline
\rule{0pt}{3ex}Joseph Smith & --- & The reader 
\end{tabular*}
\end{table}

\subsubsection{Principles and Tags\label{js:principles16}}
\begin{itemize}
\item \index{Patriarchal Order}Patriarchal Order
\end{itemize}

\subsubsection{Comments\label{js:comments16}}
It's important to me that Joseph was commanded to go to his father and let him know what was going on.  As Joseph's father, I probably would be concerned.  He had just sent him home because he had noticed that something was wrong.  I don't know if I would be relieved, or more concerned still when he came back (who knows, a couple of hours later?) and told me that he had fallen and slept for a bit before being woken up by an angelic being (one who had kept him up all night to begin with), and told him the rather unbelievable story.  If he didn't know Joseph's character as well as he did, would the reaction have been different?

\subsubsection{Additional References\label{js:references16}}
\begin{itemize}
\item Joseph Smith -- History 1:49
\end{itemize}
%%%%%%%%%%%%%%%%%%%%%%%%%%%%%%%%%%%%%%%%%%%%%%%%%%%%%%%%%%%%%%%%%%%%%%%%%%%%%%%%%%%%%%%%%%%
%%%%%%%%%%%%%%%%%%%%%%%%%%%%%%%%%%%%%%%%%%%%%%%%%%%%%%%%%%%%%%%%%%%%%%%%%%%%%%%%%%%%%%%%%%%

%%%%%%%%%%%%%%%%%%%%%%%%%%%%%%%%%%%%%%%%%%%%%%%%%%%%%%%%%%%%%%%%%%%%%%%%%%%%%%%%%%%%%%%%%%%
%%%%%%%%%%%%%%%%%%%%%%%%%%%%%%%%%%%%%%%%%%%%%%%%%%%%%%%%%%%%%%%%%%%%%%%%%%%%%%%%%%%%%%%%%%%
\subsection{17th Paragraph\label{js:17th}}
\begin{center}
\begin{quote}
``I obeyed; I returned to my father in the field, and rehearsed the whole matter to him.  He replied to me that it was of God, and told me to go and do as commanded by the messenger.  I left the field, and went to the place where the messenger had told me the plates were deposited; and owing to the distinctness of the vision which I had had concerning it, I knew the place the instant that I arrived there."
\end{quote}
\end{center}

\begin{table}[h!]
\centering
\label{table:js17}
\begin{tabular*}{\textwidth}{l @{\extracolsep{\fill}}cc}
Speaker & Important Characters & Target Audience \\
\hline
\rule{0pt}{3ex}Joseph Smith & --- & The reader 
\end{tabular*}
\end{table}

\subsubsection{Definitions\label{js:DFN17}}
\emph{Rehearse}: \begin{itemize}
\item to practice (a musical composition, a play, a speech, etc.) in private prior to a public presentation
\item to drill or train (an actor, musician, etc.) by rehearsal, as for some performance or part
\item to relate the facts or particulars of; recount
\end{itemize}
\emph{Distinct}: \begin{itemize}
\item distinguished as not being the same; not identical; separate (sometimes followed by from)
\item different in nature or quality; dissimilar (sometimes followed by from)
\item clear to the senses or intellect; plain; unmistakable
\item distinguishing or perceiving clearly
\item unquestionably exceptional or notable
\item \emph{Archaic}: distinctively decorated or adorned
\end{itemize}

\subsubsection{Principles and Tags\label{js:principles17}}
\begin{itemize}
\item \index{Obedience}Obedience
\end{itemize}

\subsubsection{Comments\label{js:comments17}}
Immediately Joseph Smith went and did as he was told.  This is reminiscent of Nephi, who ``[went] and [did] as the Lord commanded."  The relative lack of detail on the actual conversation is interesting to me.  I am curious to know the depth of the conversation the father and son had.  

\subsubsection{Additional References\label{js:references17}}
\begin{itemize}
\item Joseph Smith -- History 1:50
\item 1 Nephi 3:7
\end{itemize}
%%%%%%%%%%%%%%%%%%%%%%%%%%%%%%%%%%%%%%%%%%%%%%%%%%%%%%%%%%%%%%%%%%%%%%%%%%%%%%%%%%%%%%%%%%%
%%%%%%%%%%%%%%%%%%%%%%%%%%%%%%%%%%%%%%%%%%%%%%%%%%%%%%%%%%%%%%%%%%%%%%%%%%%%%%%%%%%%%%%%%%%

%%%%%%%%%%%%%%%%%%%%%%%%%%%%%%%%%%%%%%%%%%%%%%%%%%%%%%%%%%%%%%%%%%%%%%%%%%%%%%%%%%%%%%%%%%%
%%%%%%%%%%%%%%%%%%%%%%%%%%%%%%%%%%%%%%%%%%%%%%%%%%%%%%%%%%%%%%%%%%%%%%%%%%%%%%%%%%%%%%%%%%%
\subsection{18th Paragraph\label{js:18th}}
\begin{center}
\begin{quote}
``Convenient to the village of Manchester, Ontario county, New York, stands a hill of considerable size, and the most elevated of any in the neighborhood.  On the west side of this hill, not far from the top, under a stone of considerable size, lay the plates, deposited in a stone box.  This stone was thick and rounding in the middle on the upper side, and thinner towards the edges, so that the middle part of it was visible above the ground, but the edge all around was covered with earth."
\end{quote}
\end{center}

\begin{table}[h!]
\centering
\label{table:js18}
\begin{tabular*}{\textwidth}{l @{\extracolsep{\fill}}cc}
Speaker & Important Characters & Target Audience \\
\hline
\rule{0pt}{3ex}Joseph Smith & --- & The reader 
\end{tabular*}
\end{table}

\subsubsection{Definitions\label{js:DFN18}}
\emph{Convenient}: \begin{itemize}
\item suitable or agreeable to the needs or purpose; well-suited with respect to facility or ease in use; favorable, easy, or comfortable for use.
\item at hand; easily accessible:
\item \emph{Obsolete}: fitting; suitable.
\end{itemize}
\emph{Considerable}: \begin{itemize}
\item rather large or great in size, distance, extent, etc.
\item worthy of respect, attention, etc.; important; distinguished
\end{itemize}

\subsubsection{Principles and Tags\label{js:principles18}}
\begin{itemize}
\item \index{}\emph{None} (yet)
\end{itemize}

\subsubsection{Comments\label{js:comments18}}
I wonder what the size of the stone and stone box were.  They were at least big enough to contain a breastplate, a stack of gold (colored) plates, and the Urim and Thummim.  We know (from \Cref{js:19th}) that it took 17-year-old Joseph using a lever to remove the stone cover, indicating that the box was at least a little heavy (consider that Joseph worked on a farm, and was probably in pretty good physical shape).  There are many differing opinions on the actual location of the plates, with some people identifying a lack of physical evidence (see \href{http://www.mormonthink.com/glossary/stone-box.htm}{Stone Box}) as ``proof" that the Book of Mormon is not a real record.  My personal thoughts on the matter are, why would God (who doesn't seem to have given us any \emph{physical} evidence of His existence, but has rather focused on \emph{spiritual} evidence) give \emph{physical} evidence of the Book of Mormon, when the \emph{spiritual} evidences given \emph{by} the Book of Mormon can have a more lasting effect?  It is my opinion that the evidences that the world offers (i.e. physical evidence) could change over time.  Spiritual truths do not change however.  So why rely on something in the world that could change, when you could rely on something that will never change? For God is unchangeable (see Mormon 9:9).

\subsubsection{Additional References\label{js:references18}}
\begin{itemize}
\item Joseph Smith -- History 1:51
\item See \href{https://www.fairmormon.org/answers/Book_of_Mormon/Geography/New_World/Hill_Cumorah}{Book of Mormon Geography -- The Hill Cumorah}
\item See \href{http://www.mormonthink.com/glossary/stone-box.htm}{Stone Box}
\item Mormon 9:9
\end{itemize}
%%%%%%%%%%%%%%%%%%%%%%%%%%%%%%%%%%%%%%%%%%%%%%%%%%%%%%%%%%%%%%%%%%%%%%%%%%%%%%%%%%%%%%%%%%%
%%%%%%%%%%%%%%%%%%%%%%%%%%%%%%%%%%%%%%%%%%%%%%%%%%%%%%%%%%%%%%%%%%%%%%%%%%%%%%%%%%%%%%%%%%%

%%%%%%%%%%%%%%%%%%%%%%%%%%%%%%%%%%%%%%%%%%%%%%%%%%%%%%%%%%%%%%%%%%%%%%%%%%%%%%%%%%%%%%%%%%%
%%%%%%%%%%%%%%%%%%%%%%%%%%%%%%%%%%%%%%%%%%%%%%%%%%%%%%%%%%%%%%%%%%%%%%%%%%%%%%%%%%%%%%%%%%%
\subsection{19th Paragraph\label{js:19th}}
\begin{center}
\begin{quote}
``Having removed the earth, I obtained a lever, which I got fixed under the edge of the stone, and with a little exertion raised it up.  I looked in, and there indeed did I behold the plates, the Urim and Thummim, and the breastplate, as stated by the messenger.  The box in which they lay was formed by laying stones together in some kind of cement.  In the bottom of the box were laid two stones crossways of the box, and on these stones lay the plates and the other things with them."
\end{quote}
\end{center}

\begin{table}[h!]
\centering
\label{table:js19}
\begin{tabular*}{\textwidth}{l @{\extracolsep{\fill}}cc}
Speaker & Important Characters & Target Audience \\
\hline
\rule{0pt}{3ex}Joseph Smith & --- & The reader 
\end{tabular*}
\end{table}

\subsubsection{Definitions\label{js:DFN19}}
\emph{Crossways}: \begin{itemize}
\item crosswise (across; transversely)
\end{itemize}

\subsubsection{Principles and Tags\label{js:principles19}}
\begin{itemize}
\item \index{}\emph{None} (yet)
\end{itemize}

\subsubsection{Comments\label{js:comments19}}
The description of the box containing the sacred articles is interesting in the sense that it's so simple.  Another interesting thing is that the stone box had not been found previously.  I guess it makes sense: people generally don't overturn large rocks just to see if there is a box underneath.  The Wikipedia article on stone box graves doesn't clearly indicate the earliest findings of stone boxes in America, but if the dates that are shown are a reference of sorts, it wasn't until the second half of the 19th century that such discoveries were made, giving credence to the idea that Joseph wasn't copying the ideas of some other work.  That's not really the point though.  The ultimate question that needs to be asked, is ``Would God tell me if Joseph Smith was a liar or a fraud?"  The unequivocal answer to that is YES.  God does not want His children to be deceived, so if there is any question about anything that was done, simply ask God.  He will reveal the truth of things to the mind and heart if the individual seeking such knowledge is patient and faithful.

\subsubsection{Additional References\label{js:references19}}
\begin{itemize}
\item Joseph Smith -- History 1:52
\item See \href{https://en.wikipedia.org/wiki/Stone_box_grave}{Stone Box Graves}
\end{itemize}
%%%%%%%%%%%%%%%%%%%%%%%%%%%%%%%%%%%%%%%%%%%%%%%%%%%%%%%%%%%%%%%%%%%%%%%%%%%%%%%%%%%%%%%%%%%
%%%%%%%%%%%%%%%%%%%%%%%%%%%%%%%%%%%%%%%%%%%%%%%%%%%%%%%%%%%%%%%%%%%%%%%%%%%%%%%%%%%%%%%%%%%

%%%%%%%%%%%%%%%%%%%%%%%%%%%%%%%%%%%%%%%%%%%%%%%%%%%%%%%%%%%%%%%%%%%%%%%%%%%%%%%%%%%%%%%%%%%
%%%%%%%%%%%%%%%%%%%%%%%%%%%%%%%%%%%%%%%%%%%%%%%%%%%%%%%%%%%%%%%%%%%%%%%%%%%%%%%%%%%%%%%%%%%
\subsection{20th Paragraph\label{js:20th}}
\begin{center}
\begin{quote}
``I made an attempt to take them out, but was forbidden by the messenger, and was again informed that the time for bringing them forth had not yet arrived, neither would it, until four years from that time; but he told me that I should come to that place precisely in one year from that time, and that he would there meet with me, and that I should continue to do so until the time should come for obtaining the plates."
\end{quote}
\end{center}

\begin{table}[h!]
\centering
\label{table:js20}
\begin{tabular*}{\textwidth}{l @{\extracolsep{\fill}}cc}
Speaker & Important Characters & Target Audience \\
\hline
\rule{0pt}{3ex}Joseph Smith & --- & The reader 
\end{tabular*}
\end{table}

\subsubsection{Principles and Tags\label{js:principles20}}
\begin{itemize}
\item \index{Patience}Patience)
\end{itemize}

\subsubsection{Comments\label{js:comments20}}
I will admit that I am somewhat envious of the education that Joseph Smith must have received at the hands of Moroni (and possibly other ancient prophets). I do enjoy learning, and here Joseph Smith is promised that he will get further education on important matters for the next four years.  In a sense, this is Joseph Smith's ``university" experience.  Joseph Smith later told the apostles, ``if I were to tell you all I know of the kingdom of God, I do know that you would rise up and kill me."  Joseph Smith knew of things that I cannot even begin to fathom.  This both excites me, but also makes me feel very inadequate.  There is so much that I want to learn, and so much that I can learn, but there is so little time to actually learn it all.

\subsubsection{Additional References\label{js:references20}}
\begin{itemize}
\item Joseph Smith -- History 1:53
\item See \href{http://emp.byui.edu/satterfieldb/quotes/If%20I%20were%20to%20tell%20you%20all%20I%20know%20%20JosephSmith.html}{Joseph Smith - All I Know}
\end{itemize}
%%%%%%%%%%%%%%%%%%%%%%%%%%%%%%%%%%%%%%%%%%%%%%%%%%%%%%%%%%%%%%%%%%%%%%%%%%%%%%%%%%%%%%%%%%%
%%%%%%%%%%%%%%%%%%%%%%%%%%%%%%%%%%%%%%%%%%%%%%%%%%%%%%%%%%%%%%%%%%%%%%%%%%%%%%%%%%%%%%%%%%%

%%%%%%%%%%%%%%%%%%%%%%%%%%%%%%%%%%%%%%%%%%%%%%%%%%%%%%%%%%%%%%%%%%%%%%%%%%%%%%%%%%%%%%%%%%%
%%%%%%%%%%%%%%%%%%%%%%%%%%%%%%%%%%%%%%%%%%%%%%%%%%%%%%%%%%%%%%%%%%%%%%%%%%%%%%%%%%%%%%%%%%%
\subsection{21st Paragraph\label{js:21st}}
\begin{center}
\begin{quote}
``Accordingly, as I had been commanded, I went at the end of each year, and at each time I found the same messenger there, and received instruction and intelligence from him at each of our interviews, respecting what the Lord was going to do, and how and in what manner His kingdom was to be conducted in the last days."
\end{quote}
\end{center}

\begin{table}[h!]
\centering
\label{table:js21}
\begin{tabular*}{\textwidth}{l @{\extracolsep{\fill}}cc}
Speaker & Important Characters & Target Audience \\
\hline
\rule{0pt}{3ex}Joseph Smith & --- & The reader 
\end{tabular*}
\end{table}

\subsubsection{Principles and Tags\label{js:principles21}}
\begin{itemize}
\item \index{Intelligence}Intelligence
\item \index{Church Government}Church Government
\end{itemize}

\subsubsection{Comments\label{js:comments21}}
I find Joseph's use of the word ``interview" interesting.  An interview is generally conducted, at least in the modern sense, of someone who is looking for a specific kind of person to fill a role (such as a job interview).  The fact that Joseph called these meetings ``interviews" reveals that Joseph knew something of the role he needed to fill.  I wonder just how much of the information Joseph learned related to administrative tasks?

\subsubsection{Additional References\label{js:references21}}
\begin{itemize}
\item Joseph Smith -- History 1:54
\end{itemize}
%%%%%%%%%%%%%%%%%%%%%%%%%%%%%%%%%%%%%%%%%%%%%%%%%%%%%%%%%%%%%%%%%%%%%%%%%%%%%%%%%%%%%%%%%%%
%%%%%%%%%%%%%%%%%%%%%%%%%%%%%%%%%%%%%%%%%%%%%%%%%%%%%%%%%%%%%%%%%%%%%%%%%%%%%%%%%%%%%%%%%%%

%%%%%%%%%%%%%%%%%%%%%%%%%%%%%%%%%%%%%%%%%%%%%%%%%%%%%%%%%%%%%%%%%%%%%%%%%%%%%%%%%%%%%%%%%%%
%%%%%%%%%%%%%%%%%%%%%%%%%%%%%%%%%%%%%%%%%%%%%%%%%%%%%%%%%%%%%%%%%%%%%%%%%%%%%%%%%%%%%%%%%%%
\subsection{22nd Paragraph\label{js:22nd}}
\begin{center}
\begin{quote}
``At length the time arrived for obtaining the plates, the Urim and Thummim, and the breastplate.  On the twenty-second day of September, one thousand eight hundred and twenty-seven, having gone as usual at the end of another year to the place where they were deposited, the same heavenly messenger delivered them up to me with this charge: That I should be responsible for them; that if I should let them go carelessly, or through any neglect of mine, I should be cut off; but that if I would use all my endeavors to preserve them, until he, the messenger, should call for them, they should be protected."
\end{quote}
\end{center}

\begin{table}[h!]
\centering
\label{table:js22}
\begin{tabular*}{\textwidth}{l @{\extracolsep{\fill}}cc}
Speaker & Important Characters & Target Audience \\
\hline
\rule{0pt}{3ex}Joseph Smith & --- & The reader 
\end{tabular*}
\end{table}

\subsubsection{Definitions\label{js:DFN22}}
\emph{Charge}: \begin{itemize}
\item to impose or ask as a price or fee
\item to impose on or ask of (someone) a price or fee
\item to defer payment for (a purchase) until a bill is rendered by the creditor
\item to hold liable for payment; enter a debit against
\item to attack by rushing violently against
\item to accuse formally or explicitly (usually followed by with)
\item to impute; ascribe the responsibility for
\end{itemize}
\emph{Neglect}: \begin{itemize}
\item to pay no attention or too little attention to; disregard or slight
\item to be remiss in the care or treatment of
\item to omit, through indifference or carelessness
\item to fail to carry out or perform (orders, duties, etc.)
\item to fail to take or use
\item an act or instance of neglecting; disregard; negligence
\item the fact or state of being neglected
\end{itemize}
\emph{Endeavor}: \begin{itemize}
\item to exert oneself to do or effect something; make an effort; strive
\item to attempt; try
\item \emph{Archaic}: to attempt to achieve or gain
\item a strenuous effort; attempt
\end{itemize}

\subsubsection{Principles and Tags\label{js:principles22}}
\begin{itemize}
\item \index{Accountability}Accountability
\item \index{God!Protection of}{Protection of God}
\end{itemize}

\subsubsection{Comments\label{js:comments22}}
I wonder what state of mind Joseph was in as he received the plates.  This was something he had been anticipating for four years now.  The preparation seems to have worked in the sense that Joseph was no longer tempted to use the plates to make himself and his family rich.  The emotion that I imagine him having is a sort of reverent awe as he received the plates.  For myself, that awe would quickly turn to concern at what I might be able to do to keep the plates hidden from those who would attempt to take them.  Having undertaken four years of heavenly teaching, I would assume that Joseph Smith had a deeper understanding of the spiritual implications of the Book of Mormon.  This would drive me (if I were in his position) to do everything I could to protect the plates, but also to hurry and get the message out to the world.  That is exactly what Joseph Smith did!

As an aside, some of the websites I have come across have had difficulties with the apparent lack of evidence of the existence of the stone box, citing failed efforts of contemporaries of Joseph Smith at finding at least a hole in the ground.  My thoughts on the matter are that if God could protect and preserve the location of the plates for nearly 1500 years, who is to say that God cannot provide the same protection to evidences of the resting place of the Book of Mormon?  Some apologists have claimed that the actual location which Joseph received the plates was actually in South America (see \Cref{js:references22}), and that the actual travel there was done spiritually.  Whatever the case may be, the important thing is that Joseph Smith received the plates from a heavenly messenger, and through his own efforts (with the added strength of God to help), he was able to both protect and translate the plates.

\subsubsection{Additional References\label{js:references22}}
\begin{itemize}
\item Joseph Smith -- History 1:59
\item See \href{https://www.fairmormon.org/answers/Book_of_Mormon/Geography/New_World/Hill_Cumorah}{Book of Mormon Geography -- The Hill Cumorah}
\end{itemize}
%%%%%%%%%%%%%%%%%%%%%%%%%%%%%%%%%%%%%%%%%%%%%%%%%%%%%%%%%%%%%%%%%%%%%%%%%%%%%%%%%%%%%%%%%%%
%%%%%%%%%%%%%%%%%%%%%%%%%%%%%%%%%%%%%%%%%%%%%%%%%%%%%%%%%%%%%%%%%%%%%%%%%%%%%%%%%%%%%%%%%%%

%%%%%%%%%%%%%%%%%%%%%%%%%%%%%%%%%%%%%%%%%%%%%%%%%%%%%%%%%%%%%%%%%%%%%%%%%%%%%%%%%%%%%%%%%%%
%%%%%%%%%%%%%%%%%%%%%%%%%%%%%%%%%%%%%%%%%%%%%%%%%%%%%%%%%%%%%%%%%%%%%%%%%%%%%%%%%%%%%%%%%%%
\subsection{23rd Paragraph\label{js:23rd}}
\begin{center}
\begin{quote}
``I soon found out the reason why I had received such strict charges to keep them safe, and why it was that the messenger had said that when I had done what was required at my hand, he would call for them.  For no sooner was it known that I had them, than the most strenuous exertions were used to get them from me.  Every stratagem that could be invented was resorted to for that purpose.  The persecution became more bitter and severe than before, and multitudes were on the alert continually to get them from me if possible.  But by the wisdom of God, they remained safe in my hands, until I had accomplished by them what was required at my hand.  When, according to arrangements, the messenger called for them, I delivered them up to him; and he has them in his charge until this day, being the second day of May, one thousand eight hundred and thirty-eight."
\end{quote}
\end{center}

\begin{table}[h!]
\centering
\label{table:js23}
\begin{tabular*}{\textwidth}{l @{\extracolsep{\fill}}cc}
Speaker & Important Characters & Target Audience \\
\hline
\rule{0pt}{3ex}Joseph Smith & --- & The reader 
\end{tabular*}
\end{table}

\subsubsection{Definitions\label{js:DFN23}}
\emph{Strenuous}: \begin{itemize}
\item characterized by vigorous exertion, as action, efforts, life, etc.
\item demanding or requiring vigorous exertion; laborious
\item vigorous, energetic, or zealously active
\end{itemize}
\emph{Exertion}: \begin{itemize}
\item vigorous action or effort
\item an effort
\item exercise, as of power or faculties
\item an instance of this
\end{itemize}
\emph{Stratagem}: \begin{itemize}
\item a plan, scheme, or trick for surprising or deceiving an enemy
\item any artifice, ruse, or trick devised or used to attain a goal or to gain an advantage over an adversary or competitor
\end{itemize}
\emph{Resort}: \begin{itemize}
\item to have recourse for use, help, or accomplishing something, often as a final available option or resource
\item to go, especially frequently or customarily
\end{itemize}

\subsubsection{Principles and Tags\label{js:principles23}}
\begin{itemize}
\item \index{God!Wisdom of}Wisdom of God
\end{itemize}

\subsubsection{Comments\label{js:comments23}}
I wonder how other people came to know that Joseph Smith was getting the plates.  I suppose it's plausible that Joseph himself may have told others outside of his family, and those ``rumors" spread simply because of how unbelievable they were. Regardless of how they knew, their reasons for pursuing the plates was anything but for the glory of God.  If the persecution and violence, etc. intensified, that doesn't seem like a group of people seeking the development of God's kingdom.

\subsubsection{Additional References\label{js:references23}}
\begin{itemize}
\item Joseph Smith -- History 1:60
\end{itemize}
%%%%%%%%%%%%%%%%%%%%%%%%%%%%%%%%%%%%%%%%%%%%%%%%%%%%%%%%%%%%%%%%%%%%%%%%%%%%%%%%%%%%%%%%%%%
%%%%%%%%%%%%%%%%%%%%%%%%%%%%%%%%%%%%%%%%%%%%%%%%%%%%%%%%%%%%%%%%%%%%%%%%%%%%%%%%%%%%%%%%%%%/

%%%%%%%%%%%%%%%%%%%%%%%%%%%%%%%%%%%%%%%%%%%%%%%%%%%%%%%%%%%%%%%%%%%%%%%%%%%%%%%%%%%%%%%%%%%
%%%%%%%%%%%%%%%%%%%%%%%%%%%%%%%%%%%%%%%%%%%%%%%%%%%%%%%%%%%%%%%%%%%%%%%%%%%%%%%%%%%%%%%%%%%
\subsection{24th and 25th Paragraphs\label{js:final}}
\begin{center}
\begin{quote}
``For the complete record, see Joseph Smith -- History, in the Pearl of Great Price, and \emph{History of The Church of Jesus Christ of Latter-day Saints}, volume 1, chapters 1 through 6.

``The ancient thus brought forth from the earth as the voice of a people speaking from the dust, and translated into modern speech by the gift and power of God as attested by Divine affirmation, was first published to the world in the year 1830 as \scriptsize THE BOOK OF MORMON\normalsize."
\end{quote}
\end{center}

\begin{table}[h!]
\centering
\label{table:js_final}
\begin{tabular*}{\textwidth}{l @{\extracolsep{\fill}}cc}
Speaker & Important Characters & Target Audience \\
\hline
\rule{0pt}{3ex}--- & --- & The reader 
\end{tabular*}
\end{table}

\subsubsection{Definitions\label{js:DFNFinal}}
\emph{Affirmation}: \begin{itemize}
\item the act or an instance of affirming; state of being affirmed
\item the assertion that something exists or is true
\item something that is affirmed; a statement or proposition that is declared to be true
\item confirmation or ratification of the truth or validity of a prior judgment, decision, etc
\item \emph{Law}: a solemn declaration accepted instead of a statement under oath
\end{itemize}

\subsubsection{Principles and Tags\label{js:principlesFinal}}
\begin{itemize}
\item \index{}\emph{None} (yet)
\end{itemize}

\subsubsection{Comments\label{js:commentsFinal}}
This book truly is attested to through Divine affirmation, but not just in the general sense.  Yes, we have records in the Church that God has specifically testified to the truthfulness of the Book of Mormon, but that testimony is not just for those who read that particular document.  \emph{Anyone} who sincerely reads and studies the Book of Mormon and the truths it contains, and then asks God if said truths are in fact from Him, will receive the same Divine affirmation.  God is no respecter of persons, and will therefore give \emph{everyone} the same affirmation if they will but seek it!

\subsubsection{Additional References\label{js:referencesFinal}}
\begin{itemize}
\item \emph{None} (yet)
\end{itemize}
%%%%%%%%%%%%%%%%%%%%%%%%%%%%%%%%%%%%%%%%%%%%%%%%%%%%%%%%%%%%%%%%%%%%%%%%%%%%%%%%%%%%%%%%%%%
%%%%%%%%%%%%%%%%%%%%%%%%%%%%%%%%%%%%%%%%%%%%%%%%%%%%%%%%%%%%%%%%%%%%%%%%%%%%%%%%%%%%%%%%%%%

\section{General Comments on \Cref{chapter:testimonies}}
With these testimonies of the Book of Mormon I add my own testimony.  I know for a certainty that the Book of Mormon is the Word of God.  I have read it.  I have pondered it.  I have studied it, and continue to do so.  I have asked God about it's truth, and I have felt His confirming witness that it is true.  I invite anyone who is seeking for more in their lives, whether due to sorrow, or pain, or simply feeling a lack, to try the Book of Mormon and the Church of Jesus Christ of Latter-day Saints.  I promise that you can and will find peace of mind and heart as you open up the lines of communication with our Heavenly Father, and that you will find that your whole soul is filled with peace and joy.

\part{The First Book of Nephi\label{BoM:1Nephi}}
\section{Introduction\label{1Nephi:intro}}
\begin{center}
\begin{quote}
``An account of Lehi and his wife Sariah, and his four sons, being called, (beginning at the eldest) Laman, Lemuel, Sam, and Nephi.  The Lord warns Lehi to depart out of the land of Jerusalem, because he prophesieth unto the people concerning their iniquity and they seek to destroy his life.  He taketh three days' journey into the wilderness with his family.  Nephi taketh his brethren and returneth to the land of Jerusalem after the record of the Jews.  The account of their sufferings.  They take the daughters of Ishmael to wife.  They take their families and depart into the wilderness.  Their sufferings and afflictions in the wilderness.  The course of their travels.  They come to the large waters.  Nephi's brethren rebel against him.  He confoundeth them, and buildeth a ship.  They call the name of the place Bountiful.  They cross the large waters into the promised land, and so forth.  This is according to the account of Nephi; or in other words, I, Nephi, wrote this record."
\end{quote}
\end{center}

\begin{table}[h!]
\centering
\label{table:1stNephiIntro}
\begin{tabular*}{\textwidth}{l @{\extracolsep{\fill}}cc}
Speaker & Important Characters & Target Audience \\
\hline
\rule{0pt}{3ex}\multirow{6}{*}{Nephi} & Lehi & \multirow{6}{*}{The reader} \\
 & Sariah & \\
 & Laman & \\
 & Lemuel & \\
 & Sam & \\
 & Nephi &
\end{tabular*}
\end{table}

\subsection{Definitions\label{1Nephi:DFN_intro}}
\emph{Use of --eth}: \begin{itemize}
\item Words such as "prophesieth" are an old English style of the modernized word "prophesies."  If a word is confusing, and ends in the letters ``--eth," try replacing those letters with an --s (or --es).
\end{itemize}
\emph{Iniquity}: \begin{itemize}
\item gross injustice or wickedness
\item a violation of right or duty; wicked act; sin
\end{itemize}
\emph{Prophesy}: \begin{itemize}
\item to foretell or predict
\item to indicate beforehand
\item to declare or foretell by or as if by divine inspiration
\item to utter in prophecy or as a prophet
verb (used without object), prophesied, prophesying
\item to make predictions
\item to make inspired declarations of what is to come
\item to speak as a mediator between God and humankind or in God's stead
\item \emph{Archaic}: to teach religious subjects
\end{itemize}
\emph{Bountiful}: \begin{itemize}
\item liberal in bestowing gifts, favors, or bounties; munificent; generous
\item abundant; ample
\end{itemize}

\subsection{Principles and Tags\label{1Nephi:principles_intro}}
\begin{itemize}
\item \index{Record Keeping}Record Keeping)
\end{itemize}

\subsection{Comments\label{1Nephi:comments_intro}}
If this was simply one journal entry, that would be a whirlwind of events.  As it stands, we know that the events mentioned here took place over the course of approximately 20 to 30 years.  An important note about this introductory paragraph is that it too is a translation (citation needed).  We can expect a lot of things to happen just from this brief introduction.  

\subsection{Additional References\label{1Nephi:references_intro}}
\begin{itemize}
\item See Etymology of \href{https://onoma.lib.byu.edu/index.php/LEHI}{\emph{Lehi}}, \href{https://onoma.lib.byu.edu/index.php/NEPHI}{\emph{Nephi}}, \href{https://onoma.lib.byu.edu/index.php/LAMAN}{\emph{Laman}}, \href{https://onoma.lib.byu.edu/index.php/LEMUEL}{\emph{Lemuel}}, \href{https://onoma.lib.byu.edu/index.php/SAM}{\emph{Sam}}, and \href{https://onoma.lib.byu.edu/index.php/SARIAH}{\emph{Sariah}}
\item See \href{https://onoma.lib.byu.edu/index.php/BOUNTIFUL}{Etymology of Bountiful}
\end{itemize}
\chapter{Chapter 1}
\printindex
\end{document}
